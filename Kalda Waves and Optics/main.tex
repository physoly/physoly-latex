\documentclass[11pt]{article}
\usepackage{titling}
\usepackage[english]{babel}
\usepackage[margin=0.6in]{geometry} % Page Dimension
\usepackage{physoly}

%%%%%%%%%%%%%%%%%%%%%%%%%%%%%%%%%%%%%%%%%%
%                PACKAGES                %  
%%%%%%%%%%%%%%%%%%%%%%%%%%%%%%%%%%%%%%%%%%

% Styling Choices
\setlength{\parskip}{\baselineskip}%

% Math
\usepackage{amsmath, amsthm, amssymb}
\usepackage[inline]{asymptote}
\usepackage{xcolor}
\usepackage{cancel}

% Allows for hyperlinking
\usepackage{hyperref}
\hypersetup{
    colorlinks=true,
    linkcolor=magenta,
}

% Blind Footnote
\newcommand\blfootnote[1]{%
  \makeatletter{\footnotetext{#1}\makeatother}
}

% Fancy Header
\usepackage{fancyhdr}
\pagestyle{fancy}
\lhead{Kalda Mechanics}
\rhead{\thepage}

% Coloured Boxes
\usepackage{xcolor}
\definecolor{border}{HTML}{004D4D}
\definecolor{hard}{HTML}{ffccb3}
\definecolor{easy}{HTML}{b3e6b3}
\definecolor{normal}{HTML}{f2f2f2}

% Syntax: \colorboxed[<color model>]{<color specification>}{<math formula>}
\newcommand*{\colorboxed}{}
\def\colorboxed#1#{%
  \colorboxedAux{#1}%
}
\newcommand*{\colorboxedAux}[3]{%
  % #1: optional argument for color model
  % #2: color specification
  % #3: formula
  \begingroup
    \colorlet{cb@saved}{.}%
    \color#1{#2}%
    \boxed{%
      \color{cb@saved}%
      #3%
    }%
  \endgroup
}

% Setup Gray Solution Boxes
\usepackage[breakable,many]{tcolorbox}
\newtcolorbox[auto counter]{solution}[1]{
    enhanced, breakable,
    arc=0pt,
    % colback=default, % Background color
    colframe=white, % Border Color
    coltitle=black, % Title Color
    fonttitle=\bfseries,
    title=\fcolorbox{border}{#1}{\textcolor{border}{pr} \bfseries \textcolor{border}{\thetcbcounter}.},
    attach title to upper,
    after title={\ },
    segmentation style={dashed, gray},
}

% Title and front page
\title{Solutions to Problems in Mechanics Handout by Jaan Kalda With detailed diagrams and walkthroughs}
% Author
\author{\textsc{Ashmit Dutta, QiLin Xue, Kushal Thaman, Arhaan Ahmad}}

\begin{document}
\begin{titlepage}
    \begin{center}
        \vspace*{1cm}
 
        \Huge
        \textbf{Solutions to Jaan Kalda's Problems in Waves and Optics}
 
        \vspace{0.5cm}
        \LARGE
        \textbf{With detailed diagrams and walkthroughs}
        
        \vspace{0.1cm}
        Edition 1.3.1
        
        \vspace{1.2cm}
 
          Rakshit, Ashmit Dutta, Aarush Gupta,  Kushal Thaman, QiLin Xue, \\Daniel Yang
        \vspace{10mm}
\begin{center}
\includegraphics[width=15cm]{title.png}
\end{center}
        \vfill
        
        \Large
        Updated
        \today
 
    \end{center}
\end{titlepage}
\newpage
%%%%%%%%%%%
% Preface %
%%%%%%%%%%%
\section*{Preface}
\vspace{-5mm}
\indent Jaan Kalda's \href{https://www.ioc.ee/~kalda/ipho/}{handouts} are beloved by physics students both in for a quick challenge, to students preparing for international Olympiads. As of writing, the current \href{https://www.ioc.ee/~kalda/ipho/waveopt.pdf}{waves and optics} handout has 19 unique problems and 2 main `ideas'. \textbf{Many of the problems had solutions at the back of the handout. This solutions manual attempts to solve the problems that don't have solutions}.

This solutions manual came as a pilot project from the online community at \url{artofproblemsolving.com}. Although there were detailed hints provided, full solutions have never been written. The majority of the solutions seen here were written on a private forum given to those who wanted to participate in making solutions. In an amazing show of an online collaboration, students from around the world came together to discuss ideas and methods and created what we see today.

\subsection*{Structure of The Solutions Manual}
\vspace{-5mm}
Each chapter in this solutions manual will be directed towards a section given in Kalda's mechanics handout. There are three major chapters: statics, dynamics, and revision problems. If you are stuck on a problem, cannot make progress even with the hint, and come here for reference, look at only the start of the solution, then try again. Looking at the entire solution wastes the problem for you and ruins an opportunity for yourself to improve.

\subsection*{Contact Us}
\vspace{-5mm}
Despite editing, there is almost zero probability that there are \textit{no} mistakes inside this book. If there are any mistakes, you want to add a remark, have a unique solution, or know the source of a specific problem, then please contact us at \url{hello@physoly.tech}. The most current and updated version can be found on our website \url{physoly.tech}

Please feel free to contact us at the same email if you are confused on a solution. Chances are that many others will have the same question as you.

\newpage
\section{Solutions to Unsolved Problems Problems}
\vspace{-5mm}
This section will consist of the solutions to unsolved problems which do not have solutions at the back of the handout. These consist of problems 2, 3, 9, 15, 16, 17, 18, and 19. These problems vary from easy high school to hard physics olympiad problems.

\begin{custom-simple}[Problem 2]
The wall blocks almost all the wave front of the original wave, leaving only two points in a cross-section perpendicular to the slits (see figure below). To be precise, these are actually segments, but their size is much smaller than the wavelength; so, from the point of view of wave propagation, the segments can be considered as points. According to the Huygens principle, two point sources of electromagnetic waves of wavelength $\lambda$ will be positioned into these two points ($A$ and $B$). The point sources radiate waves in all the directions, and we need to study the interference of this radiation. Let us study, what will be observed at a far-away screen where two parallel rays (drawn in figure) meet.
\vspace{3mm}

To begin with, it is quite easy to figure out, where are the intensity maxima and minima. Indeed, as it can be seen from the figure above, the optical path difference between the two rays is $\Delta l = a \sin\varphi$. The two rays add up constructively (giving rise to an intensity maximum) if the two waves arrive to the screen at the same phase, i.e. an integer number of wavelengths fits into the interval: $\Delta l = n\lambda$. Similarly, there is a minimum if the waves arrive in an opposite phase:
\[\sin\varphi_{\text{max}} = \frac{n\lambda}{a},\hspace{10pt} \sin\varphi_{\text{min}} = \left(n + \frac{1}{3}\right)\lambda/a\]
\end{custom-simple}
\begin{solution}{normal}
Over one complete oscillation of the voltage, the heat lost by the filament must equal the heat gained by it. Let the resistance of the filament be $R$. The heat gained by the filament is $\frac{U_1^2}{R}\frac{T}{2}$ (because the voltage is applied only for $\frac{T}{2}$). Let the rate at which heat is lost to the surrounding be $r$. The heat lost to the surroundings is $rT$ therefore 
$$rT = \frac{U_1^2}{R}\frac{T}{2} \implies r  =\frac{U_1^2}{2R}.$$
From $t = 0.5T$ to $T$, the heat lost takes the temperature from the maximum temperature to the minimum temperature, a change of $2 \Delta T$ (beware, the $\Delta T$ is the amplitude of the temperature while $T$ is time period of voltage oscillations). This implies that 
\[r\frac{T}{2} = 2mc \Delta T \implies \Delta T = \frac{U_1^2 T}{8Rmc}.\]
However, $R = \frac{\rho_{\text{el}}\ell}{A}$ and $m = \rho \ell A$, where $A$ is cross-section area of the wire. Substituting these values gives 
\[\Delta T = \frac{U_1^2 T}{8c \rho_{\text{el}} \rho \ell^2} = \frac{(17)^2(0.01)}{8(235)(9.95\times 10^{-7})(18200)(0.05)^2}=\boxed{33.8 \text{ K}}\]
\end{solution}
\begin{solution}{normal}
We shall use a property in geometry. Thales's theorem states that if A, B, and C are distinct points on a circle where the line $AC$ is a diameter, then the angle $\angle ABC$ is a right angle.
\begin{center}
\begin{asy}
import graph; size(8cm); 
real labelscalefactor = 0.5;
pen dps = linewidth(0.7) + fontsize(10); defaultpen(dps); /* default pen style */ 
pen dotstyle = black; /* point style */ 
real xmin = -5.2824091003930445, xmax = 5.600007360267558, ymin = -0.761082590475042, ymax = 4.9426400720203105;  /* image dimensions */

draw(circle((-0.125,2.25), 2.25),  linetype("2 2")); 
draw((-2,1)--(0,0)--(2,3));
draw((0,0)--(0,4.5), linetype("2 2")); 
draw((-2,1)--(2,3),linetype("2 2")); 
draw((-2,1)--(1.75,1), linetype("2 2")); 
draw((-1.18,4.24)--(0,0),linetype("2 2")); 
 /* dots and labels */
dot((-2,1),dotstyle); 
label("$A$", (-1.97,1.08), NE * labelscalefactor); 
dot((2,3),dotstyle); 
label("$C$", (2.03,3.08), NE * labelscalefactor); 
dot((0,0),dotstyle); 
label("$B$", (0.0315,-0.05), S * labelscalefactor); 
dot((0,1),linewidth(4pt) + dotstyle); 
label("$D$", (0.0315,1.0606), NE * labelscalefactor); 
dot((0,2),linewidth(4pt) + dotstyle); 
label("$O$", (0.0315,2.0629), NE * labelscalefactor); 
dot((-0.49,1.75),linewidth(4pt) + dotstyle); 
label("$E$", (-0.462,1.82), NE * labelscalefactor); 
clip((xmin,ymin)--(xmin,ymax)--(xmax,ymax)--(xmax,ymin)--cycle); 
\end{asy}
\end{center}
Therefore if we draw a circle where the corners of the two pillars form the ends of the diameter $AC$, the outline of the circle gives the possible locations the mass can be located as. Let the location of the mass be $B$. We wish to minimize the height of $B$ which so happens to be at the very bottom of the circle. Let $\angle EBD=\alpha$ such that $\angle ABE = 45^\circ$. Doing some angle tracing, we can verify that
$$\angle BAD=45^\circ-\alpha$$
Now since $OA$ and $OB$ are both the radius, that means $OAB$ is an isosceles triangle where:
$$\angle OAB = \angle ABO \implies 45^\circ-\alpha+\angle OAD = 45^\circ+\alpha \implies \angle OAD=2\alpha$$This angle relates the horizontal distance of the two pillars and the vertical distance of the two pillars through:
$$\tan OAD = \boxed{\tan(2\alpha) = \frac{h}{a}}$$
\tcbline
\textbf{Solution 2:} Let $y$ be the vertical distance between the mass and the top of the left pillar. Then let $b$ and $c$ be the horizontal distances between the mass and the left and right pillars, respectively, such that $a=b+c$. Doing basic angle tracing, we can see that:
$$b = \frac{y}{\tan(45-\alpha)}$$and

$$c = (h+y)\tan(45-\alpha)$$Adding them together and letting $\beta \equiv 45 - \alpha$ yields:

\begin{align*}
a &= b + c \\
a &= \frac{y}{\tan(\beta)} + (h+y)\tan(\beta) \\
a\tan(\beta) &= y + (h+y)\tan^2(\beta) \\
a\tan(\beta) - h\tan^2(\beta) &= y + y\tan^2(\beta) \\
\frac{\tan(\beta)(a-h\tan(\beta))}{1+\tan^2(\beta)} &= y
\end{align*}Doing a quick sanity check, this yields the correct answer of $y=a/2$ when $\beta = 45^\circ$ and $h=0$

We can simplify this further with a few trig identities. You can verify that the above expression is equivalent to

$$ y = \frac{1}{2}a\sin(90-2\alpha) - \frac{h}{2}\tan(45-\alpha) $$From the energy approach, the system will be in static equilibrium if no work is needed to rotate the system by a differential amount (change in potential energy is zero). This occurs when the gravitational potential energy is at a minimum or $y$ is minimized. Taking the derivative with respect to $\alpha$ we get:

\begin{align*}
\frac{dy}{d\alpha} &= \frac{1}{2}a\cos(90-2\alpha)(-2) - (2h\sin(45-\alpha))(\cos(45-\alpha)(-1) \\
0 &= -a\cos(90-2\alpha) + h\sin(90-2\alpha) \\
\frac{a}{h} &= \tan(90-2\alpha)
\end{align*}But since $\tan(90-2\alpha) = \cot(2\alpha)$, we can rewrite this to get:
$$\tan(2\alpha) = \frac{h}{a}$$
\end{solution}

\begin{custom-simple}[Problem 15]
Let $d$ be the thickness of the film. We find optical path difference ($\Delta x$) between the two rays shown in figure. We first note that that a phase difference of $\pi$ radians occurs at each boundary if refractive index of the medium in which light is travelling is less than the the refractive index of the medium which light strikes. As in both boundaries of the thin film a phase shift occurs this doesn't change the path difference or interference pattern.

Let $\alpha$ be angle of incidence of the rays for the lower boundary (i.e. boundary between thin film and glass plates). It is well known that in case of thin film interference the optical path difference is  $\Delta x = 2n_0d\cos \alpha$
\vspace{3mm}
\[0\leq \sin \theta \leq 1 \implies 0 \leq \sin \alpha \leq \frac{1}{n_0}\implies \sqrt{1-\frac{1}{n_0^2}} \leq \cos \alpha \leq 1\]
Therefore,
\[\Delta l_ {\text{min}} = 2n_0d\sqrt{1-\frac{1}{n_0^2}}, \Delta l_{\text {max}} = 2n_0 d\] 
Changing the view direction from vertical to horizontal changes the optical path length difference by $N\lambda$ (because during this process, $N$ interference maxima can be recorded, when the optical path length difference is an integer multiple of wavelength). Therefore,
$$2n_0d\left(1-\sqrt{1-\frac{1}{n_0^2}}\right) = N\lambda \implies \boxed {d =\frac{N\lambda}{2(n_0 - \sqrt{n_0^2 - 1})}}$$
\blfootnote{A derivation of optical path difference in thin film interference can be found \hyperlink{https://en.m.wikipedia.org/wiki/Thin-film_interference#Theory}{here}}
\end{custom-simple}
\begin{custom-simple}[Problem 16]
\begin{enumerate}
\item There is no light coming out from outlet $C_2$ because at the junction point a wave is generated in upper fiber in the same direction as the circular fiber (Huygens principle can be used to prove this).As energy at steady state is constant we can say that the total energy input $=$ total energy output, hence $I_{A_2}+I_{C_1}=I_0$. The result is a mirror image of the graph in the problem text that touches $I = 0$ at the bottom and $I = I_0$ at the top.
\item At this wavelength, all intensity $I_0$ comes out from fiber $C_1$ and intensity in fiber $B$ and intensity in direction $C_1$ should have ratio $(1-\alpha)/\alpha$. So $$I_B = \frac{I_0(1-\alpha)}{\alpha} = 99I_0$$
\item The intensity of light in fiber B is maximum when the light circulating in the fiber reaches the lower junction point in the same phase as the light from fiber A. Then the intensity going to fiber C is also maximum. Thus, fiber B must accommodate an integer of n wavelengths. From the graph we see that two successive resonances occur at wavelengths $\lambda_0 = 1660$ nm and $\lambda_1 = 1680$ nm. So $n\lambda_1 ’ = (n+1) \lambda_0 ’= l$, where $l$ is the desired length and the second resonant wavelength in the fiber is $\lambda_1 ’ = \lambda_0 ’\frac{\lambda_1 }{\lambda_0}$. From this relation we find $\frac{1}{n}= \frac{\lambda_1 ’}{\lambda_0 ’} - 1$ and
$$ l = \frac {\lambda_0 ’\lambda_1}{\lambda_1-\lambda_0} = \boxed{84\;\mathrm{\mu m}}$$
\end{enumerate}
\end{custom-simple}
\newpage
\begin{custom-simple}[Problem 17]
\begin{center}
    \includegraphics[width=12cm]{p17.png}
\end{center}
We find the images of light source in the mirrors. The light incident around $O$ can then be viewed as a superposition of the light emitted from the images. Let us take the line joining point $O$ and the point of intersection as x-axis and the line perpendicular to x-axis and passing through point $O$ as y-axis. Position of the image formed by the lower mirror is
$$y&=-2a\cos\alpha\sin\alpha, x=-2a\cos\alpha\cos\alpha + a$$
$$ \tan \phi &= \frac {y}{x} = \frac {2a\cos\alpha\cdot \sin\alpha}{2a\cos\alpha\cdot \cos\alpha + a}, \sin \phi = \frac {\tan \phi}{\sqrt {1 + (\tan \phi)^2}}$$
$$\sin \phi = \frac{\sin 2\alpha}{\sqrt {8\cos^2\alpha + 1}}$$
Let some point $M$ near point $O$ be the point of constructive interference. If we move up from this point by distance $d$, the path of light from the lower image would be increased by $d \sin\phi$ and the path of light from the upper image would be shortened by $d \sin\phi$. So we get a path difference $2d \sin\phi$ compared to $M$. This is again a point of constructive interference if $\lambda = 2d \sin\phi$. So we get the answer
$$\lambda = \frac{2d \sin 2\alpha}{\sqrt{8\cos^2\alpha + 1}}$$
\end{custom-simple}
\begin{custom-simple}[Problem 18]
\textbf{(a)} By energy conservation, the amplitudes of the output wave and input wave must be the same. The output fiber wave is formed by the sum of the wave in the fiber and the wave from the other fiber. According to the energy conservation, the amplitude of each component is $\sqrt 2$ times smaller than the original when the wave enters only one fiber. Thus, while the amplitude of the incoming waves was A, the outgoing resultant wave is in an expressible form.
$$A = \sqrt {\left(\frac {A}{\sqrt 2}\right)^2 \cdot 2 + 2\left(\frac {A}{\sqrt 2}\right) \left(\frac {A}{\sqrt 2}\right)\cos \phi}$$where $\phi$ is the phase shift. So $\cos (\phi/ 2) = 1/\sqrt 2$ and consequently $\phi = \frac {\pi}{2}$
\vspace{5mm}

\textbf{(b)} Phase difference between the $2$ fibers is $\pi$, the minima condition in fiber $1$ is $\Delta l = n\lambda$, where n is an integer. Writing this as $n = \frac{\Delta l}{\lambda}$ we see that
\[\frac{\Delta l}{\lambda_{\text{min}}}\geq n \geq \frac{\Delta l}{\lambda_{\text{max}}}\]thus $49.2 \geq n \geq 45.4$ and the values of $n$ to be sought are $46, 47, 48$ and $49$. The corresponding wavelengths are given by the formula $\lambda = \frac{n}{\Delta l}$; these are $612, 625, 638$ and $652\;\mathrm{nm}$
\end{custom-simple}
\newpage
\begin{solution}{hard}
In a rotating reference frame, we have that 
\[\vec\omega_3 = \vec\omega_1 + \vec\omega_2\]
where $\vec\omega_1$ is the angular velocity in the reference frame, $\vec\omega_2$ is the angular velocity of the body in the rotating reference, and $\vec\omega_3$ is that in the stationary frame. If you consider the reference point to be at infinity, then you find that the rotational motion of the disk becomes negligible. Therefore, we have that \blfootnote{This problem was found in the book 'Aptitude Test Problems in Physics' by S.S. Krotov.}
\[0 = \vec\omega_1 + \vec\omega_2\]
\[\boxed{\vec\omega_1 = -\vec\omega}\]
\end{solution}


\end{document}