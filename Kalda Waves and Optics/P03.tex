\begin{custom-simple}[Problem 3]
If the light is coherent, then the amplitude of the light emerging from each slit can individually be written as:
$$E=E_0\cos(\omega t + \phi)$$
where $\omega$ is the frequency and $\phi$ is the phase shift. An alternative way of writing this is:
$$E=E_0e^{i\omega t}e^{i\phi}$$
Its real component corresponds with the magnitude of the field that we measure at a certain time $t$. The complex number $e^{i\omega t}e^{i\phi}$ can be represented as a phasor which is essentially a vector with a constant magnitude of one that rotates in the complex plane at an angular frequency of $\omega$ (that is, it makes an angle $\omega t+\phi$ with the real axis). We want the sum of the three different amplitudes to sum up to zero, or:
$$E_1+E_2+E_3=E_0\cdot \left[e^{i\omega t}e^{i\phi_1}+e^{i\omega t}e^{i\phi_2}+e^{i\omega t}e^{i\phi_3}\right] = 0$$
Although this may look complicated, we can simplify it by treating them geometrically as phasors. To achieve an intensity minima of zero, we need three phasors such that their vector sum equals zero, which is equivalent to making a closed shape. Since they have the same magnitude, this shape must be an equilateral triangle. Without loss of generality, let $\phi_1=0$. This means $\phi_2=\phi_3=2\pi k + \frac{2\pi}{3}$ where $k$ is an integer.

The phase shift is defined as:
$$\frac{\phi}{2\pi} = \frac{(\text{path difference})}{\lambda} \implies k+ \frac{1}{3} = \frac{d\sin\theta}{\lambda}$$
where $d$ is the separation between two arbitrary slits and $\theta$ is the angle the light makes with the horizontal. Applying this for the slit separation distances in this problem, we have:
$$a = \left(k_1+\frac{1}{3}\right)\frac{\lambda}{\sin\theta_1}$$
$$b = \left(k_2+\frac{1}{3}\right)\frac{\lambda}{\sin\theta_2}$$
If we assume that $a,b \ll L$ where $L$ is the distance between the slits and the screen, then $\sin\theta_1=\sin\theta_2$. Taking the ratio, we get:
$$\frac{a}{b} = \frac{n}{m} \equiv \frac{3k_1+1}{3k_2+1}$$
This produces a minima of zero for any integer combinations of $k_1$ and $k_2$. We will prove that for each combination, $n-m$ is a multiple of three. We have:
$$3k_2+1 - (3k_2+1) = 3r \implies 3(k_2-k_1)=3r \implies k_2-k_1=r$$
where $r$ is an integer. Since $k_1$ and $k_2$ have to be integers, then $r$ must also be an integer, proving the statement.
\end{custom-simple}