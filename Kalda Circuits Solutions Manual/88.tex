\begin{solution}{normal}
First, disregard the capacitor and consider a current $I$ that flows through the toroidal ferromagnet. We can consider the ferromagnet to be made of two inductors (each consists of a half on each side). Because of this, there will be a mutual inductance $M$ present as well. It is known by fact 21 that $M$ cannot be greater than the geometric average of both inductors i.e. $M = \sqrt{L_1 L_2}$ is achieved for transformers when all the magnetic field lines created by the both coils have identical shapes. Let the inductance of one half be $L'$ such that by symmetry, the inductance of the other half is also $L^\prime$; therefore, $M = \sqrt{L^{\prime 2}} = L^{\prime}$. We find that the voltage induced on one half of the ferromagnet will follow $L^\prime \frac{\text{d}I}{\text{d}t} + M \frac{\text{d}I}{\text{d}t} = 2L^\prime \frac{\text{d}I}{\text{d}t}$. Thus, the total voltage induced will be two times this (as there are two halves) or $4L^\prime \frac{\text{d}I}{\text{d}t}$. This must be the same as the voltage regularly induced by the inductor $L\frac{\text{d}I}{\text{d}t}$ which calls for the equality $L\frac{\text{d}I}{\text{d}t} = 4L^\prime \frac{\text{d}I}{\text{d}t} \implies L^\prime = \frac{L}{4}$. 
\\
\\
Let us consider two clockwise current loops $I_1$ and $I_2$ respectively in the circuit. By idea 46, the total voltage on one half of the inductor will follow $\frac{V_0}{2} = i\omega \frac{L}{4}(I_1 + I_2)$. The capacitor will have a current $I_1 - I_2$ flowing through as opposed to the inductor having a current of $I_1 + I_2$. Thus, the voltage on the capacitor will be $V_0 = \frac{1}{i\omega C} (I_1 - I_2)$. By Kirchoff's voltage law, it is clear that $\sum_{\nu} V_\nu = 0$ which means $i \omega \frac{L}{4} (I_1 + I_2) + \frac{1}{i\omega C} (I_1 - I_2) = 0$. Rearranging tells us 
\[\omega^2 C\frac{L}{4} (I_1 + I_2) = I_1 - I_2.\]
In terms of $V_0$, we can then solve for $I_2$ as $V_0 \left(\frac{1}{\omega L} + \frac{\omega C}{4}\right)$.
\end{solution}