\documentclass[a4paper,11pt]{article}
\usepackage[english]{babel}
\usepackage[a4paper,margin=0.6in]{geometry} % Page Dimensions

%%%%%%%%%%%%%%%%%%%%%%%%%%%%%%%%%%%%%%%%%%
%                PACKAGES                %  
%%%%%%%%%%%%%%%%%%%%%%%%%%%%%%%%%%%%%%%%%%

% Styling Choices
\setlength{\parskip}{\baselineskip}%

% Math
\usepackage{amsmath, amsthm, amssymb}
\usepackage[inline]{asymptote}

% Allows for hyperlinking
\usepackage{hyperref}
\hypersetup{
    colorlinks=true,
    linkcolor=magenta,
}

% Fancy Header
\usepackage{fancyhdr}
\pagestyle{fancy}
\lhead{Kalda Mechanics}
\rhead{\thepage}

% Coloured Boxes
\usepackage{xcolor}
\definecolor{border}{HTML}{004D4D}
\definecolor{hard}{HTML}{ffccb3}
\definecolor{easy}{HTML}{b3e6b3}
\definecolor{normal}{HTML}{f2f2f2}

% Syntax: \colorboxed[<color model>]{<color specification>}{<math formula>}
\newcommand*{\colorboxed}{}
\def\colorboxed#1#{%
  \colorboxedAux{#1}%
}
\newcommand*{\colorboxedAux}[3]{%
  % #1: optional argument for color model
  % #2: color specification
  % #3: formula
  \begingroup
    \colorlet{cb@saved}{.}%
    \color#1{#2}%
    \boxed{%
      \color{cb@saved}%
      #3%
    }%
  \endgroup
}

% Setup Gray Solution Boxes
\usepackage[breakable,many]{tcolorbox}
\newtcolorbox[auto counter]{solution}[1]{
    enhanced, breakable,
    arc=0pt,
    % colback=default, % Background color
    colframe=white, % Border Color
    coltitle=black, % Title Color
    fonttitle=\bfseries,
    title=\fcolorbox{border}{#1}{\textcolor{border}{pr} \bfseries \textcolor{border}{\thetcbcounter}.},
    attach title to upper,
    after title={\ },
    segmentation style={dashed, gray},
}

% Book's title and subtitle
\title{\Huge \textbf{Solutions to Problems in Relativity Handout by Siim Ainsar \footnote{All solutions will refer to the 1.2 English version given at \newline\url{https://www.ioc.ee/~kalda/ipho/meh_ENG2.pdf}}} \\ \huge With detailed diagrams and walkthroughs}
% Author
\author{\textsc{Ashmit Dutta, QiLin Xue, Kushal Thaman, Arhaan Ahmed}}

\begin{document}

\maketitle

%%%%%%%%%%%
% Preface %
%%%%%%%%%%%
\subsection*{Preface}
\vspace{-5mm}

This solutions manual came as a sort of pilot project on the online database \url{artofproblemsolving.com}. Kalda's problems never did have solutions written for them, thus the idea of creating solutions to these problems were an interesting idea. The majority of the solutions seen here were written on a private forum given to those who wanted to participate in making solutions. It was an interesting idea and the forum was able to have a great start. International users from India, the US, and Canada came into the forums to give ideas and methods to create what we see today.

\subsection*{Structure of The Solutions Manual}
\vspace{-5mm}
Each chapter in this solutions manual will be directed towards a section given in Kalda's mechanics handout. There are three major chapters: statics, dynamics, and revision problems. At the final page, we have an acknowledgments chapter. Each solution will be as detailed as possible and will usually contain a remark located in the footnote of each page. If you are stuck on a problem, and come here for reference, look at only the starting of the problem. Looking at the entire solution wastes the problem for you and ruins an opportunity for yourself to improve. Some solutions will have more than one solution to the problem.

\subsection*{Contact Us}
\vspace{-5mm}
Despite editing, there is almost zero probability that there are \textit{no} mistakes inside this book. If there are any mistakes amiss, you want to add a remark, have a unique solution, or know the source of a specific problem, then please contact us at \url{solutionstokaldahandouts@gmail.com}. Furthermore, please feel free to contact us at the same email if you are confused on a solution. Chances are that many others have the same confusion as you.
\newpage
\section{Solutions to Statics Problems}

This section will consist of the solutions to problems from problem 1-23 of the handout. Statics is typically the analysis of objects not in motion. However, objects travelling at constant velocity or with a uniform acceleration can be treated as a statics problem with a frame of reference change. This usually involves balancing forces, torques, and more to achieve equilibrium. 

\begin{solution}{normal}
The photon moves in a zig zag where it hits the ceiling then rebounds back down. If the distance between the floor and ceiling is $l$, and the total time is $t$, then we find that we have a triangle shaped as 
\begin{center}
\begin{asy}
size(10cm);
draw((0,0)--(2,1)--(4,0)--cycle);
draw((2,0)--(2,1));
label("$l$", (2, 0.5), E);
label("$ct/2$", (1,0.7), N);
label("$vt$", (2,0), S);
\end{asy}
\end{center}
Therefore, we find that we have a right triangle with legs $l$, and $vt/2$ and hypotenuse $ct/2$. Using Pythagorean theorem, we have that 
\[\left(\frac{ct}{2}\right)^2 = \left(\frac{vt}{2}\right)^2 + l^2\]
Expanding both squares gives 
\[\frac{c^2t^2}{4} +- \frac{v^2t^2}{4} = l^2\]
Simplifying gives 
\[\boxed{t = \frac{4l^2}{c^2 - v^2}}\]
To verify this claim, we can consider what's happening in the first scenario. When the reference frame doesn't move with a velocity perpendicular to the velocity of the photon. The total distance the photon travels is $2l$, thus the total time in the first scenario is $t' = \frac{2l}{c}$. In this problem we found that 
\[t^2 = \frac{4l^2}{c^2 - v^2}.\]
Substituting our value for $t$ into this problem gives 
\[t^2 = \frac{c^2 t'^2}{c^2 - v^2}\implies t = t'\sqrt{\frac{c^2}{c^2 - v^2}} = \boxed{t'\frac{1}{\sqrt{1-v^2/c^2}}}.\]
From fact 1, we see that this is the same as if the time interval between two events happening at a stationary point is $t$, then in a reference frame where the speed of the point is $v$ the time interval is $\gamma t$ , where the Lorentz factor
\[\gamma = \frac{1}{\sqrt{1-v^2/c^2}}.\]
\end{solution}
\begin{custom-simple}[Problem 2]
The wall blocks almost all the wave front of the original wave, leaving only two points in a cross-section perpendicular to the slits (see figure below). To be precise, these are actually segments, but their size is much smaller than the wavelength; so, from the point of view of wave propagation, the segments can be considered as points. According to the Huygens principle, two point sources of electromagnetic waves of wavelength $\lambda$ will be positioned into these two points ($A$ and $B$). The point sources radiate waves in all the directions, and we need to study the interference of this radiation. Let us study, what will be observed at a far-away screen where two parallel rays (drawn in figure) meet.
\vspace{3mm}

To begin with, it is quite easy to figure out, where are the intensity maxima and minima. Indeed, as it can be seen from the figure above, the optical path difference between the two rays is $\Delta l = a \sin\varphi$. The two rays add up constructively (giving rise to an intensity maximum) if the two waves arrive to the screen at the same phase, i.e. an integer number of wavelengths fits into the interval: $\Delta l = n\lambda$. Similarly, there is a minimum if the waves arrive in an opposite phase:
\[\sin\varphi_{\text{max}} = \frac{n\lambda}{a},\hspace{10pt} \sin\varphi_{\text{min}} = \left(n + \frac{1}{3}\right)\lambda/a\]
\end{custom-simple}
\begin{solution}{normal}
Over one complete oscillation of the voltage, the heat lost by the filament must equal the heat gained by it. Let the resistance of the filament be $R$. The heat gained by the filament is $\frac{U_1^2}{R}\frac{T}{2}$ (because the voltage is applied only for $\frac{T}{2}$). Let the rate at which heat is lost to the surrounding be $r$. The heat lost to the surroundings is $rT$ therefore 
$$rT = \frac{U_1^2}{R}\frac{T}{2} \implies r  =\frac{U_1^2}{2R}.$$
From $t = 0.5T$ to $T$, the heat lost takes the temperature from the maximum temperature to the minimum temperature, a change of $2 \Delta T$ (beware, the $\Delta T$ is the amplitude of the temperature while $T$ is time period of voltage oscillations). This implies that 
\[r\frac{T}{2} = 2mc \Delta T \implies \Delta T = \frac{U_1^2 T}{8Rmc}.\]
However, $R = \frac{\rho_{\text{el}}\ell}{A}$ and $m = \rho \ell A$, where $A$ is cross-section area of the wire. Substituting these values gives 
\[\Delta T = \frac{U_1^2 T}{8c \rho_{\text{el}} \rho \ell^2} = \frac{(17)^2(0.01)}{8(235)(9.95\times 10^{-7})(18200)(0.05)^2}=\boxed{33.8 \text{ K}}\]
\end{solution}
\begin{solution}{normal}
We notice that the graph is quadratic so we can fit it to the equation
\begin{align*}
    \alpha &= \dfrac{\pi}{180}\left(-\dfrac{60}{49}(t-7)^2 + 60\right) \\
    &= -\dfrac{\pi}{147}(t-7)^2 + \pi/3
\end{align*}
where $\alpha$ is in radians and $t$ is in minutes. \vspace{3mm}

Since we know that the upward ascending velocity is constant, it is
\begin{align*}
    v_y &=L\alpha '(0) = 1000\left(\dfrac{14\pi}{147}\right) \\
    &= 299 \, \mathrm{m/min} = \boxed{4.99\, \mathrm{m/s}}
\end{align*}

The height is simply $$h = v_y t = \boxed{2000 \, \mathrm{m}}$$

At $t=7 \, \mathrm{min}$, the change in elevation angle is momentarily 0, which means that the velocity vector also points at 60 degrees. \vspace{3mm}

Thus we can get
$$v_x = v_y \tan (30^{\circ}) \approx \boxed{2.8 \, \mathrm{m/s}}$$
\blfootnote{You don't need the equation of the curve to perform calculations, but even without it, the answer can appear a bit off.

e.g. the initial slope you get could be:
$$4^\circ / 0.2 \text{ min} = 0.0698 \text{ rad} / 12 \text{ sec} = 0.00582 \text{ sec}^{-1}$$}
\end{solution}
\newpage
\begin{solution}{normal}
a) The heat flux (or energy flux) density is $\Phi = \frac{P}{S}$ and the thermal resistivity is:
$$\rho = \frac{1}{\Phi}\frac{dT}{dx} = \frac{S}{P}\frac{dT}{dx}$$
Separating variables, we have:
$$\Delta T = \int_0^d \frac{\rho P}{S} dx =11.7 \text{ K}$$
\vspace{3mm}

\noindent b) Again, we separate variables. This time however, our expression becomes:
$$\frac{\Delta T S}{P} = \int_0^\ell \rho(x) dx.$$
The integral can be approximated as the area under the curve. In this case, we can see that the average value is approximately $0.14\;\mathrm{K\cdot m/W}$ and then use this value to approximate the integral as a rectangle. Solving for $P$ from here gives us 
$$P = 1.8 \times 10^{-2} \text{ W}.$$

\end{solution}
\begin{solution}{normal}
According to Stefan-Boltzmann's Law, the power per unit area emitted from the surface of a blackbody at temperature $T$ is $\sigma T^4$. 

The total power emitted from the sun, considered a blackbody for the sake of the problem, is therefore,
$$P_\odot=\sigma T_\odot^4 (4\pi R_\odot^2)$$

Due to the inverse square law, the solar flux stays constant through any closed surface. The portion of energy that reaches the satellite is given by the ratio between the cross-sectional area of the satellite and the area of an imaginary sphere centered around the Sun with a radius of $L$ ($R_\oplus$ is the radius of the satellite)
$$
\gamma=
\frac{\pi R_\oplus^2}{4\pi L^2}=
\left(\frac{R_\oplus}{2L}\right)^2$$

According to Prevost's theory of exchange, in order to maintain thermal equilibrium, any object must emit the same energy as it receives. If this was not true, then it would continuously lose or gain energy until it is at equilibrium. Now let the emissivity and absorptivity factors of the satellite be $\epsilon$ and $a$. Since we know that these two parameters essentially have the same value at a particular wavelength, we have
$$P_\text{in} = P_\text{out}$$
By Stefan-Boltzmann Law, we have 
\[P_{\text{output}}=4\pi R_\oplus^2\sigma T^4\]
Equating $P_{\text{in}}$ to $P_{\text{out}}$ gives us
\begin{align*}
P_\text{output} &= P_\odot \gamma_\text{eff}\\
\epsilon_ \times 4\pi R_\oplus^2\sigma T^4 &= 
a \times \sigma T_\odot^4 (4\pi R_\odot^2)\left(\frac{R_\oplus}{2L}\right)^2 \\
\cancel\epsilon_ \times 4\pi R_\oplus^2\sigma T^4 &= 
\cancel a \times \sigma T_\odot^4 (4\pi R_\odot^2)\left(\frac{R_\oplus}{2L}\right)^2 \\
 T^4 &= 
 T_\odot^4 (R_\odot^2)\left(\frac{1}{2L}\right)^2 \\
 T^4 &= \frac{T_\odot^4 R_\odot^2}{4L^2}  \\
 T &= T_\odot \sqrt{\left(\frac{R_\odot}{2L}\right)}\\
\end{align*}
Plugging in our given constants gives us $\approx\boxed{290\;\mathrm{K}}$.

\end{solution}
\begin{solution}{normal}
The center of mass of the object can be calculated by treating the wheel as a superposition of two objects, one with positive density $\rho$ and one with negative mass density $-\rho$. Taking $r=0$ to be the center, the center of mass is:
\begin{align*}
r_\text{cm} &= \frac{\rho \pi R^2 (0) - \rho \pi \left(\frac{R}{2}\right)^2 \left(\frac{R}{3}\right)}{\rho\pi R^2 - \rho \pi\left(\frac{R}{2}\right)^2} \\
&= -\frac{\frac{1}{4} \left(\frac{R}{3}\right)}{1 - 1/4} \\
&= -R/9
\end{align*}
Therefore, when the normal force is zero, we have:
\begin{align*}
m\omega^2 (R/9) &= mg\\
\omega &= 3\sqrt{g/R}
\end{align*}And therefore the speed would be
$$\boxed{v = 3\sqrt{gR}}$$
\end{solution}

\begin{solution}{hard}
Draw a right trapezoid as follows:
\vspace{3mm}

We decompose $\vec{v}$ into parallel and perpendicular components, $\vec{v} = \vec{v_x} + \vec{v_y}$; let us mark points $A, B$ and $C$ so that
$AB = \vec{v_x}$ and $BC = \vec{v_y}$ (then, $AC = \vec{v}$). \vspace{3mm}

Next we mark points $D, E$ and $F$ so that $CD = \vec{v'_y} = \vec{v_y}$, $DE = -\vec{v_x}$, and $EF = 2\vec{u_x}$; then, $CF = \vec{v_y'} - \vec{v_x} + 2\vec{u_x} \equiv \vec{v'}$ and $AF = 2\vec{v_y} + 2\vec{u_x} \equiv 2\vec{u}$. \vspace{3mm}

Due to the problem conditions, $\angle ACF = 90^{\circ}$.
\begin{center}
    \begin{asy}
        size(8.5cm);
        import olympiad;
        pair A, B, C, D, E, F;
        A = (0, 0);
        B = (-1, 0);
        C = (-1, 1);
        D = (-1, 2);
        E = (1, 2);
        F = (0.5, 1);
        
        draw(A -- B -- C -- D -- E -- F -- cycle);
        draw(C -- F);
        draw(A -- C -- E, dotted);
        markscalefactor = 0.02;
        draw(anglemark(A, C, F));
        label("$\alpha$", (-0.93, 1.03), 4SE);
        draw(anglemark(F, A, C));
        label("$\alpha$", A, 5N);
        draw(anglemark(C, F, A));
        label("$\beta$", (0.4, 0.97), 2SW);
        markscalefactor = 0.01;
        draw(rightanglemark(C, B, A));
        label("A", A, SE);
        label("B", B, SW);
        label("C", C, W);
        label("D", D, NW);
        label("E", E, NE);
        label("F", F, SE);
        label("$-\vec{v_x}$", (D+ E)/2, N);
        label("$\vec{v_y'} = \vec{v_y}$", (C + D)/2, W);
        label("$\vec{v_y}$", (B + C)/2, W);
        label("$\vec{v_x}$", (B + A)/2, S);
        label("$2\vec{v_y} + 2\vec{u_x} = 2\vec{u}$", (0.9, 1), E);
        draw(F -- (0.5, 2), dotted);
        markscalefactor = 0.01;
        draw(rightanglemark(E, (0.5, 2), F));
        label("$2\vec{u_x}$", ((0.5, 2) + E)/2, N);
    \end{asy}
\end{center}
We now can see that $\Delta ACF$ is an isoceles triangle containing the lengths provided in the figure below. \vspace{3mm}

Let us also mark point $G$ as the centre of $AF$; then, $FC$ is both the median of the right trapezoid $ABDF$ (and hence, parallel to $AB$ and the $x-$axis), and the median of the triangle $ACF$.
\begin{center}
    \begin{asy}
        size(7cm);
        import olympiad;
        pair A, B, C, D, E, F;
        A = (0, 0);
        B = (-1, 0);
        C = (-1, 1);
        D = (-1, 2);
        E = (1, 2);
        F = (0.3, 1.2);
        
        draw(A -- C -- F -- cycle);
        markscalefactor = 0.02;
        draw(anglemark(A, C, F));
        label("$\alpha$", (-0.93, 1.03), 5SE);
        draw(anglemark(F, A, C));
        label("$\alpha$", A, 7NNW);
        draw(anglemark(C, F, A));
        label("$\beta$", (0.4, 0.95), 2SW);
        label("A", A, SE);
        label("C", C, W);
        label("F", F, NE);
        label("$\vec{v}$", (A+C)/2, SW);
        label("$\vec{u}$", (C+F)/2, N);
        label("$\vec{u}$", (A+F)/2, SE);
    \end{asy}
\end{center}
By splitting $\Delta ACF$ into it's median, we find, 
\[u\cos\alpha = \frac{v}{2}\implies u = \boxed{\frac{v}{2\cos\alpha}}.\]
For this to also happen, we see that $\beta = \boxed{180 - 2\alpha}$ because $\Delta ACF$ is an isoceles triangle.
\end{solution}
\begin{solution}{normal}
We shall use a property in geometry. Thales's theorem states that if A, B, and C are distinct points on a circle where the line $AC$ is a diameter, then the angle $\angle ABC$ is a right angle.
\begin{center}
\begin{asy}
import graph; size(8cm); 
real labelscalefactor = 0.5;
pen dps = linewidth(0.7) + fontsize(10); defaultpen(dps); /* default pen style */ 
pen dotstyle = black; /* point style */ 
real xmin = -5.2824091003930445, xmax = 5.600007360267558, ymin = -0.761082590475042, ymax = 4.9426400720203105;  /* image dimensions */

draw(circle((-0.125,2.25), 2.25),  linetype("2 2")); 
draw((-2,1)--(0,0)--(2,3));
draw((0,0)--(0,4.5), linetype("2 2")); 
draw((-2,1)--(2,3),linetype("2 2")); 
draw((-2,1)--(1.75,1), linetype("2 2")); 
draw((-1.18,4.24)--(0,0),linetype("2 2")); 
 /* dots and labels */
dot((-2,1),dotstyle); 
label("$A$", (-1.97,1.08), NE * labelscalefactor); 
dot((2,3),dotstyle); 
label("$C$", (2.03,3.08), NE * labelscalefactor); 
dot((0,0),dotstyle); 
label("$B$", (0.0315,-0.05), S * labelscalefactor); 
dot((0,1),linewidth(4pt) + dotstyle); 
label("$D$", (0.0315,1.0606), NE * labelscalefactor); 
dot((0,2),linewidth(4pt) + dotstyle); 
label("$O$", (0.0315,2.0629), NE * labelscalefactor); 
dot((-0.49,1.75),linewidth(4pt) + dotstyle); 
label("$E$", (-0.462,1.82), NE * labelscalefactor); 
clip((xmin,ymin)--(xmin,ymax)--(xmax,ymax)--(xmax,ymin)--cycle); 
\end{asy}
\end{center}
Therefore if we draw a circle where the corners of the two pillars form the ends of the diameter $AC$, the outline of the circle gives the possible locations the mass can be located as. Let the location of the mass be $B$. We wish to minimize the height of $B$ which so happens to be at the very bottom of the circle. Let $\angle EBD=\alpha$ such that $\angle ABE = 45^\circ$. Doing some angle tracing, we can verify that
$$\angle BAD=45^\circ-\alpha$$
Now since $OA$ and $OB$ are both the radius, that means $OAB$ is an isosceles triangle where:
$$\angle OAB = \angle ABO \implies 45^\circ-\alpha+\angle OAD = 45^\circ+\alpha \implies \angle OAD=2\alpha$$This angle relates the horizontal distance of the two pillars and the vertical distance of the two pillars through:
$$\tan OAD = \boxed{\tan(2\alpha) = \frac{h}{a}}$$
\tcbline
\textbf{Solution 2:} Let $y$ be the vertical distance between the mass and the top of the left pillar. Then let $b$ and $c$ be the horizontal distances between the mass and the left and right pillars, respectively, such that $a=b+c$. Doing basic angle tracing, we can see that:
$$b = \frac{y}{\tan(45-\alpha)}$$and

$$c = (h+y)\tan(45-\alpha)$$Adding them together and letting $\beta \equiv 45 - \alpha$ yields:

\begin{align*}
a &= b + c \\
a &= \frac{y}{\tan(\beta)} + (h+y)\tan(\beta) \\
a\tan(\beta) &= y + (h+y)\tan^2(\beta) \\
a\tan(\beta) - h\tan^2(\beta) &= y + y\tan^2(\beta) \\
\frac{\tan(\beta)(a-h\tan(\beta))}{1+\tan^2(\beta)} &= y
\end{align*}Doing a quick sanity check, this yields the correct answer of $y=a/2$ when $\beta = 45^\circ$ and $h=0$

We can simplify this further with a few trig identities. You can verify that the above expression is equivalent to

$$ y = \frac{1}{2}a\sin(90-2\alpha) - \frac{h}{2}\tan(45-\alpha) $$From the energy approach, the system will be in static equilibrium if no work is needed to rotate the system by a differential amount (change in potential energy is zero). This occurs when the gravitational potential energy is at a minimum or $y$ is minimized. Taking the derivative with respect to $\alpha$ we get:

\begin{align*}
\frac{dy}{d\alpha} &= \frac{1}{2}a\cos(90-2\alpha)(-2) - (2h\sin(45-\alpha))(\cos(45-\alpha)(-1) \\
0 &= -a\cos(90-2\alpha) + h\sin(90-2\alpha) \\
\frac{a}{h} &= \tan(90-2\alpha)
\end{align*}But since $\tan(90-2\alpha) = \cot(2\alpha)$, we can rewrite this to get:
$$\tan(2\alpha) = \frac{h}{a}$$
\end{solution}

\begin{solution}{normal}
\begin{center}
\begin{asy}
/* Geogebra to Asymptote conversion, documentation at artofproblemsolving.com/Wiki go to User:Azjps/geogebra */
import graph; size(5cm);
real labelscalefactor = 0.5; /* changes label-to-point distance */
pen dps = linewidth(0.7) + fontsize(10); defaultpen(dps); /* default pen style */
pen dotstyle = black; /* point style */
real xmin = -4.23213963380496, xmax = 7.371171204866221, ymin = -1.6441239198422954, ymax = 4.437435927619134; /* image dimensions */

/* draw figures */
draw((0,0)--(3,0), linetype("2 2"));
draw((3,0)--(3,4), linetype("2 2"));
draw((3,4)--(0,0), linetype("2 2"));
draw((0,0)--(2,0),EndArrow(6));
draw((2,0)--(2,2.7), EndArrow(6));
label("$\ell$",(1.37,2.6),SE*labelscalefactor);
label("$\mu N$",(0.823,0.4),SE*labelscalefactor);
label("$N$",(1.6,1.43),SE*labelscalefactor);
label("$h$",(3.15,2.22),SE*labelscalefactor);
label("$\sqrt{\ell^2-h^2}$",(1.2,-0.1),SE*labelscalefactor);
/* dots and labels */
dot((0,0),dotstyle);
label("$A$", (0,0), SW * labelscalefactor);
dot((3,0),dotstyle);
label("$B$", (3,0), NE * labelscalefactor);
dot((3,4),dotstyle);
label("$C$", (3,4), NE * labelscalefactor);
clip((xmin,ymin)--(xmin,ymax)--(xmax,ymax)--(xmax,ymin)--cycle);
/* end of picture */
\end{asy}
\end{center}
Consider what happens when the applied force approaches infinity. To maintain equilibrium, the friction force between the rod and the board must also increase. This friction force will also approach infinity. When dealing with large forces, we can ignore constant forces such as the weight of both the board and the rod.
\vspace{2mm}

As a result, since the weight of the rod is negligible we can pretend it's a mass-less rod. We also know that the forces at the ends of a massless rod will always point along the rod. For example, the force exerted on the rod by the board must point along the rod as well. The angle of this force is solely dependent on the friction coefficient $\mu_1$. Therefore:
$$\tan\alpha < \frac{\mu_1 N}{N} \implies \boxed{\mu_1>\frac{\sqrt{\ell^2-h^2}}{h}}$$
\tcbline
\textbf{Solution 2:} We want that when the board is on the verge of slipping then the rod should exert a larger force on the board (the rod should be pulled towards the board and not away from it). Consider the torque on the rod about the hinge point. We want that it should be clockwise when the block is on the verge of slipping.
\vspace{2mm}

Let the sum of normal reaction and friction force on the rod be $f$ (the normal points upwards and the friction points to the right). When the block is on the verge of slipping, the resultant makes an angle $\tan^{-1} \mu$ from the normal. We have:
$$\tau = mg\sin \alpha\frac{l}{2} + f \sin (\tan^{-1} \mu - \alpha)$$
considering clockwise torque to be positive. As the applied force on the block increases, $f$ also increases without bounds and because we want the torque to be clockwise no matter how much force we apply, the $mg$ term can be neglected. So
$$f \sin(\tan^{-1}\mu-\alpha) \ge 0$$
Since both $\tan^{-1}\mu$ and $\alpha$ are less than $90^{\circ}$, we can conclude that
$$\boxed{\tan^{-1} \mu \ge \alpha \implies \mu \ge \frac{\sqrt{l^2-h^2}}{h}}$$
\end{solution}

\begin{solution}{easy}
We will use a virtual work approach.\footnote{If you are unfamiliar with virtual work, refer to the explanation in Kalda's handout, or to this \href{http://docshare04.docshare.tips/files/26737/267376365.pdf}{pdf}.} In a static situation, the net force will be zero and as a result the potential energy will be at a minimum. Any slight displacement will create no change to the potential energy in first order.
\vspace{2mm}

Consider what happens when the mass is lowered by a distance $dh$. The potential energy would drop by $-mg dh$. The distance between hinges would each increase by $dh/3$ to compensate for the length increase. This means the string gets stretched by $dh/3$. The energy stored thus is:$$T dh/3$$Setting these changes to zero gives:
$$-mg dh + T dh/3 = 0 \implies \boxed{T = 3mg}$$
\end{solution}

\begin{solution}{normal}
\begin{center}
    \begin{asy}
        unitsize(3cm);
        real a = 40*pi/180;
        filldraw((-1.5,0)--(0,1.5*sin(a))--(0.03*sin(a),1.5*sin(a)-0.03*cos(a))--(-1.5+0.03*sin(a),-0.03*cos(a))--cycle,white);
        filldraw((1.5,0)--(-0.8,2.3*sin(a))--(-0.8-0.03*sin(a),2.3*sin(a)-0.03*cos(a))--(1.5-0.03*sin(a),-0.03*cos(a))--cycle,white);
        filldraw((-2,0)--(2,0)--(2,-0.05)--(-2,-0.05)--cycle);
        draw(arc((-1.5,0),0.3,0,27));
        draw(arc((1.5,0),0.3,180,153));
        label("$\alpha$",(-1.1,0.1));
        label("$\alpha$",(1.1,0.1));
        filldraw(circle((-0.75,1.49),0.03));
        label("$A$",(-0.9,1.49));
        filldraw(circle((-0.1,0.94),0.03));
        label("$B$",(-0.23,0.94));
        draw("$x$",(-0.75,0.85)--(-0.1,0.85),Arrows,Bars,PenMargins);
    \end{asy}
\end{center}

In the reference frame of ball A, ball B accelerates to the left with
$$a_B=2g\sin\alpha\cos\alpha$$

We can find that the initial length $|AB|$ is
$$\dfrac{g\left(t_1^2-t_2^2\right)\sin\alpha}{2}$$

Therefore,
$$x=\dfrac{g\left(t_1^2-t_2^2\right)\sin\alpha\cos\alpha}{2}$$

Since there is no relative acceleration in the y-direction, we need
\begin{align*}
\dfrac{a_Bt^2}{2}&=x\\
gt^2\sin\alpha\cos\alpha&=\dfrac{g\left(t_1^2-t_2^2\right)\sin\alpha\cos\alpha}{2}\\
t&=\boxed{\sqrt{\dfrac{t_1^2-t_2^2}{2}}}
\end{align*}
\tcbline

\textbf{Solution 2:} Each ball will accelerate with the same acceleration down their platform, meaning that they will travel the same distance in the same timeframe. \vspace{3mm}

Let $x$ be the distance traveled by the individual balls and $k$ be the distance between the two balls. Let the height of the ball at point $A$ be $h_1$ and the height of the ball at point $B$ be $h_2$. \vspace{3mm}

If you draw a diagram you will find that there is a triangle formed by the position of the two balls and the intersection of the planks. The lengths of the triangle are $x, x - \dfrac{h_1 - h_2}{\sin{\alpha}}, k$.

\begin{center}
    \begin{asy}
        unitsize(4cm);
        real a = 40*pi/180;
        filldraw((-1.5,0)--(0,1.5*sin(a))--(0.03*sin(a),1.5*sin(a)-0.03*cos(a))--(-1.5+0.03*sin(a),-0.03*cos(a))--cycle,white);
        filldraw((1.5,0)--(-0.8,2.3*sin(a))--(-0.8-0.03*sin(a),2.3*sin(a)-0.03*cos(a))--(1.5-0.03*sin(a),-0.03*cos(a))--cycle,white);
        filldraw((-2,0)--(2,0)--(2,-0.05)--(-2,-0.05)--cycle);
        draw(arc((-1.5,0),0.3,0,27));
        draw(arc((1.5,0),0.3,180,153));
        label("$\alpha$",(-1.1,0.1));
        label("$\alpha$",(1.1,0.1));
        filldraw(circle((-0.44,1.29),0.03));
        label("$A$",(-0.33,1.35));
        filldraw(circle((-0.41,0.74),0.03));
        label("$B$",(-0.53,0.74));
        filldraw(circle((-0.01,1.5*sin(a)-0.02),0.02));
        label("$O$",(-0.01,0.85));
        pair A = (-0.44,1.29);
        pair B = (-0.41,0.74);
        pair O = (-0.01,1.5*sin(a)-0.02);
        draw(A--B--O--cycle);
        label("$x$",(B+O)/2+(0.05,-0.08));
        label("$k$",(A+B)/2+(-0.07,0));
        label("$\displaystyle{\frac{h_1-h_2}{\sin\alpha}-x}$",(A+O)/2+(0.3,0.1));
    \end{asy}
\end{center}

By the Law of Cosines, we have
\[k^2 = x^2 + \left(\frac{h_1 - h_2}{\sin{\alpha}}-x\right)^2 - 2x\left(\frac{h_1 - h_2}{\sin{\alpha}}-x\right) \cos(2\alpha)\]

Let $\beta = \dfrac{h_1 - h_2}{\sin{\alpha}}\;$ for simplicity.\vspace{3mm}

Simplifying the expression, we get that
\[k(x) = \sqrt{2x^2(1+\cos\left(2\alpha\right)) - 2x\beta (1+\cos\left(2\alpha\right)) + \beta^2}\]

After taking the derivative of the quadratic and setting it equal to zero, we get that
$$x_m = -\dfrac{B}{2A}=\dfrac{\beta}{2}$$

Using acceleration along the ramp we can also find that
\begin{align*}
\dfrac{h_1}{\sin\alpha}&=\dfrac{gt_1^2\sin\alpha}{2}\;\;\;\;\;\;\;\;\; \dfrac{h_2}{\sin\alpha}=\dfrac{gt_2^2\sin\alpha}{2}\\
x_m&=\dfrac{gt_m^2\sin\alpha}{2}=\dfrac{h_1-h_2}{2\sin\alpha}
\end{align*}
Plugging in everything we find that
\begin{align*}
t_m&=\sqrt{\dfrac{h_1-h_2}{g\sin^2\alpha}}\\
&=\sqrt{\dfrac{\dfrac{gt_1^2\sin^2\alpha}{2}-\dfrac{gt_2^2\sin^2\alpha}{2}}{g\sin^2\alpha}}\\
&=\boxed{\sqrt{\dfrac{t_1^2-t_2^2}{2}}}
\end{align*}
\end{solution}
\begin{solution}{normal}
Since $H\ll L$, the curvature of the rope is very small which means that we can approximate the section that is above the ground as a straight line. Furthermore, the angle between the tangent to the rope and horizon remains everywhere small. Now, consider the following diagram assuming that the mass density of the rope is $\lambda$:
\begin{center}
\begin{asy}
size(10cm);
draw((0,0) -- (2, 0));
draw((0, 0) -- (-0.5, 1));
dot((-0.25, 0.5));
draw((-0.25, 0.5) -- (-0.25, 0.1), arrow=Arrow(4));
label("$mg$", (-0.25, 0.1), W);
label("$H$", (-0.5, 0.5), W);
label("$\ell$", (-0.25, 0), S);
draw((-0.5, 1) -- (-0.5, 0) -- (0, 0), dashed);
draw((-0.55, 1.1) -- (-0.75, 1.5), arrow=Arrow(4));
label("$F$", (-0.75, 1.5), W);
draw((1, 0) -- (1, 0.25), arrow=Arrow(4));
label("$N$", (1, 0.25), N);
label("$\lambda (L - \ell) g$", (0.9, -0.25), S);
draw((0.9, 0) -- (0.9, -0.25), arrow=Arrow(4));
draw((1.8, -0.1) -- (2.2, -0.1), arrow=Arrow(4));
label("$f$", (2.2, -0.1), E);
\end{asy}
\end{center}
The mass of the rope that is on the ground is given by $\lambda (L - \ell)$ where $\ell$ represents the horizontal part of the rope that is above the ground (as shown in the picture). Since the angle is small, we can assume that $\ell$ approximately represents the total length of the part of the rope that is above the ground. Since the weight of this section of the rope balances the normal force $N$, this then means that the frictional force $f = \mu N = \mu \lambda (L - \ell) g$. By using a torque balance, we can then write that 
\[\lambda \ell g \frac{\ell}{2} = fH = \mu \lambda (L - \ell) gH.\]
Cancelling factors then yields a quadratic which has a solution of 
\[\frac{\ell^2}{2} = \mu (L - \ell) H\implies \ell = \sqrt{2HL\mu + \mu^2 H^2} - \mu H \approx \sqrt{2HL\mu} - \mu H \approx 7.2\;\mathrm{m}.\]
\end{solution}

\begin{solution}{normal}
\begin{center}
    \begin{asy}
        unitsize(3cm);
        pair O = (0,0);
        pair A = (1,1);
        pair B = (2,-0.5);
        draw(O--A--A+B--B--O--B--B+(1,0.3)--A+B);
        draw(arc(B,0.3,16,45));
        label("$\alpha$",B+(0.38,0.22));
        draw(B/2-(1,0)--B/2,arrow=Arrow());
        draw(arc(B/2,0.5,180,166));
        label("$\beta$",B/2-(0.8,-0.08));
        draw(ellipse(B/2-(1.2,0.05),0.2,0.2/3));
        filldraw(B/2-(1,0.05)--B/2-(1.4,0.05)--B/2-(1.4,-0.05)--B/2-(1,-0.05)--cycle,gray(0.949),invisible);
        draw(ellipse(B/2-(1.2,-0.05),0.2,0.2/3));
        draw(B/2-(1,-0.05)--B/2-(1,0.05));
        draw(B/2-(1.4,-0.05)--B/2-(1.4,0.05));
        label("$v_0$",B/2+(-0.6,-0.1));
    \end{asy}
\end{center}

When on the plane, the puck experiences no change in its x-velocity, which is
$$v_0\cos\beta=5\;\text{m/s}$$

However, it experiences an acceleration parallel to the plane with magnitude
$$a=g\sin\alpha$$

We note from the trajectory given that the puck drops $2.5\;\text{m}$ below the apex of its trajectory while undergoing a horizontal displacement of $x=5\;\text{m}$.

The time it takes to complete this motion is
$$t=\dfrac{x}{v_0\cos\beta}=1\;\text{s}$$

Therefore, we have that
\begin{align*}
\dfrac{gt^2\sin\alpha}{2}&=2.5\\
\sin\alpha&=\dfrac{5}{gt^2}\\
\alpha&\approx \boxed{30\degree}
\end{align*}
\end{solution}
\begin{custom-simple}[Problem 15]
Let $d$ be the thickness of the film. We find optical path difference ($\Delta x$) between the two rays shown in figure. We first note that that a phase difference of $\pi$ radians occurs at each boundary if refractive index of the medium in which light is travelling is less than the the refractive index of the medium which light strikes. As in both boundaries of the thin film a phase shift occurs this doesn't change the path difference or interference pattern.

Let $\alpha$ be angle of incidence of the rays for the lower boundary (i.e. boundary between thin film and glass plates). It is well known that in case of thin film interference the optical path difference is  $\Delta x = 2n_0d\cos \alpha$
\vspace{3mm}
\[0\leq \sin \theta \leq 1 \implies 0 \leq \sin \alpha \leq \frac{1}{n_0}\implies \sqrt{1-\frac{1}{n_0^2}} \leq \cos \alpha \leq 1\]
Therefore,
\[\Delta l_ {\text{min}} = 2n_0d\sqrt{1-\frac{1}{n_0^2}}, \Delta l_{\text {max}} = 2n_0 d\] 
Changing the view direction from vertical to horizontal changes the optical path length difference by $N\lambda$ (because during this process, $N$ interference maxima can be recorded, when the optical path length difference is an integer multiple of wavelength). Therefore,
$$2n_0d\left(1-\sqrt{1-\frac{1}{n_0^2}}\right) = N\lambda \implies \boxed {d =\frac{N\lambda}{2(n_0 - \sqrt{n_0^2 - 1})}}$$
\blfootnote{A derivation of optical path difference in thin film interference can be found \hyperlink{https://en.m.wikipedia.org/wiki/Thin-film_interference#Theory}{here}}
\end{custom-simple}
\begin{custom-simple}[Problem 16]
\begin{enumerate}
\item There is no light coming out from outlet $C_2$ because at the junction point a wave is generated in upper fiber in the same direction as the circular fiber (Huygens principle can be used to prove this).As energy at steady state is constant we can say that the total energy input $=$ total energy output, hence $I_{A_2}+I_{C_1}=I_0$. The result is a mirror image of the graph in the problem text that touches $I = 0$ at the bottom and $I = I_0$ at the top.
\item At this wavelength, all intensity $I_0$ comes out from fiber $C_1$ and intensity in fiber $B$ and intensity in direction $C_1$ should have ratio $(1-\alpha)/\alpha$. So $$I_B = \frac{I_0(1-\alpha)}{\alpha} = 99I_0$$
\item The intensity of light in fiber B is maximum when the light circulating in the fiber reaches the lower junction point in the same phase as the light from fiber A. Then the intensity going to fiber C is also maximum. Thus, fiber B must accommodate an integer of n wavelengths. From the graph we see that two successive resonances occur at wavelengths $\lambda_0 = 1660$ nm and $\lambda_1 = 1680$ nm. So $n\lambda_1 ’ = (n+1) \lambda_0 ’= l$, where $l$ is the desired length and the second resonant wavelength in the fiber is $\lambda_1 ’ = \lambda_0 ’\frac{\lambda_1 }{\lambda_0}$. From this relation we find $\frac{1}{n}= \frac{\lambda_1 ’}{\lambda_0 ’} - 1$ and
$$ l = \frac {\lambda_0 ’\lambda_1}{\lambda_1-\lambda_0} = \boxed{84\;\mathrm{\mu m}}$$
\end{enumerate}
\end{custom-simple}
\begin{custom-simple}[Problem 17]
\begin{center}
    \includegraphics[width=12cm]{p17.png}
\end{center}
We find the images of light source in the mirrors. The light incident around $O$ can then be viewed as a superposition of the light emitted from the images. Let us take the line joining point $O$ and the point of intersection as x-axis and the line perpendicular to x-axis and passing through point $O$ as y-axis. Position of the image formed by the lower mirror is
$$y&=-2a\cos\alpha\sin\alpha, x=-2a\cos\alpha\cos\alpha + a$$
$$ \tan \phi &= \frac {y}{x} = \frac {2a\cos\alpha\cdot \sin\alpha}{2a\cos\alpha\cdot \cos\alpha + a}, \sin \phi = \frac {\tan \phi}{\sqrt {1 + (\tan \phi)^2}}$$
$$\sin \phi = \frac{\sin 2\alpha}{\sqrt {8\cos^2\alpha + 1}}$$
Let some point $M$ near point $O$ be the point of constructive interference. If we move up from this point by distance $d$, the path of light from the lower image would be increased by $d \sin\phi$ and the path of light from the upper image would be shortened by $d \sin\phi$. So we get a path difference $2d \sin\phi$ compared to $M$. This is again a point of constructive interference if $\lambda = 2d \sin\phi$. So we get the answer
$$\lambda = \frac{2d \sin 2\alpha}{\sqrt{8\cos^2\alpha + 1}}$$
\end{custom-simple}
\begin{custom-simple}[Problem 18]
\textbf{(a)} By energy conservation, the amplitudes of the output wave and input wave must be the same. The output fiber wave is formed by the sum of the wave in the fiber and the wave from the other fiber. According to the energy conservation, the amplitude of each component is $\sqrt 2$ times smaller than the original when the wave enters only one fiber. Thus, while the amplitude of the incoming waves was A, the outgoing resultant wave is in an expressible form.
$$A = \sqrt {\left(\frac {A}{\sqrt 2}\right)^2 \cdot 2 + 2\left(\frac {A}{\sqrt 2}\right) \left(\frac {A}{\sqrt 2}\right)\cos \phi}$$where $\phi$ is the phase shift. So $\cos (\phi/ 2) = 1/\sqrt 2$ and consequently $\phi = \frac {\pi}{2}$
\vspace{5mm}

\textbf{(b)} Phase difference between the $2$ fibers is $\pi$, the minima condition in fiber $1$ is $\Delta l = n\lambda$, where n is an integer. Writing this as $n = \frac{\Delta l}{\lambda}$ we see that
\[\frac{\Delta l}{\lambda_{\text{min}}}\geq n \geq \frac{\Delta l}{\lambda_{\text{max}}}\]thus $49.2 \geq n \geq 45.4$ and the values of $n$ to be sought are $46, 47, 48$ and $49$. The corresponding wavelengths are given by the formula $\lambda = \frac{n}{\Delta l}$; these are $612, 625, 638$ and $652\;\mathrm{nm}$
\end{custom-simple}
\newpage
\begin{solution}{hard}
In a rotating reference frame, we have that 
\[\vec\omega_3 = \vec\omega_1 + \vec\omega_2\]
where $\vec\omega_1$ is the angular velocity in the reference frame, $\vec\omega_2$ is the angular velocity of the body in the rotating reference, and $\vec\omega_3$ is that in the stationary frame. If you consider the reference point to be at infinity, then you find that the rotational motion of the disk becomes negligible. Therefore, we have that \blfootnote{This problem was found in the book 'Aptitude Test Problems in Physics' by S.S. Krotov.}
\[0 = \vec\omega_1 + \vec\omega_2\]
\[\boxed{\vec\omega_1 = -\vec\omega}\]
\end{solution}

\begin{solution}{normal}
Note that for a parabola with equation $x^2=4p(z-k)$, the focus is located at $(0,k+p)$\vspace{3mm}

In problem 19, we have the equation
$$z\leq\dfrac{v_0^2}{2g}-\dfrac{gx^2}{2v_0^2}$$
and with some manipulation we obtain
$$x^2=\dfrac{v_0^4}{g^2}-\dfrac{2v_0^2}{g}z.$$
Factoring the equation gives
$$x^2=-\dfrac{2v_0^2}{g}\left(z-\dfrac{v_0^2}{2g}\right)$$
$$x^2=4\left(-\dfrac{v_0^2}{2g}\right)\left(z-\dfrac{v_0^2}{2g}\right)$$
Therefore, the focus of the parabola is at
$$\left(0,\dfrac{v_0^2}{2g}-\dfrac{v_0^2}{2g}\right)=(0,0).$$
In problem 19, we assumed that the cannon was located at $(0,0)$, and so we are done.
\end{solution}
\begin{solution}{normal} \textbf{a)} Let us assume that the temperature stays roughly constant. This means that the sublimation rate is also constant and exerts some pressure $p$ on the vapour. We know that the saturation vapour pressure $p_0$ is defined such that the rate of sublimation = rate of deposition. This means that the pressure exerted by the sublimation is also $p_0$. Therefore the force is:
$$p_0A=Ma \implies a = \boxed{\frac{M}{p_0 A}}$$
\vspace{3mm}

\noindent \textbf{b)} Both evaporation and condensation apply the same pressure at saturation ($p_0/2$, to be exact), but since the particles escape never to come back (because $\lambda\gg\text{the length of the vessel}$), there is no condensation and thus only half the pressure is applied. Therefore, 
\[\frac{p_0}{2}A = Ma\implies a = \boxed{\frac{p_0 A}{2M}}.\]

\end{solution}
\begin{solution}{hard} \textbf{a)} First off, we find the pressure at $20^\circ\;\mathrm{C}$ on the graph. At this point, the pressure is approximately given by $2.3\;\mathrm{kPa}$. We are told that the relative humidity is $90\%$ which means that the relative pressure is given by \[2.3\;\mathrm{kPa} \cdot 0.9 = 2.07\;\mathrm{kPa}.\]
The temperature on the graph when it is approximately $2.07\;\mathrm{kPa}$ is around $18.5^\circ\;\mathrm{C}$. This then tells us that the temperature difference is 
\[20^\circ\;\mathrm{C} - 18.5^\circ\;\mathrm{C} = \boxed{1.5^\circ\;\mathrm{C}}\]
\vspace{3mm}

\noindent \textbf{b)} We are given the equations
\begin{align*}
Q_c &= a(T_0 - T)\\
Q_e &= b[p_s(T) - p_a]
\end{align*}
Dividing these two equations through gives us 
\[\frac{Q_c}{Q_e} = \frac{a}{b}\frac{T_0 - T}{p_s (T) - p_a}\]
from here, we know imediately that $a/b = 65\;\mathrm{Pa/K}$ and $T_0 = 20^\circ\;\mathrm{C}$. Because $r = 0$, then $p_a = 0$, and because $r=0$, then $Q_c = Q_e$. Therefore, our new equation is
\[1 = 65\frac{20 - T}{p_s(T)}\implies p_s (T) = 65 (20 - T).\]
From here, we find the intersection point with this line is $(6.5, 0.87)$, which implies the temperature is $\boxed{6.5^\circ\;\mathrm{C}}$.
\vspace{3mm}

\noindent \textbf{c)} In steady state, we have that 
\[Q_c =Q_e\implies \frac{a}{b}(T_0 - T) = p_s(T) - p_a.\]
Substituting  $a/b = 65\;\mathrm{Pa/K}$ and $T_0 = 20^\circ\;\mathrm{C}$ and $T\approx 2300r\;\mathrm{kPa}$ where $r$ is the relative humidity gives us 
\[65 (20 - T) = p_s (T) - 2300r.\]
\begin{itemize}
\item When $r = 1$, we have the equation the line as
\[p_s = 3600 - 65T.\]
The intersection of this line with the given curve is at $T = 20^{\circ}\;\mathrm{C}$ and $p_s = 2300\;\mathrm{Pa}$.
\item When $r = 0.8$, we have the equation of the line as 
\[p_s = 3140 - 65T.\]
The intersection of this line with the given curve is at $T = 18.75^{\circ}\;\mathrm{C}$ and $p_s = 2000\;\mathrm{Pa}$.
\end{itemize}
Since $p\propto r$ and we have the values of pressure at two different values of $r$, we can find a linear relation between pressure and relative humidity to get the equation
\[p_s = 1500r + 800\implies 65\Delta T = 800 (1 - r)\implies \Delta T = \boxed{12.3 (1 - r)}.\]
\vspace{3mm}

\noindent \textbf{d)} For the boundary condition, heat dissipated through evaporation. Therefore, 
\[k\frac{dT}{dt} = b(p_s - 2300r) = 800(1 - r) \implies \dot{T}\propto 1 - r.\]
We then see that 
\[\frac{\dot{T}_{80}}{\dot{T}_{35}} = \frac{1 - 0.8}{1 - 0.35} = \boxed{4}.\]
\end{solution}
\begin{solution}{normal}
\begin{center}
    \begin{asy}
        import olympiad;
        unitsize(2.5cm);
        filldraw((0,0)--(5,0)--(5,0.2)--(0,0.2)--cycle,mediumgrey);
        filldraw((0,-4.5)--(5,-4.5)--(5,-4.3)--(0,-4.3)--cycle,mediumgrey);
        draw((2.5,0)--(3.2,0),arrow=Arrow(),red);
        draw((2.5,-4.3)--(1.55,-4.3),arrow=Arrow(),blue);
        label("$v_1$",(3.4,-0.1));
        label("$v_2$",(1.4,-4.2));
        filldraw((2.5,0)..(2.8,-0.2)..(3.1,-0.3)..(3.3,-0.4)..(3.5,-0.5)..(3.7,-0.8)..(3.9,-1.1)..(4,-1.3)..(4.1,-1.5)..(4.2,-1.8)..(4.3,-2.2)..(4.4,-2.5)..(4.5,-2.7)..(4.6,-2.8)..(4.5,-3.3)..(4.3,-3.2)..(4.1,-3.1)..(3.9,-3)..(3.7,-2.9)..(3.5,-2.8)..(3.2,-2.7)..(3.1,-2.7)..(3,-2.9)..(3.2,-3.2)..(3.2,-3.4)..(3.1,-3.8)..(3,-3.9)..(2.8,-4.2)..(2.5,-4.3)..(2.2,-4)..(1.9,-3.9)..(1.6,-3.8)..(1.4,-3.7)..(1.2,-3.6)..(1.1,-3.5)..(1,-3.4)..(0.9,-3.1)..(0.9,-3)..(1,-2.8)..(1.2,-2.5)..(1.3,-2.2)..(1.1,-1.9)..(0.9,-1.7)..(0.7,-1.6)..(0.5,-1.5)..(0.4,-1.3)..(0.45,-1)..(0.6,-0.8)..(0.7,-0.8)..(1,-0.9)..(1.3,-0.8)..(1.8,-0.5)..(2.1,-0.2)..(2.4,-0.02)..cycle,lightgrey);
        draw((2.5,0)--(2.5,-4.3),dashed);
        draw((3.2,0)--(1.55,-4.3),dashed);
        dot((2.5,-1.8));
        draw(scale(0.5)*rightanglemark((5,-0.1),(5,0),(5.2,0)));
        draw(scale(0.5)*rightanglemark((5,-0.1),(5,-8.6),(4.8,-8.6)));
        label("$O$",(2.8,-1.8));
        label("$l_1$",(2.3,-0.9));
        label("$l_2$",(2.7,-3));
        draw(circle((2.5,-1.8),1.8),dotted);
        draw(arc((2.5,-1.8),1.8,-41,-10),red);
        draw(arc((2.5,-1.8),1.8,214,292),red);
        draw(arc((2.5,-1.8),1.8,174,150),red);
        dot((2.5,0),red);
        draw(circle((2.5,-1.8),2.5),dotted);
        draw(arc((2.5,-1.8),2.5,-36.5,-28),blue);
        dot((2.5,-4.3),blue);
    \end{asy}
\end{center}
We know by idea 33 that the instantaneous axis of rotation $O$ of the object exists. \vspace{3mm}

Let $l_1$ and $l_2$ be the distance from $O$ to the top and bottom boards, respectively. \vspace{3mm}

In fact, we have that $$\dfrac{l_1}{l_2}=\dfrac{|v_1|}{|v_2|}$$

By the properties of the instantaneous axis of rotation, we know that all points with speed $|v_1|$ lie on a circle centered at $O$ with radius $l_1$, and all points with speed $|v_2|$ lie on a circle centered at $O$ with radius $l_2$.
\end{solution}
\begin{solution}{normal}
\begin{center}
    \begin{asy}
    size(5cm);
    draw(circle((0,0), 1));
draw((0,1)--(1.5,1), arrow=Arrow(4));
label("$2v$", (1.5, 1), N);
draw((0,0) -- (0.75,0), arrow=Arrow(4));
label("$v$", (0.75, 0), N);
dot((0,0));
    \end{asy}
\end{center}
As the wheel is rolling, we have that $\omega = \dfrac{v}{R}$.\vspace{3mm}

The speed of the highest point in the lab frame is $$v + \omega R = 2v$$

Therefore, we find that the centripetal force at the highest point is $$a_c = \frac{(2v)^2}{r}$$

The speed of highest point in frame of wheel’s centre is $\omega R = v$. \vspace{3mm}

Therefore, the centripetal force in the wheels center is $$a_c = \omega^2 R = \frac{v^2}{R}$$

As both frames are inertial frames, 
\[\frac {v^2}{R} = \frac {4v^2}{r}\implies \boxed{r = 4R}\]
\end{solution}
\begin{solution}{normal}
\begin{center}
\begin{asy}
 /* Geogebra to Asymptote conversion, documentation at artofproblemsolving.com/Wiki go to User:Azjps/geogebra */
import graph; size(10cm);
real labelscalefactor = 1; /* changes label-to-point distance */
real xmin = -15.32, xmax = 23.08, ymin = -5.97, ymax = 11.53; /* image dimensions */
pen qqwuqq = rgb(0,0.39215686274509803,0);

draw(arc((6.06217782649107,-2.5),0.6,0,71.45835838231673)--(6.06217782649107,-2.5)--cycle);
/* draw figures */
draw(circle((1.14,-0.55), 1.9604081207748552));
draw(circle((4.78,2.47), 2.79624450386687));
draw((1.18,-2.51)--(-6.06217782649107,-2.5));
draw((1.18,-2.51)--(6.06217782649107,-2.5));
draw((6.06217782649107,-2.5)--(7.4205037888728,1.5498244372109937));
draw((7.4205037888728,1.5498244372109937)--(8.660254037844386,6));
draw((6.06217782649107,-2.5)--(9.526279441628825,-2.5));
label("$2\alpha$", (6, -2.5), 4NE);
/* end of picture */   
\end{asy}
\end{center}
We tilt the plane by an angle $2\alpha$. This makes the effective gravity in this scenario become 
\[g_{\text{eff}} = mg\sin\alpha\cos\alpha\]
Since the wedge is weightless, the normal force between the wedge of both blocks have to be equal otherwise, the wedge will experience an infinite acceleration. Setting these two forces equal to each other in the horizontal direction gives us 
\[F_g\sin\alpha\cos (2\alpha) = F_g\sin\alpha\]
\[\cos 2\alpha = \frac{m}{M}\]
The lower ball will then 'climb up' if 
\[m < M\cos 2\alpha\]
\tcbline
\textbf{Solution 2:} Since the wedge is weightless, the normal force between the wedge of both blocks have to be equal otherwise, the wedge will experience an infinite acceleration. Therefore, setting the forces of inertia and weight at the point when both balls make contact, produces the equation
\[mg\cos\alpha + ma\sin\alpha = Mg\cos\alpha + Ma\sin\alpha\]We also note, that by trigonometry, after contact the smaller mass must have the ratio of the translational fictitious force to the weight of the ball must be greater than $\tan\alpha$ for the ball to slide up the ramp
\[\frac{ma}{mg}>\tan\alpha\implies a>g\tan\alpha.\]We now go to the first equation and solve for acceleration. Moving variables to the same side results in
\[a\sin\alpha(m+M) = g\cos\alpha(M-m)\implies a = \frac{g\cos\alpha(M-m)}{\sin\alpha(m+M)}\]Substituting our minimum value of acceleration yields
\[\frac{g\cos\alpha(M-m)}{\sin\alpha(m+M)} > g\tan\alpha\]Solving this inequality yields
\[\boxed{m < M\cos 2\alpha}\]
\blfootnote{This problem was found in the book 'Aptitude Test Problems in Physics' by S.S. Krotov.}
\end{solution}


\end{document}
