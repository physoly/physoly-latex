\documentclass[11pt]{article}
\usepackage{titling}
\usepackage[english]{babel}
\usepackage[margin=0.6in]{geometry} % Page Dimension
\usepackage{physics}
\usepackage{enumerate}
\usepackage{physoly}

%%%%%%%%%%%%%%%%%%%%%%%%%%%%%%%%%%%%%%%%%%
%                PACKAGES                %  
%%%%%%%%%%%%%%%%%%%%%%%%%%%%%%%%%%%%%%%%%%

% Styling Choices
\setlength{\parskip}{\baselineskip}%

% Math
\usepackage{amsmath, amsthm, amssymb}
\usepackage[inline]{asymptote}
\usepackage{xcolor}
\usepackage{cancel}

% Allows for hyperlinking
\usepackage{hyperref}
\hypersetup{
    colorlinks=true,
    linkcolor=magenta,
}

% Blind Footnote
\newcommand\blfootnote[1]{%
  \makeatletter{\footnotetext{#1}\makeatother}
}

% Fancy Header
\usepackage{fancyhdr}
\pagestyle{fancy}
\lhead{Kalda Thermodynamics}
\rhead{\thepage}

% Coloured Boxes
\usepackage{xcolor}
\definecolor{border}{HTML}{004D4D}
\definecolor{hard}{HTML}{ffccb3}
\definecolor{easy}{HTML}{b3e6b3}
\definecolor{normal}{HTML}{f2f2f2}

% Syntax: \colorboxed[<color model>]{<color specification>}{<math formula>}
\newcommand*{\colorboxed}{}
\def\colorboxed#1#{%
  \colorboxedAux{#1}%
}
\newcommand*{\colorboxedAux}[3]{%
  % #1: optional argument for color model
  % #2: color specification
  % #3: formula
  \begingroup
    \colorlet{cb@saved}{.}%
    \color#1{#2}%
    \boxed{%
      \color{cb@saved}%
      #3%
    }%
  \endgroup
}

% Setup Gray Solution Boxes
\usepackage[breakable,many]{tcolorbox}
\newtcolorbox[auto counter]{solution}[1]{
    enhanced, breakable,
    arc=0pt,
    % colback=default, % Background color
    colframe=white, % Border Color
    coltitle=black, % Title Color
    fonttitle=\bfseries,
    title=\fcolorbox{border}{#1}{\textcolor{border}{pr} \bfseries \textcolor{border}{\thetcbcounter}.},
    attach title to upper,
    after title={\ },
    segmentation style={dashed, gray},
}

% Title and front page
\title{Solutions to Problems in Thermodynamics Handout by Jaan Kalda With detailed diagrams and walkthroughs}
% Author
\author{\textsc{Ashmit Dutta,  Kushal Thaman, Rakshit, QiLin Xue }}

\begin{document}
\begin{titlepage}
    \begin{center}
        \vspace*{1cm}
 
        \Huge
        \textbf{Solutions to Jaan Kalda's Problems in Thermodynamics}
 
        \vspace{0.5cm}
        \LARGE
        \textbf{With detailed diagrams and walkthroughs}
        \author{\textsc{Ashmit Dutta,  Kushal Thaman, Rakshit, QiLin Xue }}
        
        \vspace{0.1cm}
        Edition 1.2.1
        
        \vspace{1.2cm}
        
        Rakshit, Ashmit Dutta, QiLin Xue, Kushal Thaman
        \vspace{1cm}
 \begin{center}
\begin{asy}
unitsize(2.5cm);
import graph;
draw((0, 0)--(0, 4), arrow=Arrow(TeXHead), linewidth(1));
draw((0, 0)--(4,0), arrow=Arrow(TeXHead), linewidth(1));

label("$P$", (0, 4), NW);
label("$V$", (4, 0), SE);

draw((0.5, 3.5).. (2, 2.5).. (3.5, 2));
label("$T_2$", (3.5, 2), E);
draw((0.5, 2.1).. (2, 1.1).. (3.5, 0.6));
label("$T_1$", (3.5, 0.6), E);
draw((1, 1.71) -- (1, 3.1), linewidth(1));
draw((1, 1.71)..(2, 1.1)..(3, 0.72), linewidth(1));
draw((1, 3.1)..(2, 2.5)..(3, 2.13), linewidth(1));
draw((3, 0.72) -- (3, 2.13), linewidth(1));

label("2", (1, 1.71), SW);
label("1", (3, 0.72), S);
label("3", (1, 3.1), NE);
label("4", (3, 2.13), N);

draw((0.2, (1.71 + 3.1)/1.9) -- (0.7, (1.71 + 3.1)/1.9), arrow=Arrow(8));
label("$Q_1$", (0.7, (1.71 + 3.1)/1.9), N);
draw((3.2, 1.2) -- (3.7, 1.2), arrow=Arrow(8));
label("$Q_3$", (3.7, 1.2), N);
draw((2, 3.3) -- (2, 2.7), arrow=Arrow(8));
label("$Q_2$", (2, 2.7), E);
draw((2, 1.3) -- (2, 0.8), arrow=Arrow(8));
label("$Q_4$", (2, 0.8), W);

draw((1, 0) -- (1, 0.1));
draw((3, 0) -- (3, 0.1));

label("$V_1$", (1, 0), S);
label("$V_2$", (3, 0), S);
\end{asy}
\end{center}
         
        \vspace{10mm}
        \vfill
        
        \Large
        Updated
        \today
 
    \end{center}
\end{titlepage}
\newpage
%%%%%%%%%%%
% Preface %
%%%%%%%%%%%
\section*{Preface}
\vspace{-5mm}
\indent Jaan Kalda's \href{https://www.ioc.ee/~kalda/ipho/}{handouts} are beloved by physics students both in for a quick challenge, to students preparing for international Olympiads. As of writing, the current \href{https://www.ioc.ee/~kalda/ipho/Thermodyn.pdf}{thermodynamics} handout has 84 unique problems and 20 main `ideas'.

This solutions manual came as a pilot project from the online community at \url{artofproblemsolving.com}. Although there were detailed hints provided, full solutions have never been written. The majority of the solutions seen here were written on a private forum given to those who wanted to participate in making solutions. In an amazing show of an online collaboration, students from around the world came together to discuss ideas and methods and created what we see today.

This project would not have been possible without the countless contributions from members of the community. Online usernames were used for those who did not wish to be named: 

\textit{Jonathan Qin, Anant Lunia, Ameya Deshmukh and Hermab Podar.}
\subsection*{Structure of The Solutions Manual}
\vspace{-5mm}
Each chapter in this solutions manual will be directed towards a section given in Kalda's mechanics handout. There are three major chapters: statics, dynamics, and revision problems. If you are stuck on a problem, cannot make progress even with the hint, and come here for reference, look at only the start of the solution, then try again. Looking at the entire solution wastes the problem for you and ruins an opportunity for yourself to improve.

\subsection*{Contact Us}
\vspace{-5mm}
Despite editing, there is almost zero probability that there are \textit{no} mistakes inside this book. If there are any mistakes, you want to add a remark, have a unique solution, or know the source of a specific problem, then please contact us at \url{hello@physoly.tech}. The most current and updated version can be found on our website \url{physoly.tech}

Please feel free to contact us at the same email if you are confused on a solution. Chances are that many others will have the same question as you.

\newpage
\section{Solutions to Heat and Motion Problems}
\vspace{-5mm}
This section will consist of the solutions to problems from problem 1-8 of the handout. Heat and temperature is typically the analysis of objects interacting with each other via thermal energy. These problems usually involve conservation of energy, heat flux, Stefan-Boltzmann's Law, analysis of graphs, and more. 

\begin{solution}{normal}
The photon moves in a zig zag where it hits the ceiling then rebounds back down. If the distance between the floor and ceiling is $l$, and the total time is $t$, then we find that we have a triangle shaped as 
\begin{center}
\begin{asy}
size(10cm);
draw((0,0)--(2,1)--(4,0)--cycle);
draw((2,0)--(2,1));
label("$l$", (2, 0.5), E);
label("$ct/2$", (1,0.7), N);
label("$vt$", (2,0), S);
\end{asy}
\end{center}
Therefore, we find that we have a right triangle with legs $l$, and $vt/2$ and hypotenuse $ct/2$. Using Pythagorean theorem, we have that 
\[\left(\frac{ct}{2}\right)^2 = \left(\frac{vt}{2}\right)^2 + l^2\]
Expanding both squares gives 
\[\frac{c^2t^2}{4} +- \frac{v^2t^2}{4} = l^2\]
Simplifying gives 
\[\boxed{t = \frac{4l^2}{c^2 - v^2}}\]
To verify this claim, we can consider what's happening in the first scenario. When the reference frame doesn't move with a velocity perpendicular to the velocity of the photon. The total distance the photon travels is $2l$, thus the total time in the first scenario is $t' = \frac{2l}{c}$. In this problem we found that 
\[t^2 = \frac{4l^2}{c^2 - v^2}.\]
Substituting our value for $t$ into this problem gives 
\[t^2 = \frac{c^2 t'^2}{c^2 - v^2}\implies t = t'\sqrt{\frac{c^2}{c^2 - v^2}} = \boxed{t'\frac{1}{\sqrt{1-v^2/c^2}}}.\]
From fact 1, we see that this is the same as if the time interval between two events happening at a stationary point is $t$, then in a reference frame where the speed of the point is $v$ the time interval is $\gamma t$ , where the Lorentz factor
\[\gamma = \frac{1}{\sqrt{1-v^2/c^2}}.\]
\end{solution}
\begin{custom-simple}[Problem 2]
The wall blocks almost all the wave front of the original wave, leaving only two points in a cross-section perpendicular to the slits (see figure below). To be precise, these are actually segments, but their size is much smaller than the wavelength; so, from the point of view of wave propagation, the segments can be considered as points. According to the Huygens principle, two point sources of electromagnetic waves of wavelength $\lambda$ will be positioned into these two points ($A$ and $B$). The point sources radiate waves in all the directions, and we need to study the interference of this radiation. Let us study, what will be observed at a far-away screen where two parallel rays (drawn in figure) meet.
\vspace{3mm}

To begin with, it is quite easy to figure out, where are the intensity maxima and minima. Indeed, as it can be seen from the figure above, the optical path difference between the two rays is $\Delta l = a \sin\varphi$. The two rays add up constructively (giving rise to an intensity maximum) if the two waves arrive to the screen at the same phase, i.e. an integer number of wavelengths fits into the interval: $\Delta l = n\lambda$. Similarly, there is a minimum if the waves arrive in an opposite phase:
\[\sin\varphi_{\text{max}} = \frac{n\lambda}{a},\hspace{10pt} \sin\varphi_{\text{min}} = \left(n + \frac{1}{3}\right)\lambda/a\]
\end{custom-simple}
\begin{solution}{normal}
Over one complete oscillation of the voltage, the heat lost by the filament must equal the heat gained by it. Let the resistance of the filament be $R$. The heat gained by the filament is $\frac{U_1^2}{R}\frac{T}{2}$ (because the voltage is applied only for $\frac{T}{2}$). Let the rate at which heat is lost to the surrounding be $r$. The heat lost to the surroundings is $rT$ therefore 
$$rT = \frac{U_1^2}{R}\frac{T}{2} \implies r  =\frac{U_1^2}{2R}.$$
From $t = 0.5T$ to $T$, the heat lost takes the temperature from the maximum temperature to the minimum temperature, a change of $2 \Delta T$ (beware, the $\Delta T$ is the amplitude of the temperature while $T$ is time period of voltage oscillations). This implies that 
\[r\frac{T}{2} = 2mc \Delta T \implies \Delta T = \frac{U_1^2 T}{8Rmc}.\]
However, $R = \frac{\rho_{\text{el}}\ell}{A}$ and $m = \rho \ell A$, where $A$ is cross-section area of the wire. Substituting these values gives 
\[\Delta T = \frac{U_1^2 T}{8c \rho_{\text{el}} \rho \ell^2} = \frac{(17)^2(0.01)}{8(235)(9.95\times 10^{-7})(18200)(0.05)^2}=\boxed{33.8 \text{ K}}\]
\end{solution}
\begin{solution}{normal}
We notice that the graph is quadratic so we can fit it to the equation
\begin{align*}
    \alpha &= \dfrac{\pi}{180}\left(-\dfrac{60}{49}(t-7)^2 + 60\right) \\
    &= -\dfrac{\pi}{147}(t-7)^2 + \pi/3
\end{align*}
where $\alpha$ is in radians and $t$ is in minutes. \vspace{3mm}

Since we know that the upward ascending velocity is constant, it is
\begin{align*}
    v_y &=L\alpha '(0) = 1000\left(\dfrac{14\pi}{147}\right) \\
    &= 299 \, \mathrm{m/min} = \boxed{4.99\, \mathrm{m/s}}
\end{align*}

The height is simply $$h = v_y t = \boxed{2000 \, \mathrm{m}}$$

At $t=7 \, \mathrm{min}$, the change in elevation angle is momentarily 0, which means that the velocity vector also points at 60 degrees. \vspace{3mm}

Thus we can get
$$v_x = v_y \tan (30^{\circ}) \approx \boxed{2.8 \, \mathrm{m/s}}$$
\blfootnote{You don't need the equation of the curve to perform calculations, but even without it, the answer can appear a bit off.

e.g. the initial slope you get could be:
$$4^\circ / 0.2 \text{ min} = 0.0698 \text{ rad} / 12 \text{ sec} = 0.00582 \text{ sec}^{-1}$$}
\end{solution}
\newpage
\begin{solution}{normal}
a) The heat flux (or energy flux) density is $\Phi = \frac{P}{S}$ and the thermal resistivity is:
$$\rho = \frac{1}{\Phi}\frac{dT}{dx} = \frac{S}{P}\frac{dT}{dx}$$
Separating variables, we have:
$$\Delta T = \int_0^d \frac{\rho P}{S} dx =11.7 \text{ K}$$
\vspace{3mm}

\noindent b) Again, we separate variables. This time however, our expression becomes:
$$\frac{\Delta T S}{P} = \int_0^\ell \rho(x) dx.$$
The integral can be approximated as the area under the curve. In this case, we can see that the average value is approximately $0.14\;\mathrm{K\cdot m/W}$ and then use this value to approximate the integral as a rectangle. Solving for $P$ from here gives us 
$$P = 1.8 \times 10^{-2} \text{ W}.$$

\end{solution}
\begin{solution}{normal}
According to Stefan-Boltzmann's Law, the power per unit area emitted from the surface of a blackbody at temperature $T$ is $\sigma T^4$. 

The total power emitted from the sun, considered a blackbody for the sake of the problem, is therefore,
$$P_\odot=\sigma T_\odot^4 (4\pi R_\odot^2)$$

Due to the inverse square law, the solar flux stays constant through any closed surface. The portion of energy that reaches the satellite is given by the ratio between the cross-sectional area of the satellite and the area of an imaginary sphere centered around the Sun with a radius of $L$ ($R_\oplus$ is the radius of the satellite)
$$
\gamma=
\frac{\pi R_\oplus^2}{4\pi L^2}=
\left(\frac{R_\oplus}{2L}\right)^2$$

According to Prevost's theory of exchange, in order to maintain thermal equilibrium, any object must emit the same energy as it receives. If this was not true, then it would continuously lose or gain energy until it is at equilibrium. Now let the emissivity and absorptivity factors of the satellite be $\epsilon$ and $a$. Since we know that these two parameters essentially have the same value at a particular wavelength, we have
$$P_\text{in} = P_\text{out}$$
By Stefan-Boltzmann Law, we have 
\[P_{\text{output}}=4\pi R_\oplus^2\sigma T^4\]
Equating $P_{\text{in}}$ to $P_{\text{out}}$ gives us
\begin{align*}
P_\text{output} &= P_\odot \gamma_\text{eff}\\
\epsilon_ \times 4\pi R_\oplus^2\sigma T^4 &= 
a \times \sigma T_\odot^4 (4\pi R_\odot^2)\left(\frac{R_\oplus}{2L}\right)^2 \\
\cancel\epsilon_ \times 4\pi R_\oplus^2\sigma T^4 &= 
\cancel a \times \sigma T_\odot^4 (4\pi R_\odot^2)\left(\frac{R_\oplus}{2L}\right)^2 \\
 T^4 &= 
 T_\odot^4 (R_\odot^2)\left(\frac{1}{2L}\right)^2 \\
 T^4 &= \frac{T_\odot^4 R_\odot^2}{4L^2}  \\
 T &= T_\odot \sqrt{\left(\frac{R_\odot}{2L}\right)}\\
\end{align*}
Plugging in our given constants gives us $\approx\boxed{290\;\mathrm{K}}$.

\end{solution}
\begin{solution}{normal}
The center of mass of the object can be calculated by treating the wheel as a superposition of two objects, one with positive density $\rho$ and one with negative mass density $-\rho$. Taking $r=0$ to be the center, the center of mass is:
\begin{align*}
r_\text{cm} &= \frac{\rho \pi R^2 (0) - \rho \pi \left(\frac{R}{2}\right)^2 \left(\frac{R}{3}\right)}{\rho\pi R^2 - \rho \pi\left(\frac{R}{2}\right)^2} \\
&= -\frac{\frac{1}{4} \left(\frac{R}{3}\right)}{1 - 1/4} \\
&= -R/9
\end{align*}
Therefore, when the normal force is zero, we have:
\begin{align*}
m\omega^2 (R/9) &= mg\\
\omega &= 3\sqrt{g/R}
\end{align*}And therefore the speed would be
$$\boxed{v = 3\sqrt{gR}}$$
\end{solution}

\begin{solution}{hard}
Draw a right trapezoid as follows:
\vspace{3mm}

We decompose $\vec{v}$ into parallel and perpendicular components, $\vec{v} = \vec{v_x} + \vec{v_y}$; let us mark points $A, B$ and $C$ so that
$AB = \vec{v_x}$ and $BC = \vec{v_y}$ (then, $AC = \vec{v}$). \vspace{3mm}

Next we mark points $D, E$ and $F$ so that $CD = \vec{v'_y} = \vec{v_y}$, $DE = -\vec{v_x}$, and $EF = 2\vec{u_x}$; then, $CF = \vec{v_y'} - \vec{v_x} + 2\vec{u_x} \equiv \vec{v'}$ and $AF = 2\vec{v_y} + 2\vec{u_x} \equiv 2\vec{u}$. \vspace{3mm}

Due to the problem conditions, $\angle ACF = 90^{\circ}$.
\begin{center}
    \begin{asy}
        size(8.5cm);
        import olympiad;
        pair A, B, C, D, E, F;
        A = (0, 0);
        B = (-1, 0);
        C = (-1, 1);
        D = (-1, 2);
        E = (1, 2);
        F = (0.5, 1);
        
        draw(A -- B -- C -- D -- E -- F -- cycle);
        draw(C -- F);
        draw(A -- C -- E, dotted);
        markscalefactor = 0.02;
        draw(anglemark(A, C, F));
        label("$\alpha$", (-0.93, 1.03), 4SE);
        draw(anglemark(F, A, C));
        label("$\alpha$", A, 5N);
        draw(anglemark(C, F, A));
        label("$\beta$", (0.4, 0.97), 2SW);
        markscalefactor = 0.01;
        draw(rightanglemark(C, B, A));
        label("A", A, SE);
        label("B", B, SW);
        label("C", C, W);
        label("D", D, NW);
        label("E", E, NE);
        label("F", F, SE);
        label("$-\vec{v_x}$", (D+ E)/2, N);
        label("$\vec{v_y'} = \vec{v_y}$", (C + D)/2, W);
        label("$\vec{v_y}$", (B + C)/2, W);
        label("$\vec{v_x}$", (B + A)/2, S);
        label("$2\vec{v_y} + 2\vec{u_x} = 2\vec{u}$", (0.9, 1), E);
        draw(F -- (0.5, 2), dotted);
        markscalefactor = 0.01;
        draw(rightanglemark(E, (0.5, 2), F));
        label("$2\vec{u_x}$", ((0.5, 2) + E)/2, N);
    \end{asy}
\end{center}
We now can see that $\Delta ACF$ is an isoceles triangle containing the lengths provided in the figure below. \vspace{3mm}

Let us also mark point $G$ as the centre of $AF$; then, $FC$ is both the median of the right trapezoid $ABDF$ (and hence, parallel to $AB$ and the $x-$axis), and the median of the triangle $ACF$.
\begin{center}
    \begin{asy}
        size(7cm);
        import olympiad;
        pair A, B, C, D, E, F;
        A = (0, 0);
        B = (-1, 0);
        C = (-1, 1);
        D = (-1, 2);
        E = (1, 2);
        F = (0.3, 1.2);
        
        draw(A -- C -- F -- cycle);
        markscalefactor = 0.02;
        draw(anglemark(A, C, F));
        label("$\alpha$", (-0.93, 1.03), 5SE);
        draw(anglemark(F, A, C));
        label("$\alpha$", A, 7NNW);
        draw(anglemark(C, F, A));
        label("$\beta$", (0.4, 0.95), 2SW);
        label("A", A, SE);
        label("C", C, W);
        label("F", F, NE);
        label("$\vec{v}$", (A+C)/2, SW);
        label("$\vec{u}$", (C+F)/2, N);
        label("$\vec{u}$", (A+F)/2, SE);
    \end{asy}
\end{center}
By splitting $\Delta ACF$ into it's median, we find, 
\[u\cos\alpha = \frac{v}{2}\implies u = \boxed{\frac{v}{2\cos\alpha}}.\]
For this to also happen, we see that $\beta = \boxed{180 - 2\alpha}$ because $\Delta ACF$ is an isoceles triangle.
\end{solution}

\newpage
\section{Solutions to Gases}

This chapter will focus on problems 9-14 of the handout.

\begin{solution}{normal}
We shall use a property in geometry. Thales's theorem states that if A, B, and C are distinct points on a circle where the line $AC$ is a diameter, then the angle $\angle ABC$ is a right angle.
\begin{center}
\begin{asy}
import graph; size(8cm); 
real labelscalefactor = 0.5;
pen dps = linewidth(0.7) + fontsize(10); defaultpen(dps); /* default pen style */ 
pen dotstyle = black; /* point style */ 
real xmin = -5.2824091003930445, xmax = 5.600007360267558, ymin = -0.761082590475042, ymax = 4.9426400720203105;  /* image dimensions */

draw(circle((-0.125,2.25), 2.25),  linetype("2 2")); 
draw((-2,1)--(0,0)--(2,3));
draw((0,0)--(0,4.5), linetype("2 2")); 
draw((-2,1)--(2,3),linetype("2 2")); 
draw((-2,1)--(1.75,1), linetype("2 2")); 
draw((-1.18,4.24)--(0,0),linetype("2 2")); 
 /* dots and labels */
dot((-2,1),dotstyle); 
label("$A$", (-1.97,1.08), NE * labelscalefactor); 
dot((2,3),dotstyle); 
label("$C$", (2.03,3.08), NE * labelscalefactor); 
dot((0,0),dotstyle); 
label("$B$", (0.0315,-0.05), S * labelscalefactor); 
dot((0,1),linewidth(4pt) + dotstyle); 
label("$D$", (0.0315,1.0606), NE * labelscalefactor); 
dot((0,2),linewidth(4pt) + dotstyle); 
label("$O$", (0.0315,2.0629), NE * labelscalefactor); 
dot((-0.49,1.75),linewidth(4pt) + dotstyle); 
label("$E$", (-0.462,1.82), NE * labelscalefactor); 
clip((xmin,ymin)--(xmin,ymax)--(xmax,ymax)--(xmax,ymin)--cycle); 
\end{asy}
\end{center}
Therefore if we draw a circle where the corners of the two pillars form the ends of the diameter $AC$, the outline of the circle gives the possible locations the mass can be located as. Let the location of the mass be $B$. We wish to minimize the height of $B$ which so happens to be at the very bottom of the circle. Let $\angle EBD=\alpha$ such that $\angle ABE = 45^\circ$. Doing some angle tracing, we can verify that
$$\angle BAD=45^\circ-\alpha$$
Now since $OA$ and $OB$ are both the radius, that means $OAB$ is an isosceles triangle where:
$$\angle OAB = \angle ABO \implies 45^\circ-\alpha+\angle OAD = 45^\circ+\alpha \implies \angle OAD=2\alpha$$This angle relates the horizontal distance of the two pillars and the vertical distance of the two pillars through:
$$\tan OAD = \boxed{\tan(2\alpha) = \frac{h}{a}}$$
\tcbline
\textbf{Solution 2:} Let $y$ be the vertical distance between the mass and the top of the left pillar. Then let $b$ and $c$ be the horizontal distances between the mass and the left and right pillars, respectively, such that $a=b+c$. Doing basic angle tracing, we can see that:
$$b = \frac{y}{\tan(45-\alpha)}$$and

$$c = (h+y)\tan(45-\alpha)$$Adding them together and letting $\beta \equiv 45 - \alpha$ yields:

\begin{align*}
a &= b + c \\
a &= \frac{y}{\tan(\beta)} + (h+y)\tan(\beta) \\
a\tan(\beta) &= y + (h+y)\tan^2(\beta) \\
a\tan(\beta) - h\tan^2(\beta) &= y + y\tan^2(\beta) \\
\frac{\tan(\beta)(a-h\tan(\beta))}{1+\tan^2(\beta)} &= y
\end{align*}Doing a quick sanity check, this yields the correct answer of $y=a/2$ when $\beta = 45^\circ$ and $h=0$

We can simplify this further with a few trig identities. You can verify that the above expression is equivalent to

$$ y = \frac{1}{2}a\sin(90-2\alpha) - \frac{h}{2}\tan(45-\alpha) $$From the energy approach, the system will be in static equilibrium if no work is needed to rotate the system by a differential amount (change in potential energy is zero). This occurs when the gravitational potential energy is at a minimum or $y$ is minimized. Taking the derivative with respect to $\alpha$ we get:

\begin{align*}
\frac{dy}{d\alpha} &= \frac{1}{2}a\cos(90-2\alpha)(-2) - (2h\sin(45-\alpha))(\cos(45-\alpha)(-1) \\
0 &= -a\cos(90-2\alpha) + h\sin(90-2\alpha) \\
\frac{a}{h} &= \tan(90-2\alpha)
\end{align*}But since $\tan(90-2\alpha) = \cot(2\alpha)$, we can rewrite this to get:
$$\tan(2\alpha) = \frac{h}{a}$$
\end{solution}

\begin{solution}{normal}
\begin{center}
\begin{asy}
/* Geogebra to Asymptote conversion, documentation at artofproblemsolving.com/Wiki go to User:Azjps/geogebra */
import graph; size(5cm);
real labelscalefactor = 0.5; /* changes label-to-point distance */
pen dps = linewidth(0.7) + fontsize(10); defaultpen(dps); /* default pen style */
pen dotstyle = black; /* point style */
real xmin = -4.23213963380496, xmax = 7.371171204866221, ymin = -1.6441239198422954, ymax = 4.437435927619134; /* image dimensions */

/* draw figures */
draw((0,0)--(3,0), linetype("2 2"));
draw((3,0)--(3,4), linetype("2 2"));
draw((3,4)--(0,0), linetype("2 2"));
draw((0,0)--(2,0),EndArrow(6));
draw((2,0)--(2,2.7), EndArrow(6));
label("$\ell$",(1.37,2.6),SE*labelscalefactor);
label("$\mu N$",(0.823,0.4),SE*labelscalefactor);
label("$N$",(1.6,1.43),SE*labelscalefactor);
label("$h$",(3.15,2.22),SE*labelscalefactor);
label("$\sqrt{\ell^2-h^2}$",(1.2,-0.1),SE*labelscalefactor);
/* dots and labels */
dot((0,0),dotstyle);
label("$A$", (0,0), SW * labelscalefactor);
dot((3,0),dotstyle);
label("$B$", (3,0), NE * labelscalefactor);
dot((3,4),dotstyle);
label("$C$", (3,4), NE * labelscalefactor);
clip((xmin,ymin)--(xmin,ymax)--(xmax,ymax)--(xmax,ymin)--cycle);
/* end of picture */
\end{asy}
\end{center}
Consider what happens when the applied force approaches infinity. To maintain equilibrium, the friction force between the rod and the board must also increase. This friction force will also approach infinity. When dealing with large forces, we can ignore constant forces such as the weight of both the board and the rod.
\vspace{2mm}

As a result, since the weight of the rod is negligible we can pretend it's a mass-less rod. We also know that the forces at the ends of a massless rod will always point along the rod. For example, the force exerted on the rod by the board must point along the rod as well. The angle of this force is solely dependent on the friction coefficient $\mu_1$. Therefore:
$$\tan\alpha < \frac{\mu_1 N}{N} \implies \boxed{\mu_1>\frac{\sqrt{\ell^2-h^2}}{h}}$$
\tcbline
\textbf{Solution 2:} We want that when the board is on the verge of slipping then the rod should exert a larger force on the board (the rod should be pulled towards the board and not away from it). Consider the torque on the rod about the hinge point. We want that it should be clockwise when the block is on the verge of slipping.
\vspace{2mm}

Let the sum of normal reaction and friction force on the rod be $f$ (the normal points upwards and the friction points to the right). When the block is on the verge of slipping, the resultant makes an angle $\tan^{-1} \mu$ from the normal. We have:
$$\tau = mg\sin \alpha\frac{l}{2} + f \sin (\tan^{-1} \mu - \alpha)$$
considering clockwise torque to be positive. As the applied force on the block increases, $f$ also increases without bounds and because we want the torque to be clockwise no matter how much force we apply, the $mg$ term can be neglected. So
$$f \sin(\tan^{-1}\mu-\alpha) \ge 0$$
Since both $\tan^{-1}\mu$ and $\alpha$ are less than $90^{\circ}$, we can conclude that
$$\boxed{\tan^{-1} \mu \ge \alpha \implies \mu \ge \frac{\sqrt{l^2-h^2}}{h}}$$
\end{solution}

\begin{solution}{easy}
We will use a virtual work approach.\footnote{If you are unfamiliar with virtual work, refer to the explanation in Kalda's handout, or to this \href{http://docshare04.docshare.tips/files/26737/267376365.pdf}{pdf}.} In a static situation, the net force will be zero and as a result the potential energy will be at a minimum. Any slight displacement will create no change to the potential energy in first order.
\vspace{2mm}

Consider what happens when the mass is lowered by a distance $dh$. The potential energy would drop by $-mg dh$. The distance between hinges would each increase by $dh/3$ to compensate for the length increase. This means the string gets stretched by $dh/3$. The energy stored thus is:$$T dh/3$$Setting these changes to zero gives:
$$-mg dh + T dh/3 = 0 \implies \boxed{T = 3mg}$$
\end{solution}

\begin{solution}{normal}
\begin{center}
    \begin{asy}
        unitsize(3cm);
        real a = 40*pi/180;
        filldraw((-1.5,0)--(0,1.5*sin(a))--(0.03*sin(a),1.5*sin(a)-0.03*cos(a))--(-1.5+0.03*sin(a),-0.03*cos(a))--cycle,white);
        filldraw((1.5,0)--(-0.8,2.3*sin(a))--(-0.8-0.03*sin(a),2.3*sin(a)-0.03*cos(a))--(1.5-0.03*sin(a),-0.03*cos(a))--cycle,white);
        filldraw((-2,0)--(2,0)--(2,-0.05)--(-2,-0.05)--cycle);
        draw(arc((-1.5,0),0.3,0,27));
        draw(arc((1.5,0),0.3,180,153));
        label("$\alpha$",(-1.1,0.1));
        label("$\alpha$",(1.1,0.1));
        filldraw(circle((-0.75,1.49),0.03));
        label("$A$",(-0.9,1.49));
        filldraw(circle((-0.1,0.94),0.03));
        label("$B$",(-0.23,0.94));
        draw("$x$",(-0.75,0.85)--(-0.1,0.85),Arrows,Bars,PenMargins);
    \end{asy}
\end{center}

In the reference frame of ball A, ball B accelerates to the left with
$$a_B=2g\sin\alpha\cos\alpha$$

We can find that the initial length $|AB|$ is
$$\dfrac{g\left(t_1^2-t_2^2\right)\sin\alpha}{2}$$

Therefore,
$$x=\dfrac{g\left(t_1^2-t_2^2\right)\sin\alpha\cos\alpha}{2}$$

Since there is no relative acceleration in the y-direction, we need
\begin{align*}
\dfrac{a_Bt^2}{2}&=x\\
gt^2\sin\alpha\cos\alpha&=\dfrac{g\left(t_1^2-t_2^2\right)\sin\alpha\cos\alpha}{2}\\
t&=\boxed{\sqrt{\dfrac{t_1^2-t_2^2}{2}}}
\end{align*}
\tcbline

\textbf{Solution 2:} Each ball will accelerate with the same acceleration down their platform, meaning that they will travel the same distance in the same timeframe. \vspace{3mm}

Let $x$ be the distance traveled by the individual balls and $k$ be the distance between the two balls. Let the height of the ball at point $A$ be $h_1$ and the height of the ball at point $B$ be $h_2$. \vspace{3mm}

If you draw a diagram you will find that there is a triangle formed by the position of the two balls and the intersection of the planks. The lengths of the triangle are $x, x - \dfrac{h_1 - h_2}{\sin{\alpha}}, k$.

\begin{center}
    \begin{asy}
        unitsize(4cm);
        real a = 40*pi/180;
        filldraw((-1.5,0)--(0,1.5*sin(a))--(0.03*sin(a),1.5*sin(a)-0.03*cos(a))--(-1.5+0.03*sin(a),-0.03*cos(a))--cycle,white);
        filldraw((1.5,0)--(-0.8,2.3*sin(a))--(-0.8-0.03*sin(a),2.3*sin(a)-0.03*cos(a))--(1.5-0.03*sin(a),-0.03*cos(a))--cycle,white);
        filldraw((-2,0)--(2,0)--(2,-0.05)--(-2,-0.05)--cycle);
        draw(arc((-1.5,0),0.3,0,27));
        draw(arc((1.5,0),0.3,180,153));
        label("$\alpha$",(-1.1,0.1));
        label("$\alpha$",(1.1,0.1));
        filldraw(circle((-0.44,1.29),0.03));
        label("$A$",(-0.33,1.35));
        filldraw(circle((-0.41,0.74),0.03));
        label("$B$",(-0.53,0.74));
        filldraw(circle((-0.01,1.5*sin(a)-0.02),0.02));
        label("$O$",(-0.01,0.85));
        pair A = (-0.44,1.29);
        pair B = (-0.41,0.74);
        pair O = (-0.01,1.5*sin(a)-0.02);
        draw(A--B--O--cycle);
        label("$x$",(B+O)/2+(0.05,-0.08));
        label("$k$",(A+B)/2+(-0.07,0));
        label("$\displaystyle{\frac{h_1-h_2}{\sin\alpha}-x}$",(A+O)/2+(0.3,0.1));
    \end{asy}
\end{center}

By the Law of Cosines, we have
\[k^2 = x^2 + \left(\frac{h_1 - h_2}{\sin{\alpha}}-x\right)^2 - 2x\left(\frac{h_1 - h_2}{\sin{\alpha}}-x\right) \cos(2\alpha)\]

Let $\beta = \dfrac{h_1 - h_2}{\sin{\alpha}}\;$ for simplicity.\vspace{3mm}

Simplifying the expression, we get that
\[k(x) = \sqrt{2x^2(1+\cos\left(2\alpha\right)) - 2x\beta (1+\cos\left(2\alpha\right)) + \beta^2}\]

After taking the derivative of the quadratic and setting it equal to zero, we get that
$$x_m = -\dfrac{B}{2A}=\dfrac{\beta}{2}$$

Using acceleration along the ramp we can also find that
\begin{align*}
\dfrac{h_1}{\sin\alpha}&=\dfrac{gt_1^2\sin\alpha}{2}\;\;\;\;\;\;\;\;\; \dfrac{h_2}{\sin\alpha}=\dfrac{gt_2^2\sin\alpha}{2}\\
x_m&=\dfrac{gt_m^2\sin\alpha}{2}=\dfrac{h_1-h_2}{2\sin\alpha}
\end{align*}
Plugging in everything we find that
\begin{align*}
t_m&=\sqrt{\dfrac{h_1-h_2}{g\sin^2\alpha}}\\
&=\sqrt{\dfrac{\dfrac{gt_1^2\sin^2\alpha}{2}-\dfrac{gt_2^2\sin^2\alpha}{2}}{g\sin^2\alpha}}\\
&=\boxed{\sqrt{\dfrac{t_1^2-t_2^2}{2}}}
\end{align*}
\end{solution}
\begin{solution}{normal}
Since $H\ll L$, the curvature of the rope is very small which means that we can approximate the section that is above the ground as a straight line. Furthermore, the angle between the tangent to the rope and horizon remains everywhere small. Now, consider the following diagram assuming that the mass density of the rope is $\lambda$:
\begin{center}
\begin{asy}
size(10cm);
draw((0,0) -- (2, 0));
draw((0, 0) -- (-0.5, 1));
dot((-0.25, 0.5));
draw((-0.25, 0.5) -- (-0.25, 0.1), arrow=Arrow(4));
label("$mg$", (-0.25, 0.1), W);
label("$H$", (-0.5, 0.5), W);
label("$\ell$", (-0.25, 0), S);
draw((-0.5, 1) -- (-0.5, 0) -- (0, 0), dashed);
draw((-0.55, 1.1) -- (-0.75, 1.5), arrow=Arrow(4));
label("$F$", (-0.75, 1.5), W);
draw((1, 0) -- (1, 0.25), arrow=Arrow(4));
label("$N$", (1, 0.25), N);
label("$\lambda (L - \ell) g$", (0.9, -0.25), S);
draw((0.9, 0) -- (0.9, -0.25), arrow=Arrow(4));
draw((1.8, -0.1) -- (2.2, -0.1), arrow=Arrow(4));
label("$f$", (2.2, -0.1), E);
\end{asy}
\end{center}
The mass of the rope that is on the ground is given by $\lambda (L - \ell)$ where $\ell$ represents the horizontal part of the rope that is above the ground (as shown in the picture). Since the angle is small, we can assume that $\ell$ approximately represents the total length of the part of the rope that is above the ground. Since the weight of this section of the rope balances the normal force $N$, this then means that the frictional force $f = \mu N = \mu \lambda (L - \ell) g$. By using a torque balance, we can then write that 
\[\lambda \ell g \frac{\ell}{2} = fH = \mu \lambda (L - \ell) gH.\]
Cancelling factors then yields a quadratic which has a solution of 
\[\frac{\ell^2}{2} = \mu (L - \ell) H\implies \ell = \sqrt{2HL\mu + \mu^2 H^2} - \mu H \approx \sqrt{2HL\mu} - \mu H \approx 7.2\;\mathrm{m}.\]
\end{solution}

\begin{solution}{normal}
\begin{center}
    \begin{asy}
        unitsize(3cm);
        pair O = (0,0);
        pair A = (1,1);
        pair B = (2,-0.5);
        draw(O--A--A+B--B--O--B--B+(1,0.3)--A+B);
        draw(arc(B,0.3,16,45));
        label("$\alpha$",B+(0.38,0.22));
        draw(B/2-(1,0)--B/2,arrow=Arrow());
        draw(arc(B/2,0.5,180,166));
        label("$\beta$",B/2-(0.8,-0.08));
        draw(ellipse(B/2-(1.2,0.05),0.2,0.2/3));
        filldraw(B/2-(1,0.05)--B/2-(1.4,0.05)--B/2-(1.4,-0.05)--B/2-(1,-0.05)--cycle,gray(0.949),invisible);
        draw(ellipse(B/2-(1.2,-0.05),0.2,0.2/3));
        draw(B/2-(1,-0.05)--B/2-(1,0.05));
        draw(B/2-(1.4,-0.05)--B/2-(1.4,0.05));
        label("$v_0$",B/2+(-0.6,-0.1));
    \end{asy}
\end{center}

When on the plane, the puck experiences no change in its x-velocity, which is
$$v_0\cos\beta=5\;\text{m/s}$$

However, it experiences an acceleration parallel to the plane with magnitude
$$a=g\sin\alpha$$

We note from the trajectory given that the puck drops $2.5\;\text{m}$ below the apex of its trajectory while undergoing a horizontal displacement of $x=5\;\text{m}$.

The time it takes to complete this motion is
$$t=\dfrac{x}{v_0\cos\beta}=1\;\text{s}$$

Therefore, we have that
\begin{align*}
\dfrac{gt^2\sin\alpha}{2}&=2.5\\
\sin\alpha&=\dfrac{5}{gt^2}\\
\alpha&\approx \boxed{30\degree}
\end{align*}
\end{solution}

\newpage
\section{Solutions to Adiabatic processes}

This chapter will focus on problems 15-24 of the handout.

\begin{custom-simple}[Problem 15]
Let $d$ be the thickness of the film. We find optical path difference ($\Delta x$) between the two rays shown in figure. We first note that that a phase difference of $\pi$ radians occurs at each boundary if refractive index of the medium in which light is travelling is less than the the refractive index of the medium which light strikes. As in both boundaries of the thin film a phase shift occurs this doesn't change the path difference or interference pattern.

Let $\alpha$ be angle of incidence of the rays for the lower boundary (i.e. boundary between thin film and glass plates). It is well known that in case of thin film interference the optical path difference is  $\Delta x = 2n_0d\cos \alpha$
\vspace{3mm}
\[0\leq \sin \theta \leq 1 \implies 0 \leq \sin \alpha \leq \frac{1}{n_0}\implies \sqrt{1-\frac{1}{n_0^2}} \leq \cos \alpha \leq 1\]
Therefore,
\[\Delta l_ {\text{min}} = 2n_0d\sqrt{1-\frac{1}{n_0^2}}, \Delta l_{\text {max}} = 2n_0 d\] 
Changing the view direction from vertical to horizontal changes the optical path length difference by $N\lambda$ (because during this process, $N$ interference maxima can be recorded, when the optical path length difference is an integer multiple of wavelength). Therefore,
$$2n_0d\left(1-\sqrt{1-\frac{1}{n_0^2}}\right) = N\lambda \implies \boxed {d =\frac{N\lambda}{2(n_0 - \sqrt{n_0^2 - 1})}}$$
\blfootnote{A derivation of optical path difference in thin film interference can be found \hyperlink{https://en.m.wikipedia.org/wiki/Thin-film_interference#Theory}{here}}
\end{custom-simple}
\begin{custom-simple}[Problem 16]
\begin{enumerate}
\item There is no light coming out from outlet $C_2$ because at the junction point a wave is generated in upper fiber in the same direction as the circular fiber (Huygens principle can be used to prove this).As energy at steady state is constant we can say that the total energy input $=$ total energy output, hence $I_{A_2}+I_{C_1}=I_0$. The result is a mirror image of the graph in the problem text that touches $I = 0$ at the bottom and $I = I_0$ at the top.
\item At this wavelength, all intensity $I_0$ comes out from fiber $C_1$ and intensity in fiber $B$ and intensity in direction $C_1$ should have ratio $(1-\alpha)/\alpha$. So $$I_B = \frac{I_0(1-\alpha)}{\alpha} = 99I_0$$
\item The intensity of light in fiber B is maximum when the light circulating in the fiber reaches the lower junction point in the same phase as the light from fiber A. Then the intensity going to fiber C is also maximum. Thus, fiber B must accommodate an integer of n wavelengths. From the graph we see that two successive resonances occur at wavelengths $\lambda_0 = 1660$ nm and $\lambda_1 = 1680$ nm. So $n\lambda_1 ’ = (n+1) \lambda_0 ’= l$, where $l$ is the desired length and the second resonant wavelength in the fiber is $\lambda_1 ’ = \lambda_0 ’\frac{\lambda_1 }{\lambda_0}$. From this relation we find $\frac{1}{n}= \frac{\lambda_1 ’}{\lambda_0 ’} - 1$ and
$$ l = \frac {\lambda_0 ’\lambda_1}{\lambda_1-\lambda_0} = \boxed{84\;\mathrm{\mu m}}$$
\end{enumerate}
\end{custom-simple}
\begin{custom-simple}[Problem 17]
\begin{center}
    \includegraphics[width=12cm]{p17.png}
\end{center}
We find the images of light source in the mirrors. The light incident around $O$ can then be viewed as a superposition of the light emitted from the images. Let us take the line joining point $O$ and the point of intersection as x-axis and the line perpendicular to x-axis and passing through point $O$ as y-axis. Position of the image formed by the lower mirror is
$$y&=-2a\cos\alpha\sin\alpha, x=-2a\cos\alpha\cos\alpha + a$$
$$ \tan \phi &= \frac {y}{x} = \frac {2a\cos\alpha\cdot \sin\alpha}{2a\cos\alpha\cdot \cos\alpha + a}, \sin \phi = \frac {\tan \phi}{\sqrt {1 + (\tan \phi)^2}}$$
$$\sin \phi = \frac{\sin 2\alpha}{\sqrt {8\cos^2\alpha + 1}}$$
Let some point $M$ near point $O$ be the point of constructive interference. If we move up from this point by distance $d$, the path of light from the lower image would be increased by $d \sin\phi$ and the path of light from the upper image would be shortened by $d \sin\phi$. So we get a path difference $2d \sin\phi$ compared to $M$. This is again a point of constructive interference if $\lambda = 2d \sin\phi$. So we get the answer
$$\lambda = \frac{2d \sin 2\alpha}{\sqrt{8\cos^2\alpha + 1}}$$
\end{custom-simple}
\begin{custom-simple}[Problem 18]
\textbf{(a)} By energy conservation, the amplitudes of the output wave and input wave must be the same. The output fiber wave is formed by the sum of the wave in the fiber and the wave from the other fiber. According to the energy conservation, the amplitude of each component is $\sqrt 2$ times smaller than the original when the wave enters only one fiber. Thus, while the amplitude of the incoming waves was A, the outgoing resultant wave is in an expressible form.
$$A = \sqrt {\left(\frac {A}{\sqrt 2}\right)^2 \cdot 2 + 2\left(\frac {A}{\sqrt 2}\right) \left(\frac {A}{\sqrt 2}\right)\cos \phi}$$where $\phi$ is the phase shift. So $\cos (\phi/ 2) = 1/\sqrt 2$ and consequently $\phi = \frac {\pi}{2}$
\vspace{5mm}

\textbf{(b)} Phase difference between the $2$ fibers is $\pi$, the minima condition in fiber $1$ is $\Delta l = n\lambda$, where n is an integer. Writing this as $n = \frac{\Delta l}{\lambda}$ we see that
\[\frac{\Delta l}{\lambda_{\text{min}}}\geq n \geq \frac{\Delta l}{\lambda_{\text{max}}}\]thus $49.2 \geq n \geq 45.4$ and the values of $n$ to be sought are $46, 47, 48$ and $49$. The corresponding wavelengths are given by the formula $\lambda = \frac{n}{\Delta l}$; these are $612, 625, 638$ and $652\;\mathrm{nm}$
\end{custom-simple}
\newpage
\begin{solution}{hard}
In a rotating reference frame, we have that 
\[\vec\omega_3 = \vec\omega_1 + \vec\omega_2\]
where $\vec\omega_1$ is the angular velocity in the reference frame, $\vec\omega_2$ is the angular velocity of the body in the rotating reference, and $\vec\omega_3$ is that in the stationary frame. If you consider the reference point to be at infinity, then you find that the rotational motion of the disk becomes negligible. Therefore, we have that \blfootnote{This problem was found in the book 'Aptitude Test Problems in Physics' by S.S. Krotov.}
\[0 = \vec\omega_1 + \vec\omega_2\]
\[\boxed{\vec\omega_1 = -\vec\omega}\]
\end{solution}

\begin{solution}{normal}
Note that for a parabola with equation $x^2=4p(z-k)$, the focus is located at $(0,k+p)$\vspace{3mm}

In problem 19, we have the equation
$$z\leq\dfrac{v_0^2}{2g}-\dfrac{gx^2}{2v_0^2}$$
and with some manipulation we obtain
$$x^2=\dfrac{v_0^4}{g^2}-\dfrac{2v_0^2}{g}z.$$
Factoring the equation gives
$$x^2=-\dfrac{2v_0^2}{g}\left(z-\dfrac{v_0^2}{2g}\right)$$
$$x^2=4\left(-\dfrac{v_0^2}{2g}\right)\left(z-\dfrac{v_0^2}{2g}\right)$$
Therefore, the focus of the parabola is at
$$\left(0,\dfrac{v_0^2}{2g}-\dfrac{v_0^2}{2g}\right)=(0,0).$$
In problem 19, we assumed that the cannon was located at $(0,0)$, and so we are done.
\end{solution}
\begin{solution}{normal} \textbf{a)} Let us assume that the temperature stays roughly constant. This means that the sublimation rate is also constant and exerts some pressure $p$ on the vapour. We know that the saturation vapour pressure $p_0$ is defined such that the rate of sublimation = rate of deposition. This means that the pressure exerted by the sublimation is also $p_0$. Therefore the force is:
$$p_0A=Ma \implies a = \boxed{\frac{M}{p_0 A}}$$
\vspace{3mm}

\noindent \textbf{b)} Both evaporation and condensation apply the same pressure at saturation ($p_0/2$, to be exact), but since the particles escape never to come back (because $\lambda\gg\text{the length of the vessel}$), there is no condensation and thus only half the pressure is applied. Therefore, 
\[\frac{p_0}{2}A = Ma\implies a = \boxed{\frac{p_0 A}{2M}}.\]

\end{solution}
\begin{solution}{hard} \textbf{a)} First off, we find the pressure at $20^\circ\;\mathrm{C}$ on the graph. At this point, the pressure is approximately given by $2.3\;\mathrm{kPa}$. We are told that the relative humidity is $90\%$ which means that the relative pressure is given by \[2.3\;\mathrm{kPa} \cdot 0.9 = 2.07\;\mathrm{kPa}.\]
The temperature on the graph when it is approximately $2.07\;\mathrm{kPa}$ is around $18.5^\circ\;\mathrm{C}$. This then tells us that the temperature difference is 
\[20^\circ\;\mathrm{C} - 18.5^\circ\;\mathrm{C} = \boxed{1.5^\circ\;\mathrm{C}}\]
\vspace{3mm}

\noindent \textbf{b)} We are given the equations
\begin{align*}
Q_c &= a(T_0 - T)\\
Q_e &= b[p_s(T) - p_a]
\end{align*}
Dividing these two equations through gives us 
\[\frac{Q_c}{Q_e} = \frac{a}{b}\frac{T_0 - T}{p_s (T) - p_a}\]
from here, we know imediately that $a/b = 65\;\mathrm{Pa/K}$ and $T_0 = 20^\circ\;\mathrm{C}$. Because $r = 0$, then $p_a = 0$, and because $r=0$, then $Q_c = Q_e$. Therefore, our new equation is
\[1 = 65\frac{20 - T}{p_s(T)}\implies p_s (T) = 65 (20 - T).\]
From here, we find the intersection point with this line is $(6.5, 0.87)$, which implies the temperature is $\boxed{6.5^\circ\;\mathrm{C}}$.
\vspace{3mm}

\noindent \textbf{c)} In steady state, we have that 
\[Q_c =Q_e\implies \frac{a}{b}(T_0 - T) = p_s(T) - p_a.\]
Substituting  $a/b = 65\;\mathrm{Pa/K}$ and $T_0 = 20^\circ\;\mathrm{C}$ and $T\approx 2300r\;\mathrm{kPa}$ where $r$ is the relative humidity gives us 
\[65 (20 - T) = p_s (T) - 2300r.\]
\begin{itemize}
\item When $r = 1$, we have the equation the line as
\[p_s = 3600 - 65T.\]
The intersection of this line with the given curve is at $T = 20^{\circ}\;\mathrm{C}$ and $p_s = 2300\;\mathrm{Pa}$.
\item When $r = 0.8$, we have the equation of the line as 
\[p_s = 3140 - 65T.\]
The intersection of this line with the given curve is at $T = 18.75^{\circ}\;\mathrm{C}$ and $p_s = 2000\;\mathrm{Pa}$.
\end{itemize}
Since $p\propto r$ and we have the values of pressure at two different values of $r$, we can find a linear relation between pressure and relative humidity to get the equation
\[p_s = 1500r + 800\implies 65\Delta T = 800 (1 - r)\implies \Delta T = \boxed{12.3 (1 - r)}.\]
\vspace{3mm}

\noindent \textbf{d)} For the boundary condition, heat dissipated through evaporation. Therefore, 
\[k\frac{dT}{dt} = b(p_s - 2300r) = 800(1 - r) \implies \dot{T}\propto 1 - r.\]
We then see that 
\[\frac{\dot{T}_{80}}{\dot{T}_{35}} = \frac{1 - 0.8}{1 - 0.35} = \boxed{4}.\]
\end{solution}
\begin{solution}{normal}
\begin{center}
    \begin{asy}
        import olympiad;
        unitsize(2.5cm);
        filldraw((0,0)--(5,0)--(5,0.2)--(0,0.2)--cycle,mediumgrey);
        filldraw((0,-4.5)--(5,-4.5)--(5,-4.3)--(0,-4.3)--cycle,mediumgrey);
        draw((2.5,0)--(3.2,0),arrow=Arrow(),red);
        draw((2.5,-4.3)--(1.55,-4.3),arrow=Arrow(),blue);
        label("$v_1$",(3.4,-0.1));
        label("$v_2$",(1.4,-4.2));
        filldraw((2.5,0)..(2.8,-0.2)..(3.1,-0.3)..(3.3,-0.4)..(3.5,-0.5)..(3.7,-0.8)..(3.9,-1.1)..(4,-1.3)..(4.1,-1.5)..(4.2,-1.8)..(4.3,-2.2)..(4.4,-2.5)..(4.5,-2.7)..(4.6,-2.8)..(4.5,-3.3)..(4.3,-3.2)..(4.1,-3.1)..(3.9,-3)..(3.7,-2.9)..(3.5,-2.8)..(3.2,-2.7)..(3.1,-2.7)..(3,-2.9)..(3.2,-3.2)..(3.2,-3.4)..(3.1,-3.8)..(3,-3.9)..(2.8,-4.2)..(2.5,-4.3)..(2.2,-4)..(1.9,-3.9)..(1.6,-3.8)..(1.4,-3.7)..(1.2,-3.6)..(1.1,-3.5)..(1,-3.4)..(0.9,-3.1)..(0.9,-3)..(1,-2.8)..(1.2,-2.5)..(1.3,-2.2)..(1.1,-1.9)..(0.9,-1.7)..(0.7,-1.6)..(0.5,-1.5)..(0.4,-1.3)..(0.45,-1)..(0.6,-0.8)..(0.7,-0.8)..(1,-0.9)..(1.3,-0.8)..(1.8,-0.5)..(2.1,-0.2)..(2.4,-0.02)..cycle,lightgrey);
        draw((2.5,0)--(2.5,-4.3),dashed);
        draw((3.2,0)--(1.55,-4.3),dashed);
        dot((2.5,-1.8));
        draw(scale(0.5)*rightanglemark((5,-0.1),(5,0),(5.2,0)));
        draw(scale(0.5)*rightanglemark((5,-0.1),(5,-8.6),(4.8,-8.6)));
        label("$O$",(2.8,-1.8));
        label("$l_1$",(2.3,-0.9));
        label("$l_2$",(2.7,-3));
        draw(circle((2.5,-1.8),1.8),dotted);
        draw(arc((2.5,-1.8),1.8,-41,-10),red);
        draw(arc((2.5,-1.8),1.8,214,292),red);
        draw(arc((2.5,-1.8),1.8,174,150),red);
        dot((2.5,0),red);
        draw(circle((2.5,-1.8),2.5),dotted);
        draw(arc((2.5,-1.8),2.5,-36.5,-28),blue);
        dot((2.5,-4.3),blue);
    \end{asy}
\end{center}
We know by idea 33 that the instantaneous axis of rotation $O$ of the object exists. \vspace{3mm}

Let $l_1$ and $l_2$ be the distance from $O$ to the top and bottom boards, respectively. \vspace{3mm}

In fact, we have that $$\dfrac{l_1}{l_2}=\dfrac{|v_1|}{|v_2|}$$

By the properties of the instantaneous axis of rotation, we know that all points with speed $|v_1|$ lie on a circle centered at $O$ with radius $l_1$, and all points with speed $|v_2|$ lie on a circle centered at $O$ with radius $l_2$.
\end{solution}
\begin{solution}{normal}
\begin{center}
    \begin{asy}
    size(5cm);
    draw(circle((0,0), 1));
draw((0,1)--(1.5,1), arrow=Arrow(4));
label("$2v$", (1.5, 1), N);
draw((0,0) -- (0.75,0), arrow=Arrow(4));
label("$v$", (0.75, 0), N);
dot((0,0));
    \end{asy}
\end{center}
As the wheel is rolling, we have that $\omega = \dfrac{v}{R}$.\vspace{3mm}

The speed of the highest point in the lab frame is $$v + \omega R = 2v$$

Therefore, we find that the centripetal force at the highest point is $$a_c = \frac{(2v)^2}{r}$$

The speed of highest point in frame of wheel’s centre is $\omega R = v$. \vspace{3mm}

Therefore, the centripetal force in the wheels center is $$a_c = \omega^2 R = \frac{v^2}{R}$$

As both frames are inertial frames, 
\[\frac {v^2}{R} = \frac {4v^2}{r}\implies \boxed{r = 4R}\]
\end{solution}
\begin{solution}{normal}
\begin{center}
\begin{asy}
 /* Geogebra to Asymptote conversion, documentation at artofproblemsolving.com/Wiki go to User:Azjps/geogebra */
import graph; size(10cm);
real labelscalefactor = 1; /* changes label-to-point distance */
real xmin = -15.32, xmax = 23.08, ymin = -5.97, ymax = 11.53; /* image dimensions */
pen qqwuqq = rgb(0,0.39215686274509803,0);

draw(arc((6.06217782649107,-2.5),0.6,0,71.45835838231673)--(6.06217782649107,-2.5)--cycle);
/* draw figures */
draw(circle((1.14,-0.55), 1.9604081207748552));
draw(circle((4.78,2.47), 2.79624450386687));
draw((1.18,-2.51)--(-6.06217782649107,-2.5));
draw((1.18,-2.51)--(6.06217782649107,-2.5));
draw((6.06217782649107,-2.5)--(7.4205037888728,1.5498244372109937));
draw((7.4205037888728,1.5498244372109937)--(8.660254037844386,6));
draw((6.06217782649107,-2.5)--(9.526279441628825,-2.5));
label("$2\alpha$", (6, -2.5), 4NE);
/* end of picture */   
\end{asy}
\end{center}
We tilt the plane by an angle $2\alpha$. This makes the effective gravity in this scenario become 
\[g_{\text{eff}} = mg\sin\alpha\cos\alpha\]
Since the wedge is weightless, the normal force between the wedge of both blocks have to be equal otherwise, the wedge will experience an infinite acceleration. Setting these two forces equal to each other in the horizontal direction gives us 
\[F_g\sin\alpha\cos (2\alpha) = F_g\sin\alpha\]
\[\cos 2\alpha = \frac{m}{M}\]
The lower ball will then 'climb up' if 
\[m < M\cos 2\alpha\]
\tcbline
\textbf{Solution 2:} Since the wedge is weightless, the normal force between the wedge of both blocks have to be equal otherwise, the wedge will experience an infinite acceleration. Therefore, setting the forces of inertia and weight at the point when both balls make contact, produces the equation
\[mg\cos\alpha + ma\sin\alpha = Mg\cos\alpha + Ma\sin\alpha\]We also note, that by trigonometry, after contact the smaller mass must have the ratio of the translational fictitious force to the weight of the ball must be greater than $\tan\alpha$ for the ball to slide up the ramp
\[\frac{ma}{mg}>\tan\alpha\implies a>g\tan\alpha.\]We now go to the first equation and solve for acceleration. Moving variables to the same side results in
\[a\sin\alpha(m+M) = g\cos\alpha(M-m)\implies a = \frac{g\cos\alpha(M-m)}{\sin\alpha(m+M)}\]Substituting our minimum value of acceleration yields
\[\frac{g\cos\alpha(M-m)}{\sin\alpha(m+M)} > g\tan\alpha\]Solving this inequality yields
\[\boxed{m < M\cos 2\alpha}\]
\blfootnote{This problem was found in the book 'Aptitude Test Problems in Physics' by S.S. Krotov.}
\end{solution}

\begin{solution}{normal}
\begin{center}
    \begin{asy}
        unitsize(1.2cm);
        filldraw((0,0)--(6.8,0)--(6.8,-0.3)--(0,-0.3)--cycle,gray(0.65));
        filldraw(circle((5.3,1.3),1.3),gray(0.85));
        filldraw((-0.4,1.9)--(0.3,1.9)--(0.3,4.1)--(-0.4,4.1)--cycle,gray(0.65),invisible);
        draw((0.3,1.9)--(0.3,4.1),linewidth(2));
        draw((0.3,3.6)--(5.55,2.58));
        draw(arc((0.3,3.6),1,270,348));
        label("$\alpha$",(0.7,3.1));
        draw((7.2,-0.15)--(8.2,-0.15),arrow=Arrow());
        label("$v$",(7.7,0.1));
        draw((5.3,1.3)--(6.6,1.3));
        dot((5.3,1.3));
        label("$R$",(5.3/2+6.6/2,1));
    \end{asy}
\end{center}

From constraints on the thread we can determine that
$$\omega R = v_{\text{CM}}\sin{\alpha}$$

For a no-slipping condition, we have
$$v_{\text{CM}} = v_0 - \omega R \Rightarrow v_{\text{CM}} = \boxed{\dfrac{v_0}{1+\sin{\alpha}}}$$
\tcbline

\textbf{Solution 2:} We move into the frame moving left with velocity $v$
$$\sin\alpha=\dfrac{v_0dt}{(v-v_0)dt}=\dfrac{v_0}{v-v_0}$$
$$v_0=\dfrac{v\sin\alpha}{1+\sin\alpha}$$

Moving back into the reference frame of the wall, we get
$$v_0=v-\dfrac{v\sin\alpha}{1+\sin\alpha}=\boxed{\dfrac{v}{1+\sin\alpha}}$$
\end{solution}
\begin{solution}{normal}
% Since the velocity along the rod is zero, we can equate the projected component of velocities along the rod to get
% $$ - u\sin{\alpha} = v \cos{\alpha} \Rightarrow v = -u\tan{\alpha}$$
\begin{center}
    \begin{asy}
        unitsize(1cm);
        filldraw((0,6)--(0,0)--(4,0)--(4,-0.7)--(-0.7,-0.7)--(-0.7,6)--cycle,gray(0.75),invisible);
        draw((0,6)--(0,0)--(4,0),linewidth(1.5));
        draw((3,0)--(0,5),linewidth(2.5));
        draw(arc((3,0),0.7,180,120.96));
        label("$\alpha$",(2,0.5));
        draw((0,5)--(0,3.5),arrow=Arrow(),red+linewidth(0.7));
        draw((0,5)--(0,3.5),red+linewidth(1.5));
        label("$v$",(-0.3,4.25));
        draw((3,0)--(2,0),arrow=Arrow(),blue+linewidth(0.7));
        draw((3,0)--(2,0),blue+linewidth(1.5));
        label("$u$",(2.5,-0.3));
    \end{asy}
\end{center}

Since the length of the rod is constant, we can consider the equation
$$x^2+y^2 = L^2$$

Differentiating with respect to time, this gives
\begin{align*}
2x\frac{dx}{dt} + 2y\frac{dy}{dt} &= 0 \\
xu+yv &= 0 \\
u &= -v\frac{y}{x} = \boxed{-v\tan{\alpha}}
\end{align*}

Now, we find the acceleration
$$a = \frac{dv}{dt} = \frac{d(u\tan{\alpha})}{dt}$$

Since $u$ is constant, we have
$$a = u\frac{d\tan{\alpha}}{dt} = u \sec^2{\alpha} \frac{d\alpha}{dt}$$

The angular velocity $\dfrac{d\alpha}{dt}$ is simply
$$\frac{d\alpha}{dt} = \frac{u\cos{\alpha} + v\tan{\alpha} \sin{\alpha}}{L} = \frac{u}{L\cos{\alpha}}$$

This means that
$$a=u \sec^2{\alpha} \frac{d\alpha}{dt} = \boxed{\frac{u^2}{L\cos^3{\alpha}}}$$
\end{solution}
\begin{solution}{hard}
\begin{center}
    \begin{asy}
    /* Geogebra to Asymptote conversion, documentation at artofproblemsolving.com/Wiki go to User:Azjps/geogebra */
import graph; size(10cm);
real labelscalefactor = 0.5; /* changes label-to-point distance */
pen dps = linewidth(0.7) + fontsize(10); defaultpen(dps); /* default pen style */
pen dotstyle = black; /* point style */
real xmin = -19.2, xmax = 19.2, ymin = -9.16, ymax = 9.16; /* image dimensions */
pen zzttqq = rgb(0.6,0.2,0); pen qqwuqq = rgb(0,0.39215686274509803,0);

filldraw((-4.330127018922193,-0.5)--(-4.330127018922193,-3.5)--(-9.526279441628825,-3.5)--(-9.526279441628825,-0.5)--cycle, gray(0.8));
draw((0.8660254037844388,-0.5)--(0.8660254037844386,-3.5)--(6.06217782649107,-3.5)--(6.06217782649107,-0.5)--cycle);
draw(arc((-6.928203230275509,-2),0.6,180,227.22341007114423)--(-6.928203230275509,-2)--cycle, qqwuqq);
draw(arc((-1.7320508075688772,1),0.6,-179.05144115063317,-134.16834724231848)--(-1.7320508075688772,1)--cycle, qqwuqq);
/* draw figures */
draw(circle((-1.7320508075688772,1), 3));
label("$\sqrt{2}v/2$",(-3.2,0.1),2S*labelscalefactor);
label("$\sqrt{2}v/2$",(-7.76,-2.52),SE*labelscalefactor);
label("$v$",(-8.7,-1.02),4S*labelscalefactor);
draw((-4.330127018922193,-0.5)--(-4.330127018922193,-3.5), zzttqq);
draw((-4.330127018922193,-3.5)--(-9.526279441628825,-3.5), zzttqq);
draw((-9.526279441628825,-3.5)--(-9.526279441628825,-0.5), zzttqq);
draw((-9.526279441628825,-0.5)--(-4.330127018922193,-0.5), zzttqq);
draw((0.8660254037844388,-0.5)--(0.8660254037844386,-3.5), zzttqq);
draw((0.8660254037844386,-3.5)--(6.06217782649107,-3.5), zzttqq);
draw((6.06217782649107,-3.5)--(6.06217782649107,-0.5), zzttqq);
draw((6.06217782649107,-0.5)--(0.8660254037844388,-0.5), zzttqq);
draw((-1.7320508075688772,1)--(-2.82,-0.12),EndArrow(4));
draw((-1.7320508075688772,1)--(-2.94,0.98),EndArrow(4));
draw((-1.7320508075688772,1)--(-1.72,-0.14),EndArrow(4));
draw((-6.928203230275509,-2)--(-8.660254037844386,-2),EndArrow(4));
draw((-6.928203230275509,-2)--(-8.02,-3.18), EndArrow(4));
/* end of picture */
    \end{asy}
\end{center}
Let us denote the horizontal velocity of the block as $v$. When the distance between the block and the step is $\sqrt{2}r$, the cylinder pushes on the block at an angle of $45^\circ$. By trigonometry, we see that the cylinder would have to push on the block with a velocity of $\sqrt{2}v/2$ for the block to move horizontally with a velocity $v$. Now it is easy to see that velocity of cylinder is just
\[\vec{v_{c}} = -\frac{v_b}{2} \hat{i} - \frac{v_b}{2} \hat{j}\]where $v_b$ is the speed of the block (directed towards the negative x-axis). By energy conservation
\begin{align*}
{mg}\left(r-\frac{r}{\sqrt{2}}\right) &= \frac{1}{2} m {v_b}^2 + \frac{1}{2} m {v_c}^2
\end{align*}
Also project Newton’s 2nd law onto the axis that passes through the top corner of the step and the cylinder’s centre: this axis is perpendicular both to the normal force between the block and the cylinder and to the cylinder’s tangential acceleration.
\[\frac{mg}{\sqrt{2}} = N + \frac{m{v_c}^2}{r}\implies mg\frac{\sqrt{2}}{2} -N = m\frac{(\sqrt{2}v/2)^2}{r}\]
where $N$ is the normal force by the wall. Now, we solve these systems of equations for $N$. In our first equation, we have 
\begin{align*}
mgr\left(\frac{2-\sqrt{2}}{2}\right)&= \frac{1}{2}mv^2 + \frac{1}{2}m\left(\frac{\sqrt{2}v}{2}\right)^2\\
&=\frac{3}{2}m\left(\frac{\sqrt{2}v}{2}\right)^2
\end{align*}
Taking out common factors from both sides gives us 
\[gr(\sqrt{2}-2) = 3\left(\frac{\sqrt{2}v}{2}\right)^2\implies \frac{g(2-\sqrt{2})}{3} = \frac{(\sqrt{2}v/2)^2}{r}.\]
Substituting this result into our conservation of energy equation gives us 
\[mg\frac{\sqrt{2}}{2} -N = m\frac{g(2-\sqrt{2})}{3}\]
Solving for $N$ gives us the result 
\[N = \left(\frac{\sqrt{2}}{2} - \frac{2-\sqrt{2}}{3}\right)mg = \boxed{\left(\frac{5\sqrt{2} - 4}{6}\right)mg}.\]
Let the normal force from the other block be $Q$. From here we can project Newton's Second Law onto the cylinder and block on the horizontal direction (and noting that the aceleration of the cylinder is half that of the block because it's horizontal velocity is half that of the blocks velocity) gives us
\begin{align*}
m(a/2) &= N\sin\theta - Q\sin\theta\\
ma &= Q\sin\theta
\end{align*}
where $\theta$ is the angle the normal force makes with respect to the vertical. Substituting the second equation into the first gives us 
\[\frac{1}{2}Q\sin\theta = N\sin\theta - Q\sin\theta\implies \frac{3}{2}Q = N\]
Therefore the ratio between the two normal forces are $\frac{Q}{N} = \frac{2}{3}$. As mentioned in the hint, the ratios of the normals is fixed, hence they blow up at the same instant. (only differing by a constant factor) which means that they would give $0$ at the same values.
\blfootnote{This problem was found in the book 'Aptitude Test Problems in Physics by S.S. Krotov.}
\end{solution}

\begin{solution}{normal}
Consider the time when the angle between the line joining the edge to the rod makes an angle $\theta$, with the vertical. We want that the normal should always be positive (outwards).
The velocity of the rod at this time, $\nu$, can be calculated using energy conservation $\frac{1}{2} m \nu^2 = \frac{1}{2}mv^2 + mgR(1-\cos \theta)$.
Setting total forces along the line joining the rod to the edge to $0$, we get that
\begin{align*}
\frac{m\nu^2}{R} + N &= mg \cos \theta \\
N &= mg \cos \theta - \frac{2\cdot \dfrac{1}{2}m\nu^2}{R}\\
&= mg \cos \theta - \dfrac{mv^2 + 2mgR(1-\cos \theta)}{R} \\
0 &\le 3mg \cos \theta - \frac{mv^2 + 2mgR}{R} \\
\cos \theta &\ge\frac{1}{3}\left(2+\frac{v^2}{gR}\right)
\end{align*}
for all values of $\theta$ that can be achieved. The maximum value of $\theta$ will be alpha, so 
\[\cos \alpha\ge\frac{1}{3}\left(2+\frac{v^2}{gR}\right)
\] is a necessary and sufficient condition.
\end{solution}

\begin{solution}{normal}
\blfootnote{This problem was found in the book 'Aptitude Test Problems in Physics by S.S. Krotov.}First, we choose a frame that we will work from in this problem. To cancel out as many variables as possible, we should work in the frame of the large block when it is set into motion. 
\vspace{2mm}

From reference of the bottom of the circular cavity, the initial potential energy of the small block at the top is $mgr$. When it gets to the bottom of the circular cavity, it gains a kinetic energy of $\frac{1}{2}mv^2$. By conservation of energy we get
\[\frac{1}{2}mv^2=mgr\implies v=\sqrt{2gr}.\]
When the small block is at the bottom of the cavity, it will move backwards with a velocity $v_1$ in the reference frame of the big block, while the big block itslef moves with a velocity $v_2$ forwards. Thus, conservation of momentum and energy gives us
\begin{align*}
Mv_1-mv_2=m\sqrt{2gr}\\
\frac{1}{2}Mv_1^2+\frac{1}{2}mv_2^2=mgr
\end{align*}
From our conservation of momentum equation we have
\[v_2=\frac{M}{m}v_1-\sqrt{2gr}\]
Thus by substituting $v_2$ back into our conservation of momentum equation we result in 
\begin{align*}
\frac{1}{2}Mv_1^2+\frac{1}{2}m\left(\frac{M}{m}v_1-\sqrt{2gr}\right)^2=mgr\\
\frac{1}{2}Mv_1^2+\frac{1}{2}m\left(\frac{M^2}{m^2}v_1^2+2gr-\frac{2M}{m}v_1\sqrt{2gr}\right)=mgr
\end{align*}
Expanding $\frac{1}{2}m$ inside gives
\[\frac{1}{2}Mv_1^2+\frac{1}{2}\frac{M^2}{m}v_1^2+mgr-Mv_1\sqrt{2gr}=mgr\]
Taking away $mgr$ from both sides and dividing both sides by $Mv_1$ gives us
\[\frac{1}{2}v_1+\frac{M}{2m}v_1-\sqrt{2gr}=0\]
Factoring and taking $\sqrt{2gr}$ to the other side gives us
\begin{align*}
v_1\left(\frac{M+m}{2m}\right)=\sqrt{2gr}\\
\boxed{v_1=2\frac{m}{M+m}\sqrt{2gr}}
\end{align*}
\end{solution}

\begin{solution}{normal}
\begin{center}
    \begin{asy}
    size(5cm);
    import olympiad;
    draw(circle((0,0), 1));
draw((-1,-1) -- (1, -1));
dot((0,0));
draw((0,0) -- (-sqrt(3)/2, 1/2));
draw((-sqrt(3)/2, 1/2) -- (-1/2, sqrt(3)/2 + 0.2), red, arrow=Arrow(4));
draw((0,0) -- (-1/2, sqrt(3)/2 + 0.2), dashed);
label("$R'$", (-1/4, sqrt(3)/4 + 0.1), NE);
label("$\Omega Rt$", (-1/2, sqrt(3)/2 + 0.2), N);
label("$R$", (-sqrt(3)/4, 1/4), SW);
draw(anglemark((-1/2, sqrt(3)/2 + 0.2), (0,0), (-sqrt(3)/2, 1/2)));
label("$\alpha$", (0,0), 5NW);
    \end{asy}
\end{center}
In the free-falling frame, all the particles move with constant velocities; each particle had initial velocity equal to the wheel’s velocity at the releasing point, i.e. tangential to the wheel and equal by modulus to $\Omega R$. Hence the ensemble of particle expands as a circle, the radius of which can be calculated from the Pythagorean theorem.
\[R'^2 = R^2 + \Omega^2 R^2 t^2\implies R' = R\sqrt{1 + \Omega^2 t^2}\]In the lab frame, the centre of the circle performs a free fall $d = \frac{1}{2}gt^2 - R$. A droplet reaching the point A corresponds to the expanding circle touching the ground. Therefore, setting $R' = d$ gives us
\begin{align*}
R\sqrt{1 + \Omega^2 t^2} &= \frac{1}{2}gt^2 - R\\
\frac{1}{4}g^2 t^2 - gR t^2 + R^2 &= R^2 + R^2 \Omega^2 t^2\\
\frac{1}{4}g^2 t^2 - gR &= R^2\Omega^2\\
t^2 = \frac{4\left(\Omega^2 R^2 + gR\right)}{g^2} &\implies \boxed{t = 2\sqrt{\frac{R}{g}\left(1 + \frac{R\Omega^2}{g}\right)}}
\end{align*}
We can also tell from the given diagram that
\[\alpha = \arctan\left(\frac{\Omega Rt}{R}\right) \implies \boxed{\alpha = \arctan\left(\Omega t\right)}\]
\end{solution}
\newpage
\begin{solution}{hard}
When the pulley is let go, one side of the rope will go up a distance $\xi$ while the other side will go down a distance $\xi$. The change in potential energy of this is the categorized as 
\[\Pi(\xi) = -\rho g(L - 2l - \pi R)\xi \implies -\Pi'(\xi) = \rho g(L - 2l - \pi R)\]
The kinetic energy of the system will then be 
\[K = \frac{1}{2}m\dot\xi^2 + \frac{1}{2}\rho L\dot\xi^2\]
This implies that the effective mass, $\mathcal{M}$, is $\mathcal{M} = m + \rho L$. We then get the acceleration of the system to be 
\[a \equiv \frac{\rho g(L - 2l - \pi R)}{m + \rho L}\]
Now, we write for the displacement of parts of the system times their mass divided by the total effective mass of the system $\mathcal{M}$. Differentiating that with respect to $\xi$ will give us our accelerations in the $x$ and $y$ direction. In the $x$ direction, we have
\begin{align*}
x &= \frac{2R\xi\rho}{m + \rho L}\\
a_x &= \frac{2R\rho a}{m + \rho L}
\end{align*}
In the $y$ direction, we have 
\begin{align*}
y &= \frac{(L - 2l -\pi R)\rho\xi}{m + \rho L}\\
a_y &= \frac{(L - 2l - \pi R)\rho a}{m + \rho L}
\end{align*}
By $F=ma$, we have the direction of force in the $x$ and $y$ direction to then be 
\[\boxed{F_x = 2R\rho a}\]
\begin{align*}
F_y - (m+\rho L)g = -\rho a (L - \pi R - 2l)\\
\boxed{\therefore F_y = -\rho a(L-\pi R - 2l) + (m + \rho L)g}
\end{align*}
\end{solution}

\begin{solution}{normal}
Assuming that idea 2 is correct, it may be proved that between the quantity of heat $Q$, which in a cyclical process-of the kind described above is transformed into work (or, where the process is in the reverse order, generated by work), and the quantity of heat $Q_2$ which is transferred at the same time from a hotter to a colder body Cor vice versa), there exists a relation independent of the nature of the variable body which acts as the medium of the transformation and transfer; and thus that, if several cyclical processes are performed, with the same reservoirs of heat $K_1$ and $K_2$ , but with different variable bodies, the ratio 3. will be the same for all. If we suppose the processes so arranged, according to their magnitude, that the quantity of heat $Q$, which is transformed into work, has in all of them a constant value, then we have only to consider the magnitude of the quantity of heat $Q_2$. which is transferred, and the principle which is to be proved takes the following form: 
\vspace{3mm}

If where two different variable bodies are used, the quantity of heat $Q$ transformed into work is the same, then the quantity of heat $Q_2$ which is transferred, will also be the same.
\vspace{3mm}

Let there, if possible, be two bodies $\text{A}$ and $\text{A}'$ (e.g. the perfect gas and the combined mass of liquid and vapour, described above) for which the values of $Q$ are equal, but those of the transferred quantities of heat are different, and let these different values be called $Q_2$, and $Q'_2$, respectively: $Q_2'$, being the greater of the two. Now let us in the first place subject the body a to a cyclical process, such that the quantity of heat $Q$ is transformed into work, and the quantity $Q$ is transferred from $K_2$ to $K_1$. Next let us subject $\text{A}'$ to a cyclical process of the reverse description, so that the quantity of heat $Q$ is generated out of work, and the quantity $Q_2'$. is transferred from $K_2$ to $K_1$. Then the above two changes, from heat into work, and work into heat, will cancel each other since we may suppose that when in the first process the heat $Q$ has been taken from the body $K_1$ and transformed into work, this same work is expended in the second process in producing the heat $Q$, which is then returned to the same body $K_1$. In all other respects also the bodies will have returned, at the end of the two operations, to their original condition, with one exception only. The quantity of heat $Q_2'$ transferred from $K_1$, to $K_2$ has been assumed to be greater than the quantity $Q_2$ transferred from $K_1$ to $K_2$. Hence, these two do not cancel each other, but there remains at the end a quantity of heat, represented by the difference $\Delta Q_2$, which has passed over from $K_1$ to $K_2$. Hence a passage of heat will have taken place from a colder to a warmer body without any other compensating change. But this contradicts the fundamental principle. Hence the assumption that $Q_2'$ is greater than $Q_2$, must be false. 
\vspace{3mm}

Again, if we make the opposite assumption, that $Q_2'$, is less than $Q_2$ we may suppose the body $\text{A}'$ to undergo the cyclical process in the first, and a in the reverse direction. We then arrive similarly at the result that a quantity of heat $Q_2- Q_2'$, has passed from the colder body $K_2$ to the hotter $K_1$ which is again contrary to the principle.
\vspace{3mm}

Since then $Q_2'$, can be neitlier greater nor less than $Q_2$, it must be equal to $Q_2$; which was to be proved. We will now give to the result thus obtained the mathematical form most convenient for our subsequent reasoning. Since the quotient $Q/Q_2$ is independent of the nature of the variable body (fact 2), it can only depend on the temperature of the two bodies $K_1$ and $K_2$ which act as heat reservoirs. The same will of course be true of the sum
\[1 + \frac{Q}{Q_2} = \frac{Q + Q_2}{Q_2} = \frac{Q_1}{Q_2}.\]
This last ratio, which is that between the whole heat received and the heat transferred, we shall select for further consideration; and shall express the result obtained in this section as follows: \vspace{3mm}

The ratio $Q_1/Q_2$ can only depend on the temperatures $T_1$ and $T_2$.
\vspace{3mm}

This leads to the equation:
\[\frac{Q_1}{Q_2} = f(T_1, T_2).\]
Since the process is isothermal, we can obtain the equation
\[\frac{Q_1}{Q_2} = \dfrac{nRT_1\ln \frac{V_f}{V_i}}{nRT_2 \ln \frac{V_f}{V_1}} = \frac{T_1}{T_2}.\]
Using this definition, the Carnot's Cycle efficiency can then be rewritten as 
\[\eta_C = 1 - \frac{Q_2}{Q_1} = 1 - \frac{T_2}{T_1}.\]
$\square$
\blfootnote{Part of this solution is taken from Rudolf Clausius' original work "The Mechanical Theory of Heat".}
\end{solution}
\begin{solution}{normal}
According to Prevost’s theory of exchange, in order to maintain thermal equilibrium, any object must emit the same energy as it receives. Thus the absorption and emission of light must be the same at all frequencies.
\vspace{3mm}

\noindent A consequence of this problem is that radiation and absorption properties of a material must be identical throughout the entire spectrum. It can be similarly shown that a partially reflecting material must have equal transmittance from both sides. It may seem that dark window glasses are more transparent when looking from inside of a darkly lit room, but this is a mere illusion: when looking from outside, a small fraction of reflected abundant outside light can easily dominate over the transmitted part of the light coming from inside, but the opposite is not true. The effect can be enhanced by overlaying an absorbing and reflecting layers and turning the reflecting layer outside. Then, while total transmittance is equal from both sides, the reflectance from outside is larger because from outside, reflected light does not pass through the absorbing layer.
\end{solution}
\begin{solution}{normal}
The angle two equal masses make after an elastic collision will be a right angle, and thus if the stationary ball is placed on a semi-circle where the two holes form the diameter, then the description is possible according to Thales' Theorem.
\end{solution}

\begin{solution}{normal}
Let the length of the rope be $L$, let $AO=x$, and let $A'O'=y$ as shown in the diagram below.

\begin{center}
    \begin{asy}
        unitsize(1cm);
        draw((0,0)--(0,6.3),linewidth(2));
        draw((6,0)--(6,6.3),linewidth(2));
        draw(ellipse((0,1),0.2,0.1));
        draw(ellipse((6,4.3),0.2,0.1));
        draw((0.2,1)--(5.9,4.3)--(5.9,6.1)..(5.9,6.2)..(5.92,6.3)..(5.95,6.3)..(6,6.32));
        label("$A'$",(6,6.3),W*2);
        label("$A$",(0,6.3),E*2);
        label("$B'$",(6,0),W*2);
        label("$B$",(0,0),E*2);
        label("$O$",(0,1.3),W*2);
        label("$O'$",(6,4.6),E*2);
        draw(arc((0,1),0.8,83,27));
        label("$\alpha$",(0.7,1.9));
        draw("$x$",(-0.8,6.3)--(-0.8,1),arrow=Arrows(),bar=Bars(),W*2);
        draw("$y$",(7,6.3)--(7,4.3),arrow=Arrows(),bar=Bars(),E*2);
        draw("$b$",(0,-0.5)--(6,-0.5),arrow=Arrows(),bar=Bars(),S*2);
        draw((6.3,4.3)--(6.3,3.3),arrow=Arrow());
        label("$v$",(6.7,3.8));
    \end{asy}
\end{center}

Also let $v_O$ be the velocity of ring $O$.\vspace{3mm}

By the Pythagorean Theorem, we have that
$$(x-y)^2+b^2=(L-y)^2$$

Taking the derivative, we get that
\begin{align*}
\dfrac{d}{dt}(x-y)^2+\dfrac{d}{dt}b^2&=\dfrac{d}{dt}(L-y)^2\\
2(x-y)\left(\dfrac{dx}{dt}-\dfrac{dy}{dt}\right)&=-2(L-y)\cdot\dfrac{dy}{dt}\\
(v_O-v)\left(\dfrac{x-y}{L-y}\right)&=-v\\
(v_O-v)\cos\alpha&=-v\\
v_O&=\boxed{v\left(1-\dfrac{1}{\cos\alpha}\right)}
\end{align*}

Note that this velocity is directed upwards along the rail, hence its negative value.\vspace{3mm}

Note that 
\begin{align*}
\cot\alpha &= \frac{x-y}{b}\\
-\frac{d\alpha}{dt}\csc^2\alpha &= b\cdot \frac{d}{dt}(x-y)\\
\frac{d\alpha}{dt} &= \frac{v}{b}\frac{\sin^2\alpha}{\cos\alpha}
\end{align*}
Since we have
\begin{align*}
\cot\alpha &= \frac{x-y}{b}\\
-\frac{d\alpha}{dt}\csc^2\alpha &= b\cdot \frac{d}{dt}(x-y)\\
\frac{d\alpha}{dt} &= \frac{v}{b}\frac{\sin^2\alpha}{\cos\alpha}
\end{align*}
We get the acceleration of point O as
$$a_O = \frac{dv_O}{dt} = v\sec{\alpha}\tan{\alpha} \frac{d\alpha}{dt} = \boxed{\frac{v^2}{b} \tan^3{\alpha}}$$
\end{solution}
\begin{solution}{normal}
The ball can escape the well if at the time at which the ball is at maximum height, it collides with the wall.\vspace{3mm}

The time between any 2 collisions with the wall is $\dfrac{2R\cos\alpha}{v_0}$.\vspace{3mm}

The total time of flight of 1 parabola is $2\sqrt{\dfrac{2H}{g}}$. This means that the required condition is then 
$$2p\sqrt{\frac{2H}{g}} = \dfrac{2qR\cos\alpha}{v_0}\text{, or}$$
$$\boxed{pv_0\sqrt{\frac{2H}{g}} = qR\cos\alpha,\;\;\;p,q\in\mathbb{N}}$$
\end{solution}
\begin{solution}{normal}
The velocity of the small block in the center of mass frame is $\frac{Mv}{m+M}$ while the velocity of the large block in the center of mass frame is $\frac{mv}{m+M}$. The work done by friction is $\mu mgL$, thus we can create a conservation of energy equation
\begin{align*}
\frac{1}{2}m\left(\frac{Mv}{m+M}\right)^2 + \frac{1}{2}M\left(\frac{mv}{m+M}\right)^2 &= \mu mgL\\
v^2\left(\frac{mM^2 + Mm^2}{(m+m)^2}\right)=2\mu mgL\implies v^2\left(\frac{mM(m+M)}{(m+m)^2}\right)&=2\mu mgL\\
\boxed{v=\sqrt{2\mu gL\left(1+\frac{m}{M}\right)}}
\end{align*}
\end{solution}


\newpage
\section{Solutions to Revision Problems}
This section will contain problem 38-86 of the handout. Revision problems take concepts and ideas from earlier problems and places them in a new context. As a result, many of the problems in this section will seem familiar. This however, does not mean that all the problems in this section are easy. Some of the hardest problems originate in this section.
\begin{solution}{normal}
At that moment, the acceleration of the dog is equal to
$$\vec{a} = \frac{d{\vec{v_2}}}{dt} = v_2\frac{ d\theta}{dt} \hat{r}$$

By Idea 37, we have the angular velocity of the relative position vector as 
$$\frac{d\theta}{dt} = \frac{v_1\sin{90^\circ}}{\ell} = \frac{v_1}{\ell}$$

Hence, from the two relations, the acceleration of the dog is
$$a = \boxed{\frac{v_1 v_2}{\ell}}$$
\end{solution}
\begin{solution}{normal}
The key difference between the barrels is that the walls in barrel provide a non-zero momentum to every small elemental mass that exits through the tap, while the other does not. Some non-zero work is done by the force exerted by these walls on the water molecules, which is not true for the other . So we first write the conservation of energy equation for the barrel: Let the small $dm$ mass of water element exit at a velocity $v_1$ from the tap of the barrel,$$\frac{1}{2} dm {v_1}^2 = dm g H$$and by impulse momentum theorem,$$F_{\text{walls}} = dm v_2 \implies (\rho g A_0 H)dt = (\rho A_0 v_2 dt) v_2$$From these two equations, we have the answers $\boxed{\sqrt{2gH}}$ and $\boxed{\sqrt{gH}}$.
\end{solution}

\begin{solution}{normal}
The rope will intuitively be something like a spiral.\vspace{3mm} 

Since the rope can go up to infinity, let's consider the last point instead. We set the point where all the shockwaves coincide at the origin and we use polar coordinates since we are going to be dealing with distances.\vspace{3mm}

The rope lies along the curve $r(\theta)$, where $r(0)$ is the last point to be ignited. It takes time $r(0)/c$ for the shockwave from the last ignition point to reach the origin. If we go back an angle $d\theta$ along the rope, then it takes time $$\dfrac{r(d\theta)}{c}-\dfrac{ds}{v}$$
for that shockwave to reach the origin, where $ds$ is the infinitesimal arc length. Note that \[ds=\sqrt{r^2d\theta^2+dr^2}.\] 

We can set up our differential equation from this knowledge. For each $r(\theta)$, we want \[\dfrac{r(\theta+d\theta)-r(\theta)}{c}=\dfrac{ds}{v}\]

Divide both sides by $d\theta$ to get \[r'/c=\sqrt{r^2+r'^2}/v\]

Square both sides to get \[\dfrac{r'^2}{c^2}=\dfrac{r^2}{v^2}+\dfrac{r'^2}{v^2}\]

Combining like terms and simplifying gives us \[\frac{dr}{d\theta}=r\sqrt{\frac{c^2}{v^2-c^2}}\]

This is a separable differential equation which gives us solution \[r=\boxed{Ce^{\sqrt{\dfrac{c^2}{v^2-c^2}}\theta}}\]
\end{solution}
\newpage
\begin{solution}{normal}
The velocity of the blob just when the blob is about to hit the surface is found by conservation of mechanical energy$$\frac{1}{2} m v^2 = mgh \implies v= \sqrt{2gh}$$The Impulse (change in momentum) imparted perpendicular to the blob is clearly$$\Delta{p_{\perp}} = m\sqrt{2gh}$$From idea 60,$${\Delta{p_{\perp}}} = \int{\mu Ndt} =  \mu{\Delta{p_{\parallel}}}  \implies \Delta{p_{\perp}} = \mu m\sqrt{2gh}$$Hence$$\boxed{v = u - \mu \sqrt{2gh}}$$
\end{solution}

\begin{solution}{normal}
Consider the following setup:
\begin{center}
    \begin{asy}
        unitsize(1.2cm);
        pair A = (0,0);
        pair B = (8,0);
        real a = 25*pi/180;
        real b = 60*pi/180;
        real v1 = 4;
        real v2 = 3;
        dot(A);
        dot(B);
        label("$A$",(-0.3,0));
        label("$B$",(8.3,0));
        draw("$l$",(0,-0.2)--(8,-0.2),Arrows,Bars);
        label("$\vec{v_1}$",(v1+0.3)*(cos(a),sin(a)));
        label("$\vec{v_2}$",B+(v2+0.3)*(-cos(b),sin(b)));
        draw(A--B);
        draw(A--v1*(cos(a),sin(a)),arrow=Arrow());
        draw(B--B+v2*(-cos(b),sin(b)),arrow=Arrow());
        draw(arc(A,0.7,0,a*180/pi));
        draw(arc(B,0.7,180,180-b*180/pi));
        label("$\alpha$",(cos(a/2),sin(a/2)));
        label("$\beta$",B+(-cos(b/2),sin(b/2)));
    \end{asy}
\end{center}
Now, we move into the reference frame of the boat that departed from harbour $A$.
\begin{center}
    \begin{asy}
        import olympiad;
        unitsize(1.2cm);
        pair A = (0,0);
        pair B = (8,0);
        real a = 25*pi/180;
        real b = 60*pi/180;
        real v1 = 4;
        real v2 = 3;
        dot(A);
        dot(B);
        label("$A$",(-0.3,0));
        label("$B$",(8.3,0));
        // draw("$l$",(0,-0.2)--(8,-0.2),Arrows,Bars);
        label("$\vec{v_1}$",B+(v1+0.3)*(-cos(a),-sin(a)));
        label("$\vec{v_2}$",B+(v2+0.3)*(-cos(b),sin(b)));
        draw(A--B);
        draw(B--B+v1*(-cos(a),-sin(a)),arrow=Arrow());
        draw(B--B+v2*(-cos(b),sin(b)),arrow=Arrow());
        draw(B--B+(v2*(-cos(b),sin(b))+v1*(-cos(a),-sin(a)))*1.7,dashed);
        draw(A--B+(v2*(-cos(b),sin(b))+v1*(-cos(a),-sin(a)))*1.508,dashed);
        draw(B--B+v2*(-cos(b),sin(b))+v1*(-cos(a),-sin(a)),arrow=Arrow());
        draw(arc(B,0.7,180,180+a*180/pi));
        draw(arc(B,0.7,180,180-b*180/pi));
        draw(arc(B,3.2,180,170));
        label("$\alpha$",B+(-cos(a/2),-sin(a/2)));
        label("$\beta$",B+(-cos(b/2),sin(b/2)));
        label("$\phi$",B+3.5*(-cos(5*pi/180),sin(5*pi/180)));
        label("$v_{rel}$",(B+v2*(-cos(b),sin(b))+v1*(-cos(a),-sin(a)))*1.2);
        draw(rightanglemark(A,B+(v2*(-cos(b),sin(b))+v1*(-cos(a),-sin(a)))*1.508,B));
        label("$d$",(-0.2,0.75));
    \end{asy}
\end{center}
Since $\vec{v_{rel}}=\vec{v_2}-\vec{v_1}$, we can separate them into components to find that
$$\tan\phi=\dfrac{\left| v_1\sin\alpha-v_2\sin\beta\right|}{\left| v_1\cos\alpha+v_2\cos\beta\right|}$$

From this, we can find that
$$\sin\phi=\dfrac{\left| v_1\sin\alpha-v_2\sin\beta\right|}{\sqrt{v_1^2+v_2^2+2v_1v_2\cos(\alpha+\beta)}}$$

We then have
\begin{align*}
d&=l\sin\phi\\
&=\boxed{\dfrac{l\cdot\left| v_1\sin\alpha-v_2\sin\beta\right|}{\sqrt{v_1^2+v_2^2+2v_1v_2\cos(\alpha+\beta)}}}
\end{align*}
\end{solution}
\begin{solution}{hard}
Because the ropes are constantly being unwinded at a rate of $\omega{R}$ , the disk has to move in the direction of the strings to keep the strings in tension the whole time. Using this information we can create a diagram
\begin{center}
    \begin{asy}
        import olympiad;
        size(5cm);
        real a = 15*pi/180;
        draw((-2,2)--(2,-2),arrow=Arrow());
        draw((-2,2)--(-sqrt(8)*cos(a),-sqrt(8)*sin(a)),arrow=Arrow());
        draw((-2,2)--(sqrt(8)*sin(a),sqrt(8)*cos(a)),arrow=Arrow());
        draw((-sqrt(8)*cos(a),-sqrt(8)*sin(a))--(2,-2),arrow=Arrow(4));
        draw((sqrt(8)*sin(a),sqrt(8)*cos(a))--(2,-2),arrow=Arrow(4));
        label("$\omega R$",((-2,2)+(-sqrt(8)*cos(a),-sqrt(8)*sin(a)))/2,NW*1.5);
        label("$\omega R$",((-2,2)+(sqrt(8)*sin(a),sqrt(8)*cos(a)))/2,N*1.5);
        draw(rightanglemark((-2,2),(-sqrt(8)*cos(a),-sqrt(8)*sin(a)),(2,-2)));
        draw(rightanglemark((-2,2),(sqrt(8)*sin(a),sqrt(8)*cos(a)),(2,-2)));
        label("$v$",(0,0),N*3);
        draw(arc((-2,2),0.5,-45,15));
        label("$\alpha/2$",(-1,1.8));
    \end{asy}
\end{center}
Now using trig we can see that $\cos\dfrac{\alpha}{2}=\dfrac{\omega{R}}{v}$ or $v=\boxed{\dfrac{\omega{R}}{\cos\frac{\alpha}{2}}}$.
\end{solution}
\begin{solution}{normal}
\textbf{a)} The angular momentum of the rod about the end of it's axis before the collision is defined by
\[L_0=Mvl-\frac{1}{3}Ml^2\omega.\]
After the rod collides with the post it's angular momentum right after impact is 
\[L_a=Mv'l-\frac{1}{3}Ml^2\omega'.\]
Since angular momentum is conserved in the entire process we have
\[L_0=L_a\implies Mvl-\frac{1}{3}Ml^2\omega=Mv'l-\frac{1}{3}Ml^2\omega'\implies v-\frac{1}{3}\omega l=v'-\frac{1}{3}\omega' l.\]
We know that the condition for the rod being at the end, is the relation 
\[
v' + l\omega' = 0\implies \omega' = -\frac{v'}{l}
\]Substituting our relation of $\omega'$ and $v'$ into our simplified angular momentum equation gives us
\begin{align*}
v-\frac{1}{3}\omega l = \frac{4}{3}v'\\
\boxed{v' = \frac{3v - \omega l}{4}}
\end{align*}

\textbf{b)} From part a) we have the equation 
\[v-\frac{1}{3}\omega l=v'-\frac{1}{3}\omega' l.\]
The kinetic energy before is 
\[K=\frac{1}{2}Mv^2+\frac{1}{2}\left(\frac{1}{3}Ml^2\right)\omega^2=\frac{1}{2}Mv^2+\frac{1}{6}Ml^2\omega^2.\]
Therefore we have to equations of conservation of kinetic energy and angular momentum
\begin{align*}
3v-\omega l=3v'-\omega' l\\
3v^2+\omega^2 l^2=3v'^2+\omega'^2 l^2
\end{align*}
rearranging both of these equations and factoring gives us two new equations of
\begin{align*}
3(v-v')=l(\omega-\omega')\\
3(v^2-v'^2)=l^2(\omega'^2-\omega^2)
\end{align*}
Solving these equations gives us $\boxed{v'=\frac{v-\omega l}{2}}$.
\end{solution}

\begin{solution}{normal}
First, note that $L=2\pi rk$
\vspace{3mm}

By idea 21 (tension is perpendicular to direction of motion), the velocity $v$ of the block remains constant throughout the motion. \vspace{3mm}

Let $l$ be the length of the portion of the string not in contact with the cylinder. \vspace{3mm}

The angular velocity about the point of tangency with the cylinder is $\omega=\dfrac{v}{l}$. \vspace{3mm}

Note that $r\dfrac{d\theta}{dt}=\dfrac{dl}{dt}$.

$$r\omega=\dfrac{dl}{dt}\implies \dfrac{rv}{l}=\dfrac{dl}{dt} \implies rv=l\dfrac{dl}{dt}$$
$$rv=\dfrac{1}{2}\dfrac{d(l^2)}{dt}\implies l^2=2rvt\text{ since }l_0=0$$
$$t=\dfrac{l^2}{2rv}=\dfrac{2\pi^2 rk^2}{v}$$

Note that the string also completes an additional semicircle without changing length before starting to wrap back around again.
\begin{center}
    \begin{asy}
        unitsize(8mm);
        real r = 1.5;
        draw(circle((0,0),r));
        draw((0,0)--(0,r));
        label("$r$",(0.2*r,r/2));
        filldraw((0,r)--(-4*r,r)--(-4*r,1.05*r)--(0,1.05*r)--cycle, grey);
        label("$L$",(-2*r,1.4*r));
        draw(arc((0,r),4*r,0,180),dashed);
    \end{asy}
\end{center}
Therefore, our final time is $t=2\cdot\dfrac{2\pi^2rk^2}{v}+\dfrac{\pi(2\pi rk)}{v}$, or
$$\boxed{t=\dfrac{2\pi^2 kr(2k+1)}{v}}$$
\end{solution}
\begin{solution}{normal}
If the velocity of the box is represented by the vector $v_3$, then the projection of $v_3$ onto $v_1$ must be equivalent to $v_1$ and the projection of $v_3$ onto $v_2$ must be equivalent to $v_2$.\vspace{3mm}

Essentially this means that $v_3$ must be composed of the sum of $v_1$ and a vector perpendicular to $v_1$ and likewise for $v_2$. This gives us the following diagram: 
\begin{center}
    \begin{asy}
    /* Geogebra to Asymptote conversion, documentation at artofproblemsolving.com/Wiki go to User:Azjps/geogebra */
import graph; usepackage("amsmath"); size(9cm);
real labelscalefactor = 0.5; /* changes label-to-point distance */
pen dps = linewidth(0.7) + fontsize(10); defaultpen(dps); /* default pen style */
pen dotstyle = black; /* point style */
real xmin = -1.5532606317668722, xmax = 8.835780467660939, ymin = -0.7315550874084362, ymax = 4.341456209680555; /* image dimensions */

/* draw figures */
draw((1,1)--(5,1), red,EndArrow(6));
draw((1,1)--(2,4), blue,EndArrow(6));
draw((2,4)--(5,3), blue,EndArrow(6));
draw((5,1)--(5,3), red,EndArrow(6));
draw((1,1)--(5,3),EndArrow(6));
draw((4.8,1)--(4.8,1.2));
draw((4.8,1.2)--(5,1.2));
draw((1.9477341752980104,3.843202525894031)--(2.1158723954262184,3.785753600795995));
draw((2.1158723954262184,3.785753600795995)--(2.1786596080429605,3.940446797319013));
label("$v_1$",(1.3705557595194635,2.759288461072476),SE*labelscalefactor,blue);
label("$v_2$",(3.0261193265335447,0.8986780302539092),SE*labelscalefactor,red);
label("$\alpha$",(1.25,1.35),SE*labelscalefactor);
draw(shift((1,1))*xscale(0.3325841180741864)*yscale(0.3325841180741864)*arc((0,0),1,0,71.56505117707802));
draw(circle((3,2), 2.23606797749979), dashed);
draw((2,4)--(5,1), dashed);
/* dots and labels */
dot((1,1),dotstyle);
label("$A$", (0.7478208398169193,1.0151404190743), NE * labelscalefactor);
dot((5,1),dotstyle);
label("$D$", (5.091776621156618,0.8252822118479157), NE * labelscalefactor);
dot((2,4),dotstyle);
label("$B$", (2.031262320667285,4.075654719563616), NE * labelscalefactor);
dot((5,3),dotstyle);
label("$C$", (5.031021994844175,3.073203385408307), NE * labelscalefactor);
label("$v$",(2.9273930587758246,2.07895921308088),NE*labelscalefactor);
clip((xmin,ymin)--(xmin,ymax)--(xmax,ymax)--(xmax,ymin)--cycle);
/* end of picture */
    \end{asy}
\end{center}

We want to find the magnitude of $AC$. Since this quadrilateral is formed by two right triangles, it is a cyclic quadrilateral.\vspace{3mm}

Since $\angle BDA$ and $\angle BCA$ are inscribed angles of the same arc, they are congruent. Using the law of sines, we get that $$\dfrac{BD}{\sin\alpha}=\dfrac{AB}{\sin\angle BDA},\;\dfrac{AB}{\sin\angle BCA}=AC$$\vspace{3mm}

Since $$\angle BDA=\angle BCA,\;\dfrac{BD}{\sin\alpha}=AC$$

Using the law of cosines, $BD=\sqrt{v_1^2+v_2^2-2v_1v_2\cos\alpha}$, so $$AC=\boxed{\dfrac{\sqrt{v_1^2+v_2^2-2v_1v_2\cos\alpha}}{\sin\alpha}}$$

\tcbline
\textbf{Solution 2:} Similar to above, but let us denote $\angle CAD=\theta$ such that we have $v_2=v\cos\theta$ and $v_1=v\cos(\alpha-\theta)$. We can rewrite $v_1$, using the cosine addition formula as:
$$v_1=v\left(\cos\alpha\cos\theta-\sin\theta\sin\alpha\right)=v_2\cos\alpha-v\sin\theta\sin\alpha$$
We can solve for $\sin\theta$ to be:
$$\sin\theta = \frac{CD}{AD}=\frac{\sqrt{v^2-v_2^2}}{v}$$
Substituting this in, we get:
\begin{align*}
v_1 &= v_2\cos\alpha-v\left(\frac{\sqrt{v^2-v_2^2}}{v}\right) \\
(v_2\cos\alpha-v_1)^2&=(v^2-v_2^2)\sin^2\alpha \\
v_2^2\cos^2\alpha+v_1^2-2v_1v_2\cos\alpha &= v^2\sin^2\alpha-v_2^2\sin^2\alpha \\
\end{align*}
Rearranging and solving for $v$ gives:
$$v=\boxed{\frac{\sqrt{v_1^2+v_2^2-2v_1v_2\cos\alpha}}{\sin\alpha}}$$
\end{solution}
\begin{solution}{normal}\textbf{(a)} The wording of the temperatures $T_1, T_2$, and $T_0$ may be slightly confusing so let us clarify this before solving the problem. $T_2$ is the temperature of the cold air \textit{going} into the room while $T_0$ is the warm air from the room and $T_1$ is the cold air that is already inside the room. 
\vspace{3mm}

\noindent Let us assume that there is a constant difference of temperature across the opposite sides of the plates given by $\Delta T \equiv T_0 - T_2$. By fact 6, we note that for small tempera difference $\Delta T \equiv T_0 - T_2$, the heat flux is proportional to $\Delta T$. In other words, 
\[\dot{Q} \propto T_0 - T_2 = \text{const}.\]
The heat flux is also equal to 
\[\dot{Q} = mc_p \dot{T} = \rho V c_p \dot{T} = \rho sh c_p \dot{T}\]
where $s$ is the cross sectional area for an air element of volume $V$. We remember that the thermal conductance of the metal is $\sigma$ (the heat flux through a unit area of the plate per unit time, assuming that the temperature drops by one degree per unit thickness of the plate). This means that we can write 
\[\dot{Q} = \frac{\sigma s (T_0 - T_2)}{d}.\]
Since the heat flux is proportional to the difference in temeperature which is constant, this means that the temperature gradient is linear with respect to position. If the velocity of the air is $v$, we write with dimensional arguments that 
\[\dot{T} = \frac{v (T_2 - T_1)}{x}\]
we write $\Delta T \equiv T_2 - T_1$ here since we are looking at the temperature difference horizontally from the cold air going into the room and the cold air that is already in the room. Substitituting $\dot{T}$ into our initial expression of $\dot{Q}$ and equating that to our other expression with thermal conductance, we result in the equation 
\[\rho s h c_p \frac{v (T_2 - T_1)}{x} = \frac{\sigma s (T_0 - T_2)}{d}.\]
To solve this equation for $T_2$, we can cross multiply to get 
\[\rho shc_p (T_2 - T_1) = \sigma s x(T_0 -T_2)\implies T_2(\rho hc_p v + \sigma x) = \rho shc_pd T_1 + \sigma sx T_0.\]
Dividing over gives
\[T_2 = \frac{x\sigma T_0 + \rho hc_pdT_1}{x\sigma + \rho hc_p vd}.\]
\vspace{3mm}

\noindent \textbf{(b)} Since the tempera difference is very large, the temperature gradient is not linear by fact 6. This is because (a) heat conductivity of the materials may depend on the temperature, (b) the heat flux due to heat radiation is a non-linear function of $T_1$ and $T_2$ (however, it can be still linearized for small values of $\Delta T$); (c) large temperature differences may cause convection of air and fluids which will enhance heat flux in a nonlinear way. Therefore, we have to rely on the graph to carry out calculations. Note that by idea 1:
\[P \equiv \frac{\text{d}Q}{\text{d}T} = C \frac{\text{d}T}{\text{d}t}.\]
By integrating, we find that 
\[\int_{0}^{t} \text{d}t = \int_{T_2}^{T_1}\frac{C}{P}\text{d}T\implies t = 12C = 120\;\mathrm{s}.\]
\end{solution}
\begin{solution}{normal}\textbf{(a)} The adiabatic index is defined as the ratio of $c_p$ and $c_V$ so that $\gamma = \frac{c_p}{c_V}$. Note that 
\[c_p = c_v + R\implies c_v = c_p - R\]
which means that upon substitution 
\[\gamma = \frac{c_p}{c_p - R}\implies \gamma (c_p - R) = c_p\implies c_p = \frac{\gamma}{\gamma - 1}R.\]
\vspace{3mm}

\noindent \textbf{(b)} Remember that from the ideal gas law, we have that $pV = nRT$ where $n = \frac{m}{M}$. Therefore, 
\[p_0 V = \frac{m}{M}RT\implies \rho = \frac{m}{V} = \frac{p_0 M}{RT}.\]
\vspace{3mm}

\noindent \textbf{(c)} We can conserve momentum of a cross-sectional area of air $S$ for a small interval of time $\text{d}t$ assuming that the air moves at a velocity $v$. Momentum is $p = mv = (\rho \times V)v$. So we can write for a small interval of time, velocity remains approximately constant and that $p = (\rho \times Sv\cdot \text{d}t)v$. We can also write that $p = Ft = S\Delta p \cdot \text{d}t$ where $\Delta p$ is the difference in pressure. Therefore, 
\[S\Delta p \cdot \text{d}t = (\rho \times Sv\cdot \text{d}t)v\implies \Delta p = \rho v^2.\]
The difference in pressure $\Delta p$ in terms of density inside the pipe of length $L$ can be written as $\Delta p = \Delta \rho gL = (\rho_0 - \rho)gL$. This means that 
\[(\rho_0 - \rho)gL = \rho v^2 = \left(\frac{p_0 M}{RT} - \rho\right)gL = \rho v^2.\]
\vspace{3mm}

\noindent \textbf{(d)} From idea 1, we can write $P \equiv \frac{\text{d}Q}{\text{d}t}$. The process is isobaric so 
\[P = \frac{\text{d}}{\text{d} t}\left(C_V n\Delta T\right) = C_V \Delta T\frac{\text{d}n}{\text{d}t}.\]
From the ideal gas law:
\[n = \frac{\rho V}{M}\implies \frac{\text{d}n}{\text{d}t} = \frac{\rho S \cdot \text{d}x}{M\cdot \text{d}t} = \frac{\rho Sv}{M}.\]
Therefore, from part (a) we can substitute to write
\[P = C_V \Delta T \frac{\rho Sv}{M} = \frac{\gamma}{\gamma - 1}R (T - T_0) \frac{\rho Sv}{M}.\]
\vspace{3mm}

\noindent \textbf{(e)} From part (b), we know that $\rho = \frac{p_0 M}{RT}$ which means that 
\[(\rho_0 - \rho) = \frac{p_0 M}{R} \left(\frac{1}{T_0} - \frac{1}{T}\right) = \frac{p_0 M}{R} \left(\frac{T - T_0}{T_0 T}\right)\implies \frac{\Delta \rho}{\rho} = \frac{\Delta T}{T_0}.\]
From part (c), we write 
\[\frac{\Delta \rho}{\rho} = \frac{v^2}{gL}\implies \Delta T = \frac{v^2}{gL}T_0\]
which means that when substituting into our equation in (d), we have that 
\[P = \frac{\gamma}{\gamma - 1}R \frac{v^3}{gL}T_0\frac{\rho S}{M}.\]
Remember that from the ideal gas law:
\[p_0 V = \frac{m}{M}RT\implies \rho = \frac{m}{V} = \frac{p_0 M}{RT}\]
wich means that 
\[v^3 = \frac{\gamma}{\gamma - 1} \frac{gL}{S}\frac{P}{p_0}.\]
Remember that 
\[\Delta T = \frac{v^2}{gL}T_0\implies T = T_0 \left(1 + \frac{v^2}{gL}\right) = T_0 \left[1 + \frac{1}{gL}\left( \frac{\gamma}{\gamma - 1} \frac{gL}{S}\frac{P}{p_0}\right)^{2/3}\right].\]

\end{solution}
\begin{solution}{normal}
We use the fact that effective gravity is given as $g_{\text{eff}} = g\cos\alpha$. This directly means that 
\[T = 2\pi\sqrt{\frac{R}{g\cos\alpha}}.\]
The particle will exit at B if the time to cross the trough along its axis is an integer multiple of the oscillation’s half-period. Thus, the length will be given as \footnote{There is a factor of one half added to the statement because not all the particles exit at the bottom of the gutter}
\[L = \left(n + \frac{1}{2}\right)\frac{T}{2}.\]
Thus, 
\begin{align*}
    L &= \frac{1}{2}g\sin\alpha\left(\left(n + \frac{1}{2}\right)\frac{T}{2}\right)^2\\
    &= \frac{1}{2}g\sin\alpha\left(n + \frac{1}{2}\right)^2\frac{\pi^2 r}{g\cos\alpha}\\
    &= \boxed{\frac{\pi^2}{2}\tan\alpha\left(n + \frac{1}{2}\right)^2}
\end{align*}
\end{solution}

\begin{solution}{easy}
\textbf{i)} Note that the total time is $\dfrac{2a}{v}$, so the cars can each only travel along 2 segments. \vspace{3mm}

Since $v_{dist}$ is never positive, the two cars are always approaching each other (aside from a brief instant at $t=\dfrac{a}{v}$). \vspace{3mm}

From this, we note that both cars must end up at city $O$. \vspace{3mm}

If the two cars started from cities $A$ and $B$, then their initial $v_{dist}$ would have been $0$. \vspace{3mm}

If the two cars started from cities $B$ and $C$, then their initial $v_{dist}$ would have been $v_0\sqrt{2}$. \vspace{3mm}

This leaves only the option that $\boxed{\text{the two cars started from }A\text{ and }C\text{ and both ended at }O}$. \vspace{3mm}

\textbf{ii)} Since the area under a velocity graph is just distance, the area under this velocity graph is the difference between the distance between the two cars at time $t=0$ and time $t=\dfrac{a}{v}$. \vspace{3mm}

Thus, our answer is $$2a-\sqrt{2}a=\boxed{(2-\sqrt{2})a}$$

\textbf{iii)}
$A-B:$\vspace{3mm}

For the first segment, the cars have the same velocity, so $v_{dist}=0$. \vspace{3mm}

For the second segment, the cars face each other, so $v_{dist}=-2v$.
\begin{center}
    \begin{asy}
        unitsize(3cm);
        import graph;
        real v = 0;
        pair f (real t){
        	return (t,-v);
        }
        draw((0,-1.1)--(0,1.3), arrow=Arrow(TeXHead));
        draw((0,0)--(2.25,0), arrow=Arrow(TeXHead));
        label("$v_{dist}$",(-0.25,1.1));
        draw(graph(f,0,1),red);
        v = 0.7;
        draw(graph(f,1,2),red);
        draw((1,0)--(1,-v),red);
        draw((0,-v)--(1,-v),dotted);
        label("$-2v$",(-0.3,-0.7));
        label("$t$",(2.4,0));
        label("$\frac{a}{v}$",(1,0.2));
        label("$\frac{2a}{v}$",(2,0.2));
        draw((2,0.05)--(2,-0.05));
        draw((1,0.05)--(1,-0.05));
    \end{asy}
\end{center}
$B-C:$\vspace{3mm}

For the entire course of the motion, the velocity vectors of the two cars are perpendicular to each other and both cars approach each other, so $$v_{dist}=-\sqrt{2}v$$
\begin{center}
    \begin{asy}
        unitsize(3cm);
        import graph;
        real v = 0.7;
        pair f (real t){
        	return (t,-v);
        }
        draw((0,-1.1)--(0,1.3), arrow=Arrow(TeXHead));
        draw((0,0)--(2.25,0), arrow=Arrow(TeXHead));
        label("$v_{dist}$",(-0.25,1.1));
        draw(graph(f,0,2),red);
        label("$-\sqrt{2}v$",(-0.35,-0.7));
        label("$t$",(2.4,0));
        label("$\frac{a}{v}$",(1,0.2));
        label("$\frac{2a}{v}$",(2,0.2));
        draw((2,0.05)--(2,-0.05));
        draw((1,0.05)--(1,-0.05));
    \end{asy}
\end{center}

\textbf{iv)}
$B-C:$\vspace{3mm}

As they turn, the cars face each other and then turn to perpendicular again, so $v_{dist}$ goes from $-\sqrt{2}v$ to $-2v$ and back to $-\sqrt{2}v$.
\begin{center}
    \begin{asy}
        unitsize(3cm);
        import graph;
        real v = sqrt(2);
        pair f (real t) {
        	return (t,-v);
        }
        draw((0,-2.5)--(0,0.5), arrow=Arrow(TeXHead));
        draw((0,0)--(2.25,0), arrow=Arrow(TeXHead));
        label("$v_{dist}$",(-0.25,0.2));
        draw(graph(f,0,0.9),red);
        draw(graph(f,1.1,2),red);
        label("$-\sqrt{2}v$",(-0.35,-sqrt(2)));
        label("$-2v$",(-0.35,-2));
        label("$t$",(2.4,0));
        label("$\frac{a}{v}$",(1,0.2));
        label("$\frac{2a}{v}$",(2,0.2));
        draw((2,0.05)--(2,-0.05));
        draw((1,0.05)--(1,-0.05));
        pair g (real t) {
        	return (t,-2.7183^(-(t-1)^2/(2*0.025^2))*1/1.7-sqrt(2));
        }
        draw(graph(g,0.9,1.1),red);
        draw((-0.05,-2)--(0.05,-2));
        draw((-0.05,-sqrt(2))--(0.05,-sqrt(2)));
    \end{asy}
\end{center}
\end{solution}
\newpage
\begin{solution}{normal}
\textbf{(a)} Since the process is adiabatic, we have:
$$T_0^\gamma P_0^{1-\gamma}=T_1^\gamma P_1^{1-\gamma} \implies T_1 = \left(\frac{P_{0}}{P_{1}}\right)^{\frac{1-\gamma}{\gamma}} T_{0}$$Solving, we get $T_1=279.2 \text{ K}$.
\vspace{2mm}

\textbf{(b)} We know from the ideal gas law that:
$$\Delta \rho = \frac{M}{R}\left(\frac{P_1}{T_1}-\frac{P_0}{T_0}\right)$$Since the change is linear, we have $\rho_1=\rho_0-\alpha h \implies \Delta \rho = -\alpha h$. The change in pressure is given by:
$$-\Delta P /g = \int_0^h (\rho_0-\alpha h) dh = \rho_0 h - \frac{1}{2}\alpha h^2 \implies -\alpha h = -\frac{2\Delta P}{gh}-2\rho_0$$Therefore, we have:
$$-\frac{2\Delta P}{gh}-2\rho = \frac{M}{R}\left(\frac{P_1}{T_1}-\frac{P_0}{T_0}\right)$$Solving this, we get $h=1440 \text{ m}$. Alternatively, we can write down the first line as:
$$\frac{P}{\rho T}=\text{constant}$$and go from there.
\vspace{2mm}

\textbf{(c)} If there was no rain, the temperature would be given by:
$$T_2'=\left(\frac{P_0}{P_2}\right)^\frac{1-\gamma}{\gamma}T_0=264.4 \text{ K}.$$However, there is also condensation, which is an exothermic process and the condensation of the water vapor releases heat into the atmosphere. This is given by:
$$\frac{m_\text{air}}{M}c_V\Delta T = m_\text{water}L_V\implies \Delta T = \frac{m_\text{water}}{m_\text{air}}\frac{ML_V}{R\left(\frac{\gamma}{\gamma-1}\right)}=6.1 \text{ K}$$Therefore,
$$T_2=T_2'+\Delta T=270.7 \text{ K}.$$
\textbf{(d)} Given a certain unit area, $2000 \text{ kg}$ of moist air can travel up the mountain ridge in $1500 \text{ s}$, so the rate at which water condenses is:
$$r=\frac{2000\cdot \frac{2.45}{1000}}{1500}=0.00327 \,\mathrm{ kg/(s\cdot m^2)}$$The height of the water column after $t=3 \text{ hours}$ is hus:
$$\frac{rt}{\rho}=3.5 \text{ cm}$$
\textbf{(e)} We use the same relationship:
$$T_3=\left(\frac{P_3}{P_2}\right)^\frac{1-\gamma}{\gamma}T_2=300 \text{ K}$$
\end{solution}
\begin{solution}{hard}
Let the incoming mass flow rate be labelled $\mu_i$ and it is divided into the flow rates $\mu_1$ and $mu_2$. By the equation of continuity, we have$$\mu_i = \mu_1 + \mu_2$$By Bernoulli's law (or by idea 71 and fact 30), the velocity of both the left and the right compartment is same and equal to $v$. Now, by conservation of momentum in the horizontal direction (idea 72), we get$$\mu_i \rho (v\cos{\alpha}) = \mu_1 \rho (v) - \mu_2 \rho (v)$$which simplifies to$$\mu_i \cos{\alpha} = mu_1 - \mu_2$$From this equation and our initial equation of continuity, we have$$ \mu_1 = \mu_i \cos^{2}{\frac{\alpha}{2}}$$and$$\mu_2 = \mu_i \sin^{2}{\frac{\alpha}{2}}$$Hence,$$\frac{\mu_1}{\mu_2} = \frac{\mu_i \cos^{2}{\frac{\alpha}{2}}}{\mu_i \sin^{2}{\frac{\alpha}{2}}} = \boxed{\cot^{2}{\frac{\alpha}{2}}}$$
\end{solution}

\begin{solution}{hard}
\textbf{a)} We notice that there is no image of the orange pulse, hence it must have taken place immediately before the shutter release. So the blue pulse is first, then red, then green, and finaly yellowl. As $4$ pulses are recorded exposure time must be between $300\;\mathrm{ms}$ and $500\;\mathrm{ms}$.
\vspace{5mm}

\textbf{b)} In the frame of disk’s centre, the displacement vector $\vec d$ between neighbouring flashes has always the same modulus 
\[d = 2R\sin(\omega \tau/2),\]
 and neighbouring displacement vectors are always rotated by the same angle $\omega\tau$. In the lab frame, additional constant displacement vector $\vec v \tau$ is to be added due to the translational motion of the frame: 
\[\vec d’ = \vec d + \vec v \tau.\] 
Because of that, if we bring all the displacement vectors to such positions that their starting points coincide, the endpoints will lie on a circle of radius $d$. So, we redraw the displacement vectors $\vec {br}$, $\vec {rg}$, and $\vec {gy}$ draw the circumcircle of the triangle formed by the endpoints of the vectors
\begin{center}
    \begin{asy}
    size(8cm);
draw(circle((0,0), 1));
dot("O", (0,0), S);
dot("B", (-sqrt(3)/2, -1/2), SW);
dot("A", (-sqrt(3)/2 + 0.1, -1/2 +0.2), NW);
draw((-sqrt(3)/2, -1/2) --  (-sqrt(3)/2 + 0.1, -1/2 +0.2));
dot("C", (1/2, sqrt(3)/2), NE);
dot("D", (1/2, -sqrt(3)/2), SE);
draw((-sqrt(3)/2 + 0.1, -1/2 +0.2) -- (1/2, sqrt(3)/2));
draw((-sqrt(3)/2 + 0.1, -1/2 +0.2) -- (1/2, -sqrt(3)/2));
draw((-sqrt(3)/2 + 0.1, -1/2 +0.2) -- (0,0), dashed);
draw((-sqrt(3)/2, -1/2) -- (1/2, -sqrt(3)/2), dashed);
draw((1/2, sqrt(3)/2) -- (0,0));
draw((1/2, -sqrt(3)/2) -- (0,0));
draw((-sqrt(3)/2, -1/2) -- (0,0));
    \end{asy}
\end{center}
From the figure we measure rotation angle 
\[\omega\tau = 2.2\;\mathrm{rad} \Rightarrow \omega = 22\;\mathrm{rad/s}.\] 
The constant displacement 
\[a = v\tau = 6.55\;\mathrm{cm} \Rightarrow v = 65.5\;\mathrm{cm/s}\]
and the circle’s radius 
\[d = 2R\sin(\omega\tau/2) = 8.27\;\mathrm{cm} \Rightarrow R = \boxed{4.6\;\mathrm{cm}}.\]
\end{solution}
\begin{solution}{easy}
Let the distance between the minor gridlines provided be $d$.\vspace{3mm}

We note that $d$ is also (approximately) the diameter of the water jet.\vspace{3mm}

We can use the points $(0,0)$, $(20d,-5.5d)$, and $(30d,-12.5d)$ to find the equation of the water trajectory, which we determine to be $$y=-\dfrac{0.014}{d}x^2$$

The trajectory of the water is also given by the parametric equation $x=vt,\;y=-\dfrac{g}{2}t^2$, which gives $$y=-\dfrac{g}{2v^2}x^2$$

Thus, we have $$\dfrac{0.014}{d}=\dfrac{g}{2v^2}\implies v=\sqrt{\dfrac{gd}{0.028}}$$

Since the pipe outflow rate must be the same as the bucket inflow rate,

$$\dfrac{\pi d^2v}{4}=\dfrac{\pi d^2\sqrt{\dfrac{gd}{0.028}}}{4}=\dfrac{V}{t}$$

This equation gives $$d=\left(\dfrac{4V}{\pi t}\sqrt{\dfrac{0.028}{g}}\right)^{2/5}=\boxed{1.03\;\text{mm}}$$
\end{solution}
\begin{solution}{normal}
\textbf{a)} Let the normal force from the floor on the ladder be $N$. Then, at the cutoff case, the friction force takes on it's maximum, so the friction from the floor is $\mu N$.

\begin{center}
\begin{asy}
size(4cm);
defaultpen(fontsize(10pt));

real wall_th = 0.1, wall_h = 1.1, wall_w = 0.7;
fill((-wall_th, -wall_th)--(-wall_th, wall_h)--(0, wall_h)--(0, 0)--(wall_w, 0)--(wall_w, -wall_th)--cycle, gray(0.8));
draw((0, wall_h)--(0, 0)--(wall_w, 0));

real ladder_w = 0.5, ladder_h = 0.8;
draw((0, ladder_h)--(ladder_w, 0));

draw("$\theta$", arc((ladder_w, 0), 0.08, 180 - aTan(ladder_h/ladder_w), 180));

real N = 1/5, mu = 0.8, eps = 0.02;
draw(Label("$N_2$", Relative(1), dir(0)), shift(ladder_w, 0)*((0, 0)--(0, N)), rgb(0, 0.4, 0), Arrow(TeXHead));
draw(Label("$\mu N$", Relative(1), dir(-90)), shift(ladder_w, -eps)*((0, 0)--(-mu*N, 0)), rgb(0, 0.4, 0), Arrow(TeXHead));
draw(Label("$N_1$", Relative(1), dir(90)), shift(0, ladder_h)*((0, 0)--(mu*N, 0)), rgb(0, 0.4, 0), Arrow(TeXHead));

real mg = N+mu^2*N;
draw(Label("$mg$", Relative(1), dir(190)), shift(ladder_w/2, ladder_h/2)*((0, 0)--(0, -mg)), rgb(0, 0.3, 0.6), Arrow(TeXHead));
\end{asy}
\end{center}

Since the ladder is in equilibrium, we have three equations. These is the equation of equilibrium of force in the horizontal and vertical direction and as well as torques. Looking quickly at the vertical forces, we can see easily that $N_2=mg$. Then by looking at the horizontal forces, we see that $N_1=\mu N$. Therefore, there is only one equation of torque remaining.
\vspace{3mm}

\noindent We first have to find the pivot point of the ladder. Generally, the pivot point of the system is located where there are more forces. Thus, by looking at the ladder, we see that the pivot point of the system is the bottom of the ladder. Balancing the torques due to gravity and $N_1$, we have
$$N_1\ell\sin\theta=mg(\ell/2)\cos\theta\implies N_1=\frac{mg}{2\tan\theta}$$This is also the value of the frictional force $F$ as we have found before. Thus, by using $F\geq\mu{mg}$ we find
$$\frac{mg}{2\tan\theta}\leq\mu{mg}\implies \boxed{\tan\theta\geq\frac{1}{2\mu}}$$

\textbf{b)} Drawing a freebody diagram gives us the following diagram
\begin{center}
\begin{asy}
size(4cm);
defaultpen(fontsize(10pt));

real wall_th = 0.1, wall_h = 1.1, wall_w = 0.7;
fill((-wall_th, -wall_th)--(-wall_th, wall_h)--(0, wall_h)--(0, 0)--(wall_w, 0)--(wall_w, -wall_th)--cycle, gray(0.8));
draw((0, wall_h)--(0, 0)--(wall_w, 0));

real ladder_w = 0.5, ladder_h = 0.8;
draw((0, ladder_h)--(ladder_w, 0));

draw("$\theta$", arc((ladder_w, 0), 0.08, 180 - aTan(ladder_h/ladder_w), 180));

real N = 1/5, mu = 0.8, eps = 0.02;
draw(Label("$N$", Relative(1), dir(0)), shift(ladder_w, 0)*((0, 0)--(0, N)), rgb(0, 0.4, 0), Arrow(TeXHead));
draw(Label("$\mu N$", Relative(1), dir(90)), shift(0, ladder_h)*((0, 0)--(mu*N, 0)), rgb(0, 0.4, 0), Arrow(TeXHead));
draw(Label("$\mu N$", Relative(1), dir(180)), shift(-eps, ladder_h)*((0, 0)--(0, mu^2*N)), rgb(0, 0.4, 0), Arrow(TeXHead));

real mg = N+mu^2*N;
draw(Label("$mg$", Relative(1), dir(190)), shift(ladder_w/2, ladder_h/2)*((0, 0)--(0, -mg)), rgb(0, 0.3, 0.6), Arrow(TeXHead));
\end{asy}
\end{center}
We see that from force balance that there is not an opposite force to oppose the force of $\mu N$ from the wall. This means that it is impossible for the ladder to stay still in this case.
\tcbline
There is an easier way to solve part a. Let us project the gravitational force vector $mg$ and the normal force from the wall $N_1$ such that they meet at a point above the middle of the ladder. At this location, the torque caused by these two forces is zero. In order to be in static equilibrium, the force from the ground must also intersect this point. The slope the force from the ground makes with the horizontal is $2\tan\theta$.

\begin{center}
\begin{asy}
size(4cm);
defaultpen(fontsize(10pt));

real wall_th = 0.1, wall_h = 1.1, wall_w = 0.7;
fill((-wall_th, -wall_th)--(-wall_th, wall_h)--(0, wall_h)--(0, 0)--(wall_w, 0)--(wall_w, -wall_th)--cycle, gray(0.8));
draw((0, wall_h)--(0, 0)--(wall_w, 0));

real ladder_w = 0.5, ladder_h = 0.8;
draw((0, ladder_h)--(ladder_w, 0));

draw("$\theta$", arc((ladder_w, 0), 0.08, 180 - aTan(ladder_h/ladder_w), 180));

real N = 1/5, mu = 0.8, eps = 0.02;
draw(Label("$N_1$", Relative(1), dir(90)), shift(0, ladder_h)*((0, 0)--(mu*N, 0)), rgb(0, 0.4, 0), Arrow(TeXHead));
draw(shift(0, ladder_h)*((mu*N, 0)--(ladder_w/2, 0)), dashed);

draw(Label("$f_\text{ground}$", Relative(1), dir(0)), shift(ladder_w, -eps)*((0, 0)--(-ladder_w/4, ladder_h/2)), rgb(0, 0.4, 0), Arrow(TeXHead));
draw(shift(ladder_w, -eps)*((-ladder_w/4, ladder_h/2)--(-ladder_w/2, ladder_h)), dashed);

real mg = N+mu^2*N;
draw(Label("$mg$", Relative(1), dir(190)), shift(ladder_w/2, ladder_h/2)*((0, 0)--(0, -mg)), rgb(0, 0.3, 0.6), Arrow(TeXHead));
draw(shift(ladder_w/2, ladder_h/2)*((0, ladder_h/2)--(0,0)),dashed);
\end{asy}
\end{center}
Since the force from the ground is consisted of both the normal force $N_2$ and the friction force $f_s$, we have:
$$2\tan\theta=\frac{N_2}{f_s}$$Combining this with $f_s \le \mu N_2$ gives:
$$\boxed{\tan\theta \ge \frac{1}{2\mu}}$$
\end{solution}

\begin{solution}{normal}
The smoke rises until it's density becomes the same as the density of the air around it. From problem 69 and 70, note that 
\[\frac{\text{d}T}{T} = (1 - \gamma^{-1}) \frac{\text{d}p}{p}\implies \Delta T = T (1 - \gamma^{-1}) \frac{\Delta p}{p}.\]
The change in pressure is given by $\Delta p = \rho gh$ and since $\rho = p\mu/RT$ from the ideal gas law, we get 
\[\Delta p = \frac{p \mu}{RT} gh\]
also note that 
\[\gamma^{-1} = \frac{c_V}{c_p} = \frac{c_V}{c_V + R}\]
which means that 
\[\Delta T = \frac{R}{c_V + R}\frac{\mu gh}{R}.\]
Rearranging to solve for $h$ tells us that 
\[h = \frac{\Delta T R (c_V + R)}{\mu g}.\]
\end{solution}
\begin{solution}{normal}
\textbf{a)} Consider a rectangular prism of length $l$, width $w$, and height $h$. \vspace{3mm}

Assume that the volume of rain that the man receives per second is proportional to $Av_r$ by some proportionality factor $k$, where $A$ is the cross sectional area of where the rain strikes and where $v_r$ is the velocity of the rain.\vspace{3mm}

Let $V$ be the critical volume of rain needed for the man to "get wet".\vspace{3mm}

When the man is not moving, we find that $$V=Akv_rt_1=lwkv_rt_1$$

When the man is moving at speed $v_m$, in his frame of reference, the rain falls on him at an angle $\theta=\arctan\left(\dfrac{v_m}{v_r}\right)$ to the vertical at a speed of $\sqrt{v_r^2+v_m^2}$, as shown in the following diagram:
\begin{center}
    \begin{asy}
        unitsize(3cm);
        real x = 20*pi/180;
        draw((0,0)--(0,2)--(1,2)--(1,0)--(0,0));
        draw((-1,0)--(2,0));
        draw((1,0)--(1+2*cos(x)*sin(x),2*cos(x)*cos(x))--(1-cos(x)*cos(x),2+sin(x)*cos(x))--(0,2));
        draw(arc((1,0),0.4,90,90-20));
        draw(arc((1,2),0.3,180,180-20));
        label("$\theta$",(1.08,0.5));
        label("$\theta$",(0.63,2.06));
        label("$l$",(0.5,1.9));
        label("$h$",(0.92,1));
        label("$l\cos\theta$",(0.6,2.3));
        label("$h\sin\theta$",(1.5,2));
    \end{asy}
\end{center}
Thus, we see that $$V=Ak\sqrt{v_r^2+v_m^2}t_2=w(l\cos\theta+h\sin\theta)k\sqrt{v_r^2+v_m^2}t_2$$

Since we have $\cos\theta=\dfrac{v_r}{\sqrt{v_r^2+v_m^2}}$, $\sin\theta=\dfrac{v_m}{\sqrt{v_r^2+v_m^2}}$, the expression is equivalent to $$V=w(lv_r+hv_m)kt_2$$

We now have $$lwkv_rt_1=w(lv_r+hv_m)kt_2 \implies v_r=\dfrac{hv_mt_2}{l(t_1-t_2)}$$

Plugging in $v_m=\dfrac{18}{3.6}\;\text{m/s}$, $t_1=120\;\text{s}$, $t_2=30\;\text{s}$, we get that $$v_r=\dfrac{5h}{3l}\;\text{m/s}$$

This gives, for $$v_m=6\;\text{km/h}, t=\dfrac{lv_rt_1}{lv_r+hv_m}=\boxed{60\;\text{s}}$$

\textbf{b)} Consider a sphere of radius $R$. \vspace{3mm}

Assume that the volume of rain that the man receives per second is proportional to $Av_r$ by some proportionality factor $k$, where $A$ is the cross sectional area of where the rain strikes and where $v_r$ is the velocity of the rain. \vspace{3mm}

Let $V$ be the critical volume of rain needed for the man to "get wet". \vspace{3mm}

When the man is not moving, we find that $$V=Akv_rt_1=\pi R^2kv_rt_1$$

When the man is moving at speed $v_m$, in his frame of reference, the rain falls on him at an angle $\theta=\arctan\left(\dfrac{v_m}{v_r}\right)$ to the vertical at a speed of $\sqrt{v_r^2+v_m^2}$, as shown in the following diagram:
\begin{center}
    \begin{asy}
        unitsize(3cm);
        draw(circle((0,0),1));
        real x = -20*pi/180;
        draw((cos(x),sin(x))--(cos(x)-sin(x),sin(x)+cos(x))--(cos(pi+x)-sin(x),sin(pi+x)+cos(x))--(cos(pi+x),sin(pi+x)));
        draw((cos(x),sin(x))--(0,0)--(1,0));
        draw((-2,-1)--(2,-1));
        draw(arc((0,0),0.3,0,-20));
        label("$\theta$",(0.4,-0.07));
        draw((cos(x)-sin(x),sin(x)+cos(x))--(cos(x)-sin(x),-1),dotted);
        draw(arc((cos(x)-sin(x),sin(x)+cos(x)),0.4,270,250));
        label("$\theta$",(1.2,0.1));
        label("$R$",(0.5,0.1));
        label("$2R$",(0.4,1.05));
    \end{asy}
\end{center}
Thus, we see that $$V=Ak\sqrt{v_r^2+v_m^2}t_2=\pi R^2k\sqrt{v_r^2+v_m^2}t_2$$

We now have $$\pi R^2kv_rt_1=\pi R^2k\sqrt{v_r^2+v_m^2}t_2\implies v_rt_1=\sqrt{v_r^2+v_m^2}t_2$$

Solving the system of equations with $v_m=\dfrac{18}{3.6}\;\text{m/s}$, $t_1=120\;\text{s}$, $t_2=30\;\text{s}$, we get $$v_r=\dfrac{v_mt_2}{\sqrt{t_1^2-t_2^2}}=\dfrac{\sqrt{15}}{3}\approx1.29\;\text{m/s}$$

This gives, for $v_m=6\;\text{km/h}$, $$t=30\sqrt{6}\approx\boxed{73.5\;\text{s}}$$
\end{solution}
\begin{solution}{normal}
Two forces act on the rod in the vertical direction, it's weight and the force of friction, where at its maximum is $\mu N_1$. As the weight increases, we must have $N_1$ increase as well. Let us look at the limiting case where $W_\text{rod} \to \infty$. The normal and friction forces acting on the cylinder will be so large that the mass of the cylinder will be negligible, thus we can ignore the force $mg$. This allows us to effectively turn gravity off.
\vspace{2mm}

Let us now rotate the setup by an angle $\alpha/2$ such that it is completely symmetrical  along its vertical axis. It is clear that the horizontal forces will cancel each other out.
\begin{center}
    \begin{asy}
    size(5cm);
draw(circle((0,0),3));
draw((-2.62,-1.47)--(-0.59,-0.33),EndArrow);
draw((-2.62,-1.47)--(-0.59,-5.07),EndArrow);

draw((2.62,-1.47)--(0.59,-0.33),EndArrow);
draw((2.62,-1.47)--(0.59,-5.07),EndArrow);
    \end{asy}
\end{center}

Now we just have to balance out vertical forces. Due to symmetry, the y-component of each friction force cancels out with the y-component of each normal force. For the left side, we have:
$$N_1\sin(\alpha/2)=\mu_1 N_1\cos(\alpha/2) \implies \boxed{\mu_1 > \tan(\alpha/2)}$$
We have the inequality since the force balance equation gives the maximum friction. Similarly for the other side:
$$\boxed{\mu_1 > \tan(\alpha/2)}$$
\end{solution}

\begin{solution}{normal}\textbf{(a)} In analogy to electrical circuits, the resistance of a single wire is given by:
$$R = \frac{L}{\kappa S}$$
Since we have four wires in parallel, the effective resistance is:
$$R = \frac{L}{4\kappa S}$$
\vspace{3mm}

\noindent \textbf{(b)} For a small period of time $\text{d}t$, the temperature is changed by $\Delta T$ degrees. This means that 
\[P \equiv \frac{\text{d}Q}{\text{d}t} = C\Delta\dot{T}.\]
Let us look at the electrical analogy between heat conduction and electric circuits. In this case, $\dot{Q}$ acts as the current while the temperature difference $\Delta T \equiv T - T_0$ acts as the voltage difference. From this perspective, we can also define 
\[\dot{Q} = \frac{\Delta T}{R}.\]
This means that from adding these two individual expressions (as there is an extra $\dot{Q}$ due to our "voltage difference") we have that 
\[P\cos \omega t = C\Delta\dot{T} + \frac{\Delta T}{R}.\]
As power varies with time, we attempt to seek the solution in the form of 
\[T = T_0 + \Delta T \sin (\omega t + \phi).\]
Taking the derivative of our sought solution and substituting into our differential equation gives us 
\[P\cos\omega t = C\Delta T\omega \cos (\omega t + \phi) + \frac{\Delta T}{R}\cos \left(\omega t + \phi - \frac{\pi}{2}\right)\]
which can be seen as the sum of multiple rotating vectors or:
\[\vec{P} = \vec{P}_C + \vec{P}_R.\]
Each of these vectors have an individual magnitude of 
\[P_C = C\Delta T \omega, \quad P_R = \frac{\Delta T}{R}\]
\begin{center}
    \includegraphics[width=6cm]{phasor.png}
\end{center}
Therefore, from pythagorean theorem, the amplitude of oscillations is 
\[P_0^2 = (C\omega\Delta T)^2 + \left(\frac{\Delta T}{R}\right)^2\implies \Delta T = \frac{P_0}{\sqrt{C^2\omega^2 + R^{-2}}}.\]
Our solution is in the form of 
\[T = T_0 + \Delta T \sin (\omega t + \phi) = T_0 +  \frac{P_0\cos (\omega t + \phi)}{\sqrt{C^2\omega^2 + R^{-2}}}.\]
Let us estimate the phase difference. We can go back to our phase diagram and see that 
\[\phi = \arcsin\left(\frac{P_C}{P_0}\right) = \arcsin\left(\frac{C\omega}{\sqrt{C^2 \omega^2 + R^{-2}}}\right).\]
This means that our final solution is 
\[T = T_0 +  \frac{P_0\cos \left(\omega t + \arcsin\left(C\omega/\sqrt{C^2 \omega^2 + R^{-2}}\right)\right)}{\sqrt{C^2\omega^2 + R^{-2}}} .\]
\vspace{3mm}

\noindent \textbf{(c)} Remember from part b that our amplitude of oscillations of $\Delta T$ is given by $\Delta T = \frac{P_0}{\sqrt{C^2\omega^2 + R^{-2}}}$. We want there to be as large a change as possible with a small change in $C$. This is equivalent to maximizing the derivative ie setting the double derivative to $0$. Upon taking two derivatives, we get the function
\[\frac{3P_0 C^2 \omega^4}{(C^2 \omega^2 + R^{-2})^{5/2}} - \frac{P_0 \omega^2}{(C^2 \omega^2 + R^{-2})^{3/2}} = 0.\]
Seperating and solving for $\omega$ yields $\omega = 1/\sqrt{2}CR$.
\vspace{3mm}

\noindent \textbf{(d)} As the questions asks to estimate, we ignore all numerical prefactors. From part c), we know that $C \approx 1/\omega R$. This should be of the same order as the heat capacity of the bridges. Therefore, we have 
\[\frac{1}{l/\kappa S\cdot \omega} = \rho lSc\implies \omega_c \approx \frac{\kappa}{c\rho L^2}.\]
\end{solution}
\begin{solution}{normal}
Similar to problem 16, we want:
$$\rho gh(\pi R^2) = mg + V\rho g$$
except this time:
\begin{align*}
V &= \frac{2}{3}\pi R^3 - \pi H^2\left(R-\frac{H}{3}\right) \\
&= \frac{2}{3}\pi R^3 - \pi (R-h)^2\left(\frac{2R+h}{3}\right) \\
&= \frac{2}{3}\pi R^3 - \frac{\pi}{3} (2R^3+R^2h-4R^2h-\cancel{2Rh^2}+\cancel{2Rh^2}+h^3) \\
&= \pi R^2h - \frac{\pi}{3}h^3
\end{align*}
Plugging this in gives:
$$\cancel{\rho gh(\pi R^2)} = mg +  \cancel{\pi R^2h\rho g} - \frac{\pi}{3}h^3\rho g$$
or:
$$mg = \frac{\pi}{3}h^3\rho g \implies \boxed{h = \sqrt[3]{\frac{3m}{\pi \rho}}}$$
Verifying, if we plug $m=\frac{\pi}{3}\rho R^3$, we do indeed get $h=R$.
\end{solution}

\begin{solution}{hard}\textbf{a)} After each pumping cycle, the amount of gas inside, reduces by a factor of $1-\alpha$. This means that after $N$ pumping cycles, the pressure inside the bulb decreases by a factor of $\beta = (1-\alpha)^N$. When $N\to\infty$, the factor $\beta$ can be approximated by $\beta = e^{-N\alpha}$. This means that 
\[\beta = e^{-N\alpha}\implies N = \boxed{\frac{1}{\alpha}\ln\beta}.\]
\vspace{3mm}

\noindent\textbf{b)} We know that work is generally defined by 
\[W = -pV.\]
However, since the volume in the cylinder is given by $\alpha V$, then after $N$ cycles, the work is defined by 
\[W = -N\alpha p_0 V\implies W = \boxed{p_0 V \ln\beta}.\]
\vspace{3mm}

\noindent \textbf{c)} We know that the adiabatic law gives us the proportion $pV^\gamma\propto T^\gamma$ and the ideal gas law tells us $pV\propto T\implies (pV)^\gamma \propto T^\gamma.$ Substituting these in, gives us $p^{\gamma - 1} \propto T^\gamma$. When the pressure in the bulb becomes $\beta p_0$, the pressure in the cylinder, will in proportion, be increased by a factor of $1/\beta$. Therefore, 
\[T = \boxed{T_0\beta^{1/\gamma -1}}.\]
\vspace{3mm}

\noindent \textbf{d)} The trick to this part is that the only time energy is lost is in the heating up of the released air since everywhere else the work is "reused". There is two places work is done. Actually creating the vacuum and then heating the air. To actually create the vacuum takes $PV$ energy and to heat up the air takes the sum of all the temperature heating. We calculate the second part by sum of 
\[C_v \cdot \beta \cdot PV \alpha \frac{\beta^{(1/\gamma-1)}-1}{1+\alpha}.\]
Since $\beta = 1/(1+\alpha)^{(n/\gamma)}$ where $n$ is the pump number we just use a geometric series to sum these up and replace $C_v = 1/\gamma - 1$. Thus, multiplying out everything cancels and you just end up with another $PV$ so the answer is $2PV$.
\end{solution}
\begin{solution}{normal}
First, let us move into an accelerated reference frame such that $M$ is stationary. The acceleration of $M$ is:
$$Ma=T-T\sin\alpha \implies a_M = T\left(\frac{1-\sin\alpha}{M}\right)$$
Thus, $m$ will have a fictitious force acting towards the right. The actual gravitational force and the fictitious force combine together to give us the effective gravity.

Now keep in mind that even in this accelerated reference frame, $m$ is not stationary. It is actually moving in the direction parallel to the rope holding it and due to conservation of rope, the acceleration of $m$ in this frame is $a_m=a_M$. Balancing forces, we get:
$$mg_\text{eff}-T=ma_M$$
Substituting in $g_\text{eff}=\frac{mg}{\cos\alpha}$ and $a_M$, we get:
$$\frac{mg}{\cos\alpha}-T=T\left(\frac{m}{M}\right)(1-\sin\alpha)$$
We can solve for $T$ to be:
$$T = \frac{mg}{\cos\alpha} \cdot \frac{1}{1+(m/M)(1-\sin\alpha)}$$
The ratio of the fictitious force and the gravitational force form a right angle, such that:
$$\tan\alpha = \frac{a_M}{g} = \frac{T}{mg}(1-\sin\alpha)$$
Substituting in $T$, cancelling out $\cos\alpha$ on both sides, and solving for $m/M$ (the algebra takes some time), we get:
$$\boxed{\frac{m}{M}}=\frac{\sin\alpha}{(1-\sin\alpha)^2}$$
\end{solution}

\begin{solution}{normal}
Let the normal force on the cylinder be $N_1$, the normal force on the wedge be $N_2$, and the normal force between the cylinder and wedge be $N$. If the cylinder moves downwards with an acceleration $a_1$ and the wedge moves upwards with an acceleration $a_2$, we find from geometry that we have a constraint equation of 
\begin{align*}
Na_1\cos (180 - \alpha) + Na_2 \cos\alpha = 0\\
-a_1 + a_2 = 0 \implies a_1 = a_2
\end{align*}
To find the acceleration, we use lagrangian formalism. Let the cylinder move down by a small amount $\xi$, the wedge will then move upwards $\xi$ meaning that the change in potential energy will become
\[
\Pi(\xi) = (m - M)g\sin\alpha \xi
\]Differentiating this with respect to $\xi$ gives us
\[
\Pi'(\xi) = (m - M)g\sin\alpha.
\]The kinetic energy of the system in this case will then be given by
\[K = \frac{1}{2}(m + M)\dot\xi^2 
\]which implies that $\mathcal{M} = m + M.$ This lets the acceleration become
\[
a = \frac{m - M}{m + M}g\sin\alpha.
\] Projecting Newton's laws onto the cylinder gives us the equation
\[mg\sin\alpha - N\cos\alpha = ma\]
Substituting our value of accceleration gives us 
\[mg\sin\alpha - N\cos\alpha = m\frac{m - M}{m + M}g\sin\alpha\]
rearranging variables gives us 
\begin{align*}
mg\sin\alpha\left(1 - \frac{m -M}{m + M}\right) = N\cos\alpha\\
\boxed{N = 2\frac{Mm}{m + M}g\tan\alpha}
\end{align*}
\blfootnote{This problem was found in the book ’Aptitude Test Problems in Physics’ by S.S. Krotov though in that problem, only velocity was asked for.}
\end{solution}

\begin{solution}{normal}
We consider the boat in the frame of reference of the air. Since we have the wind measurements, we can find the displacement of the boat in the frame of reference of the air. We'll be taking south and east as positive. \vspace{3mm}

During the first segment we have the wind blowing east, so in the frame of reference of the air, the boat is displaced east by $$x_1 = 60v_1t_1$$. \vspace{3mm}

The 60 is there for a unit conversion of the time to seconds. \vspace{3mm}

We do the same for the second leg, however we must account for the fact that the wind blows southeast, so there is both a southern displacement and an eastern displacement.
$$x_2 = 60v_2\cos(\pi/4)t_2 = 30\sqrt{2}v_2t_2$$
$$y_2 = 60v_2\sin(\pi/4)t_2=30\sqrt{2}v_2t_2$$

Similarly for the last leg, we have that
$$x_3 = 60v_3 \cos(3\pi/4)t_3 = -30\sqrt{2}v_3t_3$$
$$y_3 = 60v_3\cos(3\pi4)t_3 = 30\sqrt{2}v_3t_3$$

The total southern displacement is
$$y = 30\sqrt{2}v_2t_2+30\sqrt{2}v_3t_3 \approx 955 \, \mathrm{m}$$

The total eastern displacement is 
$$x = 60v_1t_1+30\sqrt{2}v_2t_2-30\sqrt{2}v_3t_3 \approx 3018 \, \mathrm{m}$$

Since the displacement in the lab frame is south 4000 m, we can find the displacement caused by the wind (in the lab frame) as $4000-y$ south and $-x$ east. From there we just divide by the total time to find the wind speed as it's constant, so
$$v_\text{wind} = \dfrac{\sqrt{(4000-y)^2 + (-x)^2}}{60(t_1 + t_2 +t_3)} = \dfrac{\sqrt{3045^2 + 3018^2}}{60} \approx \boxed{12 \, \mathrm{m/s}}$$
\end{solution}
\begin{solution}{hard}
\textbf{Solution 1.} Let the acceleration of mass $m$ along the incline be $a$ and acceleration of mass $M$ in downward direction be$\ a_1$.
Since length of string remains constant therefore we have
\begin{equation*}
    a\sin\alpha=a_1
\end{equation*}
Writing the force equations for both masses gives us 
\begin{equation*}
    T\sin\alpha+mg\sin\alpha=ma
\end{equation*}
\begin{equation*}
    mg-T=ma_1
\end{equation*}
Solving the three equations we get $\boxed{a_1=g\sin^2\alpha\frac{M+m}{m+Msin^2\alpha}}$
\tcbline 
\noindent \textbf{Solution 2.} In this problem it serves just fine to treat the angle that the mass leans back with respect to the vertical as $\varphi = 0$ since we only care about the acceleration of the mass at the [i]instant[/i] that it is released. Let the vertical generalized coordinate $\xi$ of the mass $M$ be directed vertically downwards as shown in the diagram below:
\begin{center}
    \begin{asy}
    size(8cm);
    import olympiad;
draw((0, 0) -- (2.3, 1.15));
draw((0, 0) -- (1, 0), dashed);
draw(anglemark((1, 0), (0, 0), (2, 1)));
draw((1.9, 1) -- (1.9, 0.9), linewidth(10));
draw((1.9, 0.9) -- (1.9, 0.3));
label("$M$", (1.9, 0.3), 5E);
label("$m$", (1.9, 1), 2N);
filldraw(circle((1.9, 0.3), 0.1), black);
draw((1.9, 0.2) -- (1.9, 0), arrow=Arrow(4));
label("$\xi$", (1.9, 0), S);
draw((1.9, 1) -- (1.6, 0.85), arrow=Arrow(4));
label("$\frac{\xi}{\sin\alpha}$", (1.6, 0.85), NW);
label("$\alpha$", (0.23, 0), NE);
    \end{asy}
\end{center}
Consider the kinetic energy of the system given by 
\[K = \frac{1}{2}M\dot{\xi}^2 + \frac{1}{2}m \left(\frac{\dot{\xi}}{\sin\alpha}\right)^2 = \frac{1}{2}\left(\frac{m}{\sin\alpha} + M\right)^2 \dot{\xi}^2 = \frac{1}{2}\mathcal{M}\dot{\xi}^2\]
where $\mathcal{M}$ is the effective mass given by 
\[\mathcal{M} = \frac{m}{\sin\alpha} + M.\]
Similarly, we can write the potential energy $\Pi (\xi)$ as (since both masses go down by a distance $\xi$)
\[\Pi = - (M + m)g \xi\implies \Pi^{\prime} (\xi) = -(M + m)g.\]
By method 6, we note that the acceleration $\ddot{\xi}$ can then be written as 
\[\ddot{\xi} = -\frac{\Pi' (\xi)}{\mathcal{M}} = g \frac{m + M}{m + M\sin^2\alpha}\sin^2\alpha.\]
\end{solution}

\begin{solution}{hard}
The total light emission power of the LED is $P_0 = 1 \mu W$. 
\begin{center}
    \begin{asy}
    unitsize(2cm);
    draw(circle((0,0), 0.75));
label("LED", (0,0));
draw((-2, 0) -- (-0.75, 0), arrow=Arrow(4));
label("$W$", (-1.2, 0), N);
draw((0, -2) -- (0, -0.75), arrow=Arrow(4));
label("$Q$", (0, -1.35), E);
draw((0, 0.75) -- (0, 2), arrow=Arrow(4));
label("$Q'$", (0, 1.75), E);
    \end{asy}
\end{center}
Since we have $$\frac{P_0}{A} = \int_{\nu_1}^{\nu_2}{I(\nu, T)} \mathrm{d}\nu$$ We substitute the well known expression for $I(\nu, T)$ from Planck's radiation law: 
$$ \frac{P_0}{A} = \int_{\lambda_1}^{\lambda_2}{\frac{2\pi h c^2 \mathrm{d}\lambda}{\lambda^5(e^{\frac{hc}{\lambda k T}} - 1)}} = \frac{2\pi h}{c^2} \int_{\nu_1}^{\nu_2}{\frac{\nu^3}{e^{\frac{h\nu}{kT_0}}-1} \mathrm{d}\nu}$$ We can use the well known integral $$\int_{0}^{\infty}{\frac{\eta^3\text{d}\eta}{e^\eta - 1}} = 6\xi(4) = \frac{\pi^4}{15}$$ and substitute the known values and using the problem condition. We find from evaluating the integral that $T_1 \geq 1116 K$. Now, the maximum possible efficiency of the LED is just $$\eta_{\text{max}} = \frac{T_1}{T_1 - T_0} -1 = 1+ 0.354 = \boxed{1.354}$$

\end{solution}
\begin{solution}{normal}
\textbf{Solution 1:} Note that the internal energy is given by $c_V T$ and when this goes to the rocket nozzel, part of this turns into kinetic energy $\mu v^2/2$. Since the gas is monoatomic, part of it transfers into work $pV = RT$. Therefore, we have our conservation of energy equation to be 
\[c_V T_0 + RT_0 = c_V T_1 + RT_1 + \frac{1}{2}\mu v_{\text{exit}}^2.\]
Using the fact that $c_v = \frac{5}{2}R$, we can simplify this to solve for the final exit velocity
\[\frac{5}{2}RT_0 + RT_0 = \frac{5}{2}RT_1 + RT_1 + \frac{1}{2}\mu v_{\text{exit}}^2\implies v_{\text{exit}}^2 = \sqrt{\frac{7 (T_0 - T_1)}{\mu}}.\]
The force is then given by the exit velocity or 
\[F = mv = \rho_1 A v_{\text{exit}}^2 = \rho_1 A \frac{7 (T_0 - T_1)}{\mu}.\]
We can attempt to use the approximation that $T_0 \gg T_1$ to simplify this problem:
\[F = \frac{7A\rho T_1}{\mu} \left(\frac{T_0}{T_1} - 1\right)\approx \frac{7Ap_1}{\mu}\frac{T_0}{T_1}.\]

\tcbline 

\textbf{Solution 2:} Applying Bernoulli's principle for the gas between the nozzle and the chamber exit,
$$\frac{v^2}{2}+\frac{C_p T}{M}=\textup{const.}\implies 0+\frac{C_p T_0}{M}=\frac{v_{exit}^2}{2}+\frac{C_p T_1}{M}$$
Now, the thrust applied equals
$$F=\rho S v_{exit}^2$$
and $\rho = \frac{P_1 M}{RT_1}, v_{exit}^2=\frac{2C_p (T_0-T_1)}{M}$. This means that 
$$F=\frac{2C_p(T_0-T_1)P_1 S}{RT_1}$$
Given that: $T_0 \gg T_1 \implies T_0-T_1 \approx T_0$ and $C_p =C_v+R=\frac{7R}{2}$
$$\therefore F\approx \frac{7T_0P_1 S}{T_1} $$

\end{solution}
\begin{solution}{hard}
Let us analyze a small section $dz$ in the atmosphere with density $\rho$. The relation between pressure and density is then given by 
\[\dd p = - \rho g \dd z.\]
We know that $p = n k_B T$ and $\rho = nm$ where $m$ is the mass of a single molecule and therefore, 
\[\dv{p}{z} = - \frac{mg p}{k_B T}\implies T \frac{\dd p}{p} = - \frac{mg}{k_B}dz.\]
Note that 
\[p^{1 - \gamma} T^{\gamma} = \text{const.}\]
and therefore, 
\[(1 - \gamma) \frac{\dd p}{p} + \gamma \frac{\dd T}{T} = 0.\]
Now substituting gives us 
\[\frac{\dd T}{\dd z} = -\left(\frac{\gamma - 1}{\gamma}\right) \frac{mg}{k_B} \implies \dd T =  -(1 - \gamma^{-1}) \frac{\mu g}{R}\dd z.\]
Integrating from $T_0$ to $T$ gives us 
\[\int_{T_0}^{T} \dd T = \int_{0}^{z} -(1 - \gamma^{-1}) \frac{\mu g}{R}\dd z \implies \boxed{T = T_0 - (1 - \gamma^{-1}) \frac{\mu g z}{R}}.\]

\end{solution}
\begin{solution}{normal}
\textbf{(a)} We essentially apply the same process as problem 69. Note that 
\[p^{1 - \gamma} T^{\gamma} = \text{const.}\]
and therefore, 
\[(1 - \gamma) \frac{\dd p}{p} + \gamma \frac{\dd T}{T} = 0.\]
This means that 
\[\frac{\dd T}{T} = (1 - \gamma^{-1})\frac{\dd p}{p}.\]
\vspace{3mm}

\noindent \textbf{(b)} Let us analyze a small section $dz$ in the atmosphere with density $\rho$. The relation between pressure and density is then given by 
\[dp = - \rho g dz.\]
We know that $p = n k_B T$ and $\rho = nm$ where $m$ is the mass of a single molecule and therefore, 
\[\frac{dp}{dz} = - \frac{mg p}{k_B T}\implies T \frac{dp}{p} = - \frac{mg}{k_B}dz.\]
\vspace{3mm}

\noindent \textbf{(c)} From parts a and b we can substitute to give us 
\[\frac{\dd T}{\dd z} = -\left(\frac{\gamma - 1}{\gamma}\right) \frac{mg}{k_B} \implies \dd T =  -(1 - \gamma^{-1}) \frac{\mu g}{R}\dd z.\]
Integrating from $T_0$ to $T$ gives us 
\[\int_{T_0}^{T} \dd T = \int_{0}^{z} -(1 - \gamma^{-1}) \frac{\mu g}{R}\dd z \implies T = T_0 - (1 - \gamma^{-1}) \frac{\mu g z}{R} = \boxed{20.6\;\mathrm{C}^{\circ}}.\]

\end{solution}
\begin{solution}{normal}
\textbf{(a)} The internal energy will be given by 
\[U=\mu C V \Delta T=\mu C\left(\frac{4}{3} \pi R^{3}\right)\left(T_{c}-T_{0}\right) \approx 16768\;\mathrm{J}.\]
\vspace{3mm}

\noindent \textbf{(b)} By Fourier's Law,
$$J=\dot{Q}=\kappa\frac{\Delta T}{R}=\frac{\kappa \left(T_{1}-T_{0}\right)}{R} \approx 2458 \;\mathrm{W/m^2}
$$
\vspace{3mm}

\noindent \textbf{(c)} Heat is transferred to the egg via its surface
$$
P=J \times 4 \pi R^{2}=\left[4 \pi K R\left(T_{1}-T_{0}\right)\right] \approx 19.3 \;\mathrm{W}
$$
\vspace{3mm}

\noindent \textbf{(d)} Power is equal to the change in internal energy over time or:
\[P=\frac{d U}{d t} \approx \frac{\Delta U}{\Delta t} \Rightarrow \Delta t=\frac{\Delta U}{P}=\frac{\mu c\left(\frac{4}{3} \pi R^{3}\right)\left(T_{c}-T_{0}\right)}{4 \pi \kappa R\left(T_{1}-T_{0}\right)}.\]
This means that 
\[\tau=\frac{\mu C R^{2}}{3 K}\left(\frac{T_{c}-T_{0}}{T_{1}-T_{0}}\right) \approx 869\;\mathrm{s}.\]
\end{solution}
\begin{solution}{normal}
\begin{enumerate}
    \item The temperature dependence of the bubble volume is $ w= w_0 - \alpha(T-T_0)V$. Therefore, by ideal gas law,
$$ \frac{pw}{T} = \frac{p_0w_0}{T_0} \implies p = \frac{p_0T}{T_0} \frac{w_0}{w_0 - V\alpha(T-T_0)}.$$
\item The equality no longer holds when the volume is negative or $$w_0-V\alpha(T-T_0) \leq 0 \implies \frac{w_0}{V\alpha} + T_0 \leq T.$$As $T$ approaches $T_{\text{max}}$ pressure increases. Due to high pressure compressibility of glycerin and elasticity of walls must be considered.
\end{enumerate} 

\end{solution}
\begin{solution}{hard}\textbf{A-i)} First, note that by the ideal gas law $pV = nRT_0$ which means that $\rho = \frac{PM_{air}}{RT_0}$. The differential change in pressure for a differential change in height $\mathrm{d}z$ is $\mathrm{d}P = -\rho g\mathrm{d}z$. Then, after substituting the above expression for the density of air as a function of pressure, we get the following expression: 
\[\int_{P_0}^{P} \frac{\mathrm{d}P}{P} = -\frac{M_{air}g}{RT_0}\int_{0}^{Z} \mathrm{d}z.\]
Evaluating, we find that $\alpha = \frac{M_{air}g}{RT_0}$.
\vspace{3mm}

\noindent \textbf{A-ii)} The density of air, $\rho$, can be taken as a constant; thus, \[\int_{P_0}^{P}\dd P = -\rho g\int_{0}^{Z} \mathrm{d}z,\]
and $P(z) = P_0 - \rho gh$.
\vspace{3mm}

\noindent \textbf{A-iii)} After substituting the given values, $P_B \approx 88.24\;\mathrm{kPa}$.
\vspace{3mm}

\noindent \textbf{B-i)} Consider an infinitesimally thin rectangular of prism of air with an area $A$ and a thickness $\mathrm{d}r$. If the pressure goes from $P$ to $P + \mathrm{d}P$ from one side of the piece of air to the other, the net force on it is $A \mathrm{d}P$. This net force is the centripetal force, so $A \mathrm{d}P = \frac{mv^2}{r}$. Noting that $m = \rho V = \rho A\mathrm{d}r$, and solving algebraically, $\frac{\mathrm{d}P}{\mathrm{d}r} = \rho_{\text{air}}\frac{v^2}{r}$.
\vspace{3mm}

\noindent \textbf{B-ii)}  Since angular momentum is conserved, $mv_Gr_G = mvr$; thus, $v = \frac{v_Gr_G}{r}$.
\vspace{3mm}

\noindent \textbf{B-iii)} Point G is on the isobar boundary layer, and so is Point B; therefore, it can be said that $P_G = P_B$. Then, using the Bernoulli equation, approximating $v_A \approx 0$, and substituting $P_G = P_B$, $P_A = \frac{1}{2}\rho_{air}v_G^2 + P_B$. Some algebraic manipulation yields $v_G = \sqrt{\frac{2(P_A - P_B)}{\rho_{\text{air}}}} = \sqrt{2gh} \approx 141\;\mathrm{m/s}$.
\vspace{3mm}

\noindent \textbf{B-iv)} Points C and G are also on the isobar boundary layer, meaning that $P_C = P_G$. Again using Bernoulli's equation, $P_G + \frac{1}{2}\rho_{air} v_G^2 = P_C + \rho_{air}gz + \frac{1}{2}\rho_{air}v_C^2$. Cancelling out $P_C$ and $P_G$ and $\rho_{air}$ as well as substituting the expression $v = \frac{v_Gr_G}{r}$ which was derived before, $v_G^2(1 - \frac{r_G}{r_C}^2) = 2gz$. As a result, $z = \frac{v_G^2}{2g}(1 - \frac{r_G}{r_C}^2)$. However, through a simple rearrangement of $v = \sqrt{2gh}$, $h = \frac{v_G^2}{2g}$. Thus, $\frac{z}{h} = 1 - \frac{r_G}{r}^2$, which enables us to graph $\frac{z}{h}$ vs. $\frac{r}{r_G}$, as shown.
\begin{center}
    \includegraphics[width=15cm]{graph.jpeg}
\end{center}
\vspace{3mm}

\noindent \textbf{B-v)}  Since the term $\frac{2gz}{v_G^2}$ will decrease as $v_G$ increases, the radius will be less dependent on the height for high-speed tornadoes. As a result, tornadoes with relatively uniform diameter tend to have higher ground rotation speeds.
\vspace{3mm}

\noindent \textbf{C-i)} The angular velocity inside the core is constant since it behaves like a rigid body. Therefore, $v = \omega r = v_G\frac{r}{r_G}$.
\vspace{3mm}

\noindent \textbf{C-ii)}  From before, $\frac{\mathrm{d}P}{\mathrm{d}r} = \rho_{\text{air}}\frac{v^2}{r}$. Multiplying both sides by $\mathrm{d}r$ and integrating the expression from the center to a far distance away, we have: 
\[P_D - P_0 = \rho \left(\int_{0}^{r_G}\frac{v_G^2r}{r_G^2} \mathrm{d}r + \int_{r_G}^{\infty}\frac{v_G^2r}{r_G^2} \mathrm{d}r\right).\]
Evaluating and solving algebraically for $P_D$, we get $P_D = P_0 - \rho v_G^2 \approx 76.48\;\mathrm{kPa}$.
\vspace{3mm}

\noindent \textbf{C-iii)} Assuming adiabatic behavior, $P^{1 - \gamma} \propto \frac{1}{T^\gamma}$. Therefore, $$T_G = T_0\left(\frac{P_G}{P_0}\right)^{\frac{\gamma - 1}{\gamma}} \approx 4.89\;\mathrm{^{\circ} C}$$ and $$T_D= T_0\left(\frac{P_D}{P_0}\right)^{\frac{\gamma - 1}{\gamma}} \approx -6.25\;\mathrm{^{\circ} C}.$$
\vspace{3mm}

\noindent \textbf{C-iv)} The low temperature causes condensation of moisture that gets sucked into the core, which releases a lot of latent heat, a source of the tornado's energy.
\vspace{3mm}

\noindent \textbf{D-i)} We assume that the house is tightly enclosed with pressure $P_0$ inside.
By substituting $v = v_G\frac{r}{r_G}$ into $\frac{\mathrm{d}P}{\mathrm{d}r} = \rho_{\text{air}}\frac{v^2}{r}$, the following integral can be set up: 
\[\Delta P = \rho_{air}\int_{2r_G}^{\infty} \frac{v_G^2r_G^2}{r^3} \mathrm{d}r.\]
Evaluating, $\Delta P = \frac{1}{8}\rho_{air}v_G^2$. Since the lift force is $F_L = \Delta PA$, the ratio of the lift force to weight for the roof is $\frac{F_L}{F_G} = \frac{\Delta PA}{\rho_{\text{roof}}Atg} = 3.75$.
\vspace{3mm}

\noindent \textbf{D-ii)} The lift force is not much larger than the weight of the roof, most roof are mounted firmly to withstand forces many times it’s weight. Therefore, chances are the pressure difference would not cause the house to explode so soon (unless the roof is very poorly mounted). Opening windows during a tornado isn't a good idea for another reason. Flying debris is responsible for most twister-related injuries, so standing next to an opening that could potentially blast you with shards of glass and other projectiles isn’t a great idea.

\end{solution}
\begin{solution}{normal}
\textbf{A-i)} The buoyant force is going to be: $$F=\rho_\text{air}V_\text{balloon}g$$
where $\rho_\text{air}=\frac{PM_A}{RT}$ from the ideal gas law. Applying the ideal gas law to inside the balloon gives:
$$RT = \frac{(P+\Delta P)V}{n}$$
Therefore, the buoyant force is:
$$\frac{P}{P+\Delta P}nM_Ag$$
\vspace{3mm}

\noindent \textbf{A-ii)} We know that $\rho = \frac{PM_{A}}{RT} \implies \rho = \frac{PM_Az_0}{RT_0(z_0-z)}$. The differential change in pressure for a differential change in height $\mathrm{d}z$ is $$\mathrm{d}P = -\rho g\mathrm{d}z \implies \mathrm{d}P = -\frac {PM_Az_0g}{RT_0(z_0-z)} \mathrm{d}z.$$
This means that by integrating,
$$\int_{P_0}^P \frac{\mathrm{d}P}{P} = -\frac {M_Az_0g}{RT_0} \int_0^z \frac{\mathrm{d}z}{z_0-z} \implies \ln \left(\frac{P}{P_0}\right) = \frac {M_Az_0g}{RT_0}\ln\left(\frac{z_0-z}{z_0}\right) $$
and
$$ P = P_0\left( 1 - \frac{z}{z_0}\right)^{\frac {M_A{z_0}g}{RT_0}}. $$
Also, note that $\rho_0 = \frac{P_0M_A}{RT_0} \implies \frac{M_A}{RT_0} = \frac{\rho_0}{P_0}$ and therefore:
$$\eta = \frac{\rho_0z_0g}{P_0}$$
\vspace{3mm}

\noindent \textbf{B-i)} We can apply the method of virtual work. The work needed to change the radius by $dr$ is:
$$\Delta P (4\pi r^2) dr$$which causes a change in energy of
$$dU = 4\pi r_0^2\kappa RT(4\lambda -4\lambda^{-5}) dr/r_0$$Equating gives:
$$\Delta P (4\pi r^2) dr = 4\pi r_0\kappa RT(4\lambda -4\lambda^{-5}) \implies \Delta P = \frac{4\kappa RT}{r_0}(2\lambda^{-1}+\lambda^{-7})$$The graph pretty much looks like:
$$\Delta P = \frac{8\kappa RT}{r_0\lambda}$$except for small values, at which it increases to infinity quickly.
\vspace{1.5mm}

\noindent We can alternatively treat the energy as:
$$U=4\pi r^2\gamma$$where $\gamma$ is a varying surface tension. Solving for $\gamma$ and dropping the constant factor, we can apply Laplace's pressure $\Delta P=4\gamma/R$ and solve for $\Delta P$.
\vspace{3mm}

\noindent \textbf{B-ii)} Initially $P_0V_0 = n_0RT_0$ where $V_0$ is the unstretched volume. Finally
$$(P_0+\Delta P)V_0\lambda^3 = nRT_0$$as $V \propto r^3$
using the result from part B(i) we get $\Delta P = \frac{4\kappa RT}{r_0}(\lambda^{-1}-\lambda^{-7}).$
This gives us 
$$ P_0V_0 \lambda^3 ( 1 + a (\lambda^{-1}-\lambda^{-7})) = nRT_0 \implies a = \frac{\left(\frac{n}{n_0}\right) \frac{1}{\lambda^3} - 1}{(\lambda^{-1}-\lambda^{-7})} = 0.11$$
\end{solution}
\begin{solution}{hard}
\definecolor{crimsonglory}{rgb}{0.75, 0.0, 0.2}
Energy conservation gives:
$$\frac{1}{2}v^2=\frac{1}{2}g\ell(1-\sin\theta)$$
where $\theta$ is the angle the rod makes with the ground at the point of maximum extension of the string. We are restricted to a total vertical length of $2\ell$ so we have:
$$H=2\ell\sin\theta \implies \sin\theta = \frac{H}{2\ell}$$
Applying this to our energy conservation expression gives:
$$v^2 = g\ell(1-H/2\ell) \implies \boxed{v=\sqrt{g(\ell-H/2)}}$$
Now, we use idea 44 and notice that horizontal acceleration of the centre must be zero; this follows from the Newton’s 2nd law for the horizontal motion (there are no horizontal forces at that moment). Further, notice that the vertical coordinate of the centre of mass is arithmetic average of the coordinates of the endpoints,
\[x_O = \frac{1}{2}(x_A + x_B)\]
Noting that $x_B$ must be constant, taking the time derivatives gives us 
\begin{align*}
\dot{x_O} = \frac{1}{2}\dot{x}_A\\
\ddot{x_O} = \frac{1}{2}\ddot{x}_A
\end{align*}
Hence, the acceleration of O can be found as half of the vertical acceleration of the rod’s upper end A; this is the radial, i.e. centripetal component of the acceleration of point A on its circular motion around the hanging point. From here, we know from the common formula, that 
\[a = \frac{v^2}{\ell}\]
substituting our expression for $v^2$ from part a) gives us 
\[a = \frac{g\ell(1-H/2\ell)}{\ell} \implies a = g(1 - H/2\ell).\]
At point $x_O$, the acceleration is then given by $\boxed{\frac{g}{2}\left(1 - \frac{H}{2\ell}\right)}$.
\tcbline
\textbf{Solution 2:} First, we make the following claim:
\vspace{2mm}

\textbf{\textcolor[HTML]{3D85C6}{Claim.}} At any position the potential energy lost is converted into $E_\text{rotational}+E_\text{translational}$.
i.e. $$\Delta U = \frac{1}{2}I_{CM}\omega ^2+\frac{1}{2}mv_{CM}^2$$
Coincidentally for this system $\Delta U $ reaches its maxima and $\omega$ becomes $0$ at the same time. 
When the thread becomes vertical, the angle made by the rod with the ground, $\beta$ is minimum $\implies \omega=0$.
\begin{proof} If $\alpha$ is the angle made by the thread with the vertical, 
$$l\cos\alpha + l\sin\beta=H$$
$$\sin\beta=\frac{H-l\cos\alpha}{l}$$
$\left | \alpha \right |$ is always acute here so $\cos\alpha\ $ reaches its maxima and $\beta$ reaches its minimum at $\alpha=0$.
At the same instant, $y_{CM}=\frac{l\sin\beta}{2}$ reaches its minima.
\end{proof}
When the thread is vertical:
$$y_{CM}=\frac{H-l}{2}$$
Initially:
$$y_{CM}=\frac{H}{4}$$
$$\therefore v_{max}=\sqrt{2g\left ( \frac{H}{4} -\frac{(H-l)}{2}\right )}=\sqrt{g\left ( l-\frac{H}{2} \right )}$$
At this instant, let the angular acceleration of the rod be $\alpha$ (into the plane) and COM's acceleration $a$ (upwards)
$$\frac{ml^2}{12}\alpha=(T-N)\frac{l}{2}\cos\beta$$
$$T+N-mg=ma$$
The bottom most point must have $0$ vertical acceleration:
$$\alpha\frac{l}{2}\cos\beta=a$$
The point connected to thread must have vertical acceleration $=\frac{v^2}{l}$
$$\alpha\frac{l}{2}\cos\beta+a=\frac{v^2}{l}=g\left ( 1-\frac{H}{2l} \right )$$
Also, $\sin\beta=\frac{H-l}{l}$
Solving gives:
$$a=\frac{g}{2}\left ( 1-\frac{H}{2l} \right )$$
$$T=\frac{mg}{4}\left(3+\frac{l}{6H}-\frac{H}{2l}\right)$$
\end{solution}

\begin{solution}{normal}
Since an ideal heat engine has practically zero heat and friction losses, it takes some amount of heat from one object and delivers  heat to another object having the same thermal capacity. Say $Q_A$ amount of heat was initially withdrawn from A and $Q_B$ heat given to B, and let $T_B<T_C<T_A$, where $T_C$ is the common temperature reached. By second law of thermodynamics, entropy is equal or:
$$\frac{Q_A}{T_1} = \frac{Q_B}{T_2}$$ 
Also, by definition 
\begin{align*}
    \dd Q_A &= -C  \dd T_1 \\
    \dd Q_B &= C \dd T_2
\end{align*}
From these we have $$-C\frac{\dd T_1}{T_1} = C \frac{\dd T_2}{T_2} \Rightarrow -C\int_{T_A}^{T_C}{\frac{\dd T_1}{T_1}} = C \int_{T_B}^{T_C}{\frac{\dd T_2}{T_2}} \Rightarrow \ln{\frac{T_C}{T_B}} = \ln{\frac{T_A}{T_C}} \Rightarrow T_C = \sqrt{T_A T_B}$$
We can also apply relations for differential quantities of $Q_A$ and $Q_B$ as we wrote earlier in the solution. For $\dd Q_A$, we have:
$$\dd Q_A = -C\dd T_1 \Rightarrow Q_A = C\int_{T_A}^{T_C}{\dd T_1} = C (T_A - T_C) = C(T_A - \sqrt{T_A T_B}).$$
For $\dd Q_B$:
$$\dd Q_B = -C\dd T_2 \Rightarrow Q_B = C\int_{T_C}^{T_B}{\dd T_2} = C (T_C - T_B) = C(\sqrt{T_A T_B} - T_B)$$
Now clearly by the first law of thermodynamics, we have 
$$\Delta{W} = Q_A - Q_B = C (T_A - T_B - 2\sqrt{T_A T_B}) = C {(\sqrt{T_A} - \sqrt{T_B})}^2.$$
\end{solution}
\begin{solution}{normal}
 Clearly the angular momentum is conserved about any point lying on the line passing through the rod. For convenience, let us choose the point where the collision occurs: $$M v \frac{\ell}{2} \ \hat{k} - \frac{M{\ell}^2}{12} \omega \ \hat{k} = 0 \Rightarrow \vec{\omega} = -\frac{6v}{\ell} \hat{k}$$ By momentum conservation, we have $$M v = m v_f \Rightarrow v_f = \frac{M}{m} v$$ where $v_f$ is the final velocity of the puck. Since the collision is elastic, $$ e = 1 = \frac{(v_f)-0}{(v+ \frac{\omega \ell}{2}) - 0} \Rightarrow v_f = v+\frac{\omega \ell}{2}$$ From these three equations, we obtain $$\frac{M}{m} = 4$$ 
\tcbline
Instead of using the equation of the restitution coefficient, we use energy conservation. $$\frac{1}{2} Mv^2 + \frac{1}{2} \left(\frac{M{\ell}^2}{12}\right) \omega^2 = \frac{1}{2}m{v_f}^2$$ Solving this equation with the equation for linear and angular momentum conservation yields the same answer. 
\end{solution}

\begin{solution}{normal}
The friction in the disk disappears because the $\text{CO}_2$ evaporates. The created pressure due to the force applied to the disk is given by 
\[P = \frac{F}{A} = \frac{F}{\pi r^2}.\]
Dry ice begins to sublime only when vapor pressure exceeds ambient pressure. Therefore, the minimum vapour pressure needed is given by 
\[P_{\text{vap}} = P_{\text{air}} + \frac{F}{\pi r^2} = 419.3\;\mathrm{Pa}.\]
From the graph, we can look at the points to see that $T \sim \boxed{213\;\mathrm{K}}$
\begin{center}
\includegraphics[width=15cm]{thermo78.jpeg}
\end{center}
\end{solution}
\begin{solution}{easy}
The heat exchange through walls is equal to $P\frac{t -t_0}{t_1-t_0}$ while the heat exchanged with incoming air $= \frac{v}{V} C_p(t-t_0)$ where $V$ is molar volume. Therefore 
\[P = P\frac {t -t_0}{t_1-t_0}+ \frac{v}{V} C_p(t-t_0)\]
which implies that
\[t = t_0 + \frac{P}{\frac{P}{t_1-t_0}+ \frac{vC_p}{V}} \Rightarrow t = 13.16^{\circ}\text{C}.\]
\end{solution}
\begin{solution}{easy}
The density of some matter of mass $m$ can be given by 
\[\rho = N \frac{m}{V} \]
where $N$ is the number density of the substance. This means that for dry and humid air on each respective air can be given as 
\[\rho_d = N_d \frac{m_d}{V}, \quad N_h \frac{m_h}{V}.\]
The number density of the dry air is given by 
\[N_d = \frac{M}{M_a} = \frac{M}{28.8}\]
while the number density of the humid air will be 
\[N_h = 0.02\cdot \frac{M^{\prime}}{M_w} + 0.98\cdot \frac{M^{\prime}}{M_a} = 0.02\cdot \frac{M^{\prime}}{28.8} + 0.98\cdot \frac{M^{\prime}}{18}.\]
We can now compare the ratio of densities since we require the number density to be the same or in other words, 
\[\frac{\rho_d}{\rho_h} = \frac{N_d m_d}{N_h m_h} = \frac{\frac{M}{28.8}\cdot M^{\prime}}{M^{\prime}\cdot \left(0.02\cdot \frac{M^{\prime}}{28.8} + 0.98\cdot \frac{M^{\prime}}{18}\right)M} = \frac{1}{28.8\left(\frac{0.02}{18} + \frac{0.98}{18}\right)} = 0.9881.\]
This means that the moist air is then 
\[\rho_m = \rho_h = 0.9881 \rho_d = 1.2352\;\mathrm{kg/m^3}.\]
\end{solution}
\begin{solution}{hard} First, let us notice that the period of oscillations $T=0.01 \text{ s}$ is extremely small, so any deviation in the velocity caused by friction can be ignored if we only focus on the average velocity. We assume that the block travels at a constant velocity $u$ rightwards in the positive direction. Let us examine the movement qualitatively.
\vspace{3mm}

As the board starts moving rightwards, it is important to note that the velocity of the block relative to the board is rightwards, so the friction force $mg\mu_1$ points leftwards. This goes on for a time $t_1$ until the velocity of the board matches the velocity of the block and overtakes it. This goes on for a time $t_2$ where the board reaches a maximum and starts to slow down all the way until it has a velocity of $u$ again. During this time period, the friction force points to the right with a magnitude $mg\mu_2$. Finally, for a time $t_3=t_1$, the board is still moving towards the right but the friction force points towards the left. The total duration is $t_1+t_2+t_3=T/2$.
\vspace{3mm}

Finally, the board starts travelling in the leftwards direction. The friction force here is a constant $mg\mu_1$ directed towards the left and lasts for a time $t_4=T/2$
\vspace{3mm}

Now let's do the math. Let's work with the assumption we made that the block has a roughly constant average velocity. If this was not the case, then friction forces would either speed it up or slow it down until the motion is roughly constant. As a result, the total change in momentum, or impulse is zero. We have:
$$(-mg\mu_1)t_1+(mg\mu_2)t_2+(-mg\mu_1)t_3+(-mg\mu_1)t_4=0$$
Letting $t_1=t_3$ we get:
$$\mu_2t_2=\mu_1t_4+2\mu_1t_1$$
Having $2t_1+t_2=t_4$ then we have:
$$\mu_2(t_4-2t_1)=\mu_1t_4+2\mu_2t_1 \implies (\mu_2-\mu_1)t_4=4\mu_1t_1$$
or:
$$t_1 = \frac{(\mu_2-\mu_1)t_4}{2(\mu_2+\mu_1)}$$
Since $t_4=0.005 \text{ s}$ we get:
$$t_1 = \frac{t_4}{8}$$
this is an eighth of half a period and corresponds to the time where the board has the same velocity as the block. I put this through a visual program and determined this corresponds to $0.64 \text{ m/s}$. To one significant digit, , the average velocity of the board is $\boxed{v=0.6 \text{ m/s}}$
\end{solution}

\begin{solution}{hard}
At a temperature of $77.4\;\mathrm{K}$ (i.e. at the boiling point of nitrogen), the pressure of saturated nitrogen vapour is $1\;\mathrm{atm}$ ,while the saturated pressure of oxygen becomes $1\;\mathrm{atm}$ at a higher temperature of $90.2\;\mathrm{K}$.
\vspace{3mm}

\noindent The molar ratio of oxygen and nitrogen is $21:79$. The ratio of the partial pressures of the two components will also be very close to molar ratio, because, until the start of liquefaction, the behaviour of each gas constituent is very close to that of an ideal gas. When the partial pressure of oxygen is $0.2226\;\mathrm{atm}$, liquefaction of oxygen starts. The partial pressure of oxygen does not increase further as temperature is constant.The partial pressure of $N_2$ at the onset of oxygen liquefaction is $0.2226\times \frac{79}{21}\,\text{atm} = 0.8374\;\mathrm{atm}$. This is less than the saturated vapour pressure of nitrogen at this temperature, which, since $77.4\,\text{K}$ is nitrogen’s boiling point, has a value of $1\,\text{atm}$.Consequently, nitrogen does not liquefy at this pressure. Therefore total pressure at this point $P_1$ is $0.8374+0.2226=1.06 \,\text{atm}$. As the compression is isothermal, volume at this point $V_1$ is $1.001\times \frac{0.500}{1.06} \,\text{l} = 0.4721 \,\text{l}$. During the subsequent compression, the partial pressure of the oxygen, already in two phases, does not change, while the nitrogen pressure increases from $0.8374\,\text{atm}$ to $1.00\,\text{atm}$. This latter pressure will be reached when the volume has been reduced by a factor of $(0.8374/1.00)=0.8374$. Therefore volume at this point $V_2$ is $0.8374\times 0.4721\,\text{l} = 0.3953\,\text{l}$.After that, the total pressure remains constant (at $0.2226+1.00= 1.22\,\text{atm}$) until the liquefaction is complete, just the volume is decreased now.
\end{solution}
\begin{solution}{hard} \textbf{(a)} Consider a cross-section with width $\text{d}h$. The pressure is then given by 
\[\dd P = - \rho g \dd h.\]
From the ideal gas law, we have that 
\[\rho = \frac{PM}{RT}\]
which means that 
\[\int_{0}^{H} -\frac{Mg}{RT}\dd h = \int_{P}^{P/e} \frac{\dd P}{P}\implies - \frac{MgH}{RT} = \ln \left(\frac{1}{e}\right) = -1.\]
This means that 
\[MgH = RT\implies \bar{M} = \frac{RT}{gH} = 14.5\;\mathrm{g/mol}.\]
\vspace{3mm}

\noindent \textbf{(b)} By the ideal gas law 
\[\frac{n}{N_A}RT = P_0\implies n = \frac{N_A P_0}{RT}.\]
The mean free path $\lambda$ is given by 
\[\lambda = (\sigma n)^{-1}\implies \lambda = \frac{RT}{N_A \sigma P_0} = 3286.67\;\mathrm{m}.\]
\vspace{3mm}

\noindent \textbf{(c)} We are given in part (a) that 
\[\lambda_{\text{EB}} = H \implies n_{\text{EB}} = \frac{1}{\lambda_{\text{EB}} \sigma} = \frac{1}{\sigma H}.\]
This means that from the ideal gas law:
\[P_{\text{EB}} = \frac{n_{\text{EB}} RT}{N_A}.\]
Now, once again integrating and doing the same steps as in part (a), we have that 
\[-\int_{h_0}^{h_{\text{EB}}} \frac{Mg \dd h}{RT} = \int_{P_0}^{P_{\text{EB}}} \frac{\dd P}{P}\implies -(h_{\text{EB}} - h )\frac{Mg}{RT} = \ln \left(\frac{P_{\text{EB}}}{P_0}\right)\implies h_{\text{EB}} = 425\;\mathrm{km}.\]
\vspace{3mm}

\noindent \textbf{(d)} Note that Maxwell's distribution in spherical form is given by 
\[f(v) dv = v^2 \left(\frac{m}{2\pi k_B T}\right)^{3/2} \exp \left(-\frac{mv^2}{2k_B T}\right) \sin\theta d\theta d\varphi.\]
The probability of the particle having a velocity more than the escape velocity will then be given by $\int_{v_{\text{esc}}}^{\infty} f(v) dv$ or in other words, integrating over all spaces gives us 
\begin{align*}
P(v) &= \left(\frac{m}{2\pi k_B T}\right)^{3/2} \int_{v_{\text{esc}}}^{\infty} v^2 \exp \left(-\frac{mv^2}{2k_B T}\right) \int_{0}^{\pi/2} \sin\theta d\theta \int_{0}^{2\pi} d\varphi\\
&= \sqrt{\frac{2m^3}{\pi k_B^3T^3}} \int_{v_{\text{esc}}}^{\infty} v^2 \exp \left(-\frac{mv^2}{2k_B T}\right)
\end{align*}
We need to calculate the escape velocity at the exobase which can be done with the equation 
\[v_{\text{esc}} = \sqrt{\frac{2GM}{R + h}} = 1.08\times 10^4 \;\mathrm{m/s}\]
and therefore, the integral gives us
\[P(v) = \sqrt{\frac{2m^3}{\pi k_B^3T^3}} \int_{1.08\times 10^4}^{\infty} v^2 \exp \left(-\frac{mv^2}{2k_B T}\right) = \boxed{2.6\cdot 10^{-3}}.\]
\vspace{3mm}

\textbf{(e)} Note that flux is given by $\Phi = \frac{dN}{dA\cdot dt}$. We can calculate $dN$ first using the spherical Maxwell's distrubution and using the fact that the number of molecules coming in at a certain time is given by $nv\cos\theta dAdt$ where $dA$ is a surface element. Therefore, expanding $N$ gives us
\[dN = \sum n_H v\cos\theta dAdt \cdot f(v, \theta, \varphi) dv d\theta d\varphi\]
which tells us 
\[dN = n_H \left(\frac{m}{2\pi k_B T}\right)^{3/2} \exp \left(-\frac{mv^2}{2k_B T}\right) v^3 dv \cdot \sin\theta \cos\theta d\theta \cdot d\varphi \cdot dA dt.\]
Integrating over $\theta$ and $\varphi$ tells us 
\[dN = n_H \left(\frac{m}{2\pi k_B T}\right)^{3/2} \exp \left(-\frac{mv^2}{2k_B T}\right) v^3 dA dt\]
and therefore, 
\[d\Phi = \frac{dN}{dA dt} = n_H \left(\frac{m}{2\pi k_B T}\right)^{3/2} \exp \left(-\frac{mv^2}{2k_B T}\right) v^3 dv.\]
The flux of the escaping atoms are then given as 
\[\Phi = \int_{v_{\text{esc}}}^{\infty} n_H \left(\frac{m}{2\pi k_B T}\right)^{3/2} \exp \left(-\frac{mv^2}{2k_B T}\right) v^3 dv = \boxed{7.5\times 10^{11}\;\mathrm{1/m^2}}.\]
\vspace{3mm}

\textbf{(f)} Note that the atmosphere produces a force of $F = PA = 4\pi R_{\odot}^2 P$. Equating this force to the mass of the atmosphere tells us that 
\[mg = 4\pi R_{\odot}^2 P\implies m = \frac{4\pi R_{\odot}^2 P}{g}.\]
We then write 
\[N = \frac{N_A m}{M_{\text{Atm}}} = \frac{4\pi N_A R_{\odot}^2 P}{M_{\text{Atm}} g}.\]
Then note that since Nitrogen is diatomic, we have $N_H = 2\chi_H N = 1.2\times 10^{38}$ which gives us our answer.
\vspace{3mm}

\textbf{(g)} The number of molecules escaping after a unit time is 
\[\dot{N}_H = \Phi (R_{\odot} + h_{EB})^2 = 4.35\times 10^{26}\;\mathrm{1/s}.\]
The total time required for half the atmosphere to evaporate is then given by the characteristic time interval:
\[\tau = \frac{N_H/2}{\dot{N}_H} \approx \boxed{4500\;\mathrm{years}}.\]
\vspace{3mm}

\textbf{(h)} Redoing the calculations from the other parts for hydrogen gives us our new answer of $9.6\times 10^{11}\;\mathrm{years}.$
\vspace{3mm}

\textbf{(i)} One possible reason of why there is currently still some hydrogen in the Earth’s atmosphere is that the hydrogen inside the Earth's water gets cycled through the hydrologic cycle.

\end{solution}
\begin{solution}{hard}
\textbf{(a)} Since the temperature is constant, we have an isothermal compression where:
$$P_0V_0=P_1V_1 \implies P_0r_0^3 = P_1r_1^3$$Since $r_1=\frac{1}{2}r_0$, we have $P_1=8P_0$.
\vspace{2mm}

\textbf{(b)} The heat generated is equal to the negative work done:
$$Q=-W=-\int P dV$$Using $PV=nRT$, we get:
$$Q=-nRT\ln\left(\frac{V_0}{V_3}\right)=\frac{3mRT_0}{\mu}\ln\left(\frac{r_0}{r_3}\right)$$
\textbf{(c)} We now have an adiabatic compression since not heat is exiting the system:
$$PV^\gamma=\text{constant} \implies TV^{\gamma-1}=\text{constant} \implies T_3=T_0\left(\frac{r_3}{r_0}\right)^{3\gamma-3}.$$
\textbf{(d)} The final radius refers to the point in which the gravitational potential energy is comparable to the thermal energy:
$$\frac{GM\mu}{r_4}=RT_0\left(\frac{r_3}{r_4}\right)^{3\gamma-3}\implies r_4=r_3\left(\frac{RT_0r_3}{\mu Gm}\right)^\frac{1}{4-3\gamma}$$and using the relationship from part (c), we get:
$$T_4=T_0\left(\frac{RT_0r_3}{\mu Gm}\right)^\frac{3\gamma-3}{4-3\gamma}$$
\end{solution}
\begin{solution}{hard}
\textbf{(a)} First, we assume that the process is reversible (even though this is not very likely). Then, the work done on the liquid is:
$$dW=(P-P_0)4\pi r^2 dR = dE_k \implies (P-P_0) R^2 dR = \frac{\rho}{2}d(r^3 \dot{r}^2)$$
The initial pressure is given by:
$$P_i=\frac{P_0R_0^3}{R_i^3}$$
so using $PV^\gamma$, we get:
$$P = P_0 \left(\frac{R_0^3}{R_i^3}\right)\left(\frac{R_i}{r}\right)^{3\gamma}$$
Since $\gamma=5/3$, this simplifies to:
$$P_0\left(49\left(\frac{R_0}{r}\right)^{5}-1\right) r^2 dr = \frac{\rho}{2}d(r^3 \dot{r}^2)$$
Integrating the left hand side, we can first make the substitution $\beta=\frac{r}{R_0}$ to simplify it to:
$$P_0R_0^3 \int_7^\alpha \left(\frac{49}{\beta^3}-\beta^2\right) d\beta = P_0R_0^3 \left(\frac{1}{2}-\frac{49}{2\alpha^2}+\frac{7^3}{3}-\frac{\alpha^3}{3}\right)$$
The right hand side evaluates to zero since it starts and ends off at rest. Thus, setting this to zero, we get the equation:
$$6\alpha^{5}+147-689\alpha^{2}$$
Making the assumption that $6\alpha^5 \ll 1$, we get a quadratic:
$$\alpha=\sqrt{\frac{147}{689}} \implies R_\text{min}=0.462R_0=2.31 \,\mathrm{\mu m}$$
We also know from $TV^{\gamma-1}$ that the maximum temperature is thus:
$$T_\text{max}=6.86 \times 10^4 \text{ K}.$$
\vspace{3mm}

\noindent \textbf{(b)} We can apply the same differential equation. The LHS stays the same, but the RHS no longer becomes zero. The RHS can be evaluated to:
$$\int_{0}^{\alpha^3\dot{\beta}^2} \frac{\rho R_0^5}{2} d(\beta^3\dot{\beta}^2)=\frac{\rho R_0^5}{2}(\alpha^3\dot{\beta}^2)$$
Setting it equal, we see that:
$$P_0R_0^3 \left(\frac{1}{2}-\frac{49}{2\alpha^2}+\frac{7^3}{3}-\frac{\alpha^3}{3}\right)=\frac{\rho R_0^5}{2}(\alpha^3\dot{\beta}^2) \implies \dot{\beta}^2 \propto \frac{689}{6\alpha^3}-\frac{49}{2\alpha^5}-\frac{1}{3}$$
This is at a maximum when:
$$\alpha=\sqrt{\frac{6\cdot5\cdot49}{2\cdot3\cdot689}}=0.596 \implies R_f=2.98 R_0$$
\vspace{3mm}

\noindent \textbf{(c)} We make the assumption that between these two times, the speed is roughly the same. The average radius is:
$$\langle R\rangle = 2.645 \,\mathrm{\mu m}$$
and thus plugging in this into our expression for $\dot{r}$ gives $\dot{r}=192.77 \text{ m/s}$ such that the total time is:
$$\Delta t= 3.48 \times 10^{-9} \text{ s}$$
\vspace{3mm}

\noindent \textbf{(d)} By Stefan-Boltzmann Law, we have that 
\[\dot{Q} = a\sigma \cdot 4\pi r^2 T^4.\]
Substituting 
\[T = T_0 \left(\frac{R_i}{r}\right)^2,\]
we result in the expression of
\[\dot{Q} = 4\pi a \sigma R_i^8 T_0^4/r^6.\]
We require that 
\[Q\leq \frac{1}{5}U\implies \left|\dot{Q}\right| \leq \left|\frac{1}{5}\dot{U}\right|\]
and therefore, we attempt to find $\dot{U}$ as well. Note that 
\[\dot{U} = -P_i \dot{V} = -P_i \left(\frac{V_i}{V}\right)^{\gamma} \dot{V} =  -4\pi P_i R_i^5 \dot{r}/r^3.\]
We now can set our expression to be 
\[\frac{4\pi a\sigma R_i^8 T_0^4}{r^6} \leq \frac{1}{5}\cdot \frac{4\pi P_i R_i^5\dot{r}}{r^3}\implies a \leq \frac{P_i r^3 \dot{r}}{5R_i^3\sigma T_0^4}\approx 0.0107\]
\end{solution}
\begin{solution}{hard}
\textbf{(a)} We know that $$C_v = \frac{\text{d}E_{\text{avg}}}{\text{d}T}$$
A free electron has $3$ degrees of freedom. Therefore: $$E_{\text{avg}} = \frac{3}{2} k_BT\implies C_v = \frac{3}{2} k_B$$
\vspace{3mm}

\noindent \textbf{(b)} The total energy is 
$$U = \int_0^S Ef(E) dS = CV\int_0^{\infty}E^{3/2}f(E) dE.$$
From the graph
$$ f(E) =
\begin{cases}
1 &\quad 0 \leq E \leq E_{\text{F}}- 2k_BT \\
\frac{E}{4k_BT} + \frac{E_{\text{F}}+ 2k_BT}{4k_BT} &\quad E_{\text{F}}- 2k_BT\leq E \leq E_{\text{F}}+ 2k_BT\\
0 &\quad E \geq E_{\text{F}}+ 2k_BT \\
\end{cases} $$
Therefore, by integrating from all cases we have:
$$ \int_0^{\infty} E^{3/2} f(E) dE = \int_0^{E_{\text{F}}- 2k_BT} E^{3/2} \cdot 1 \cdot dE + \int _{E_{\text{F}}- 2k_BT}^{E_{\text{F}}+2k_BT} E^{3/2}\frac{E_{\text{F}}- 2k_BT-E}{4k_BT} dE + \int_{E_{\text{F}}+2k_BT}^{\infty} E^{3/2}\cdot 0 \cdot dE$$
Simplifying and evaluating the integral tells us 
\begin{align*}\int_0^{\infty} E^{3/2} f(E) dE = \frac{2\left(E_{\text{F}}- 2k_BT\right)^{5/2}}{5} &+ \frac{2\left(E_{\text{F}}+ 2k_BT\right)\left(\left(E_{\text{F}}+2k_BT\right)^{5/2}-\left(E_{\text{F}}- 2k_BT\right)^{5/2}\right)}{5\times 4k_BT} \\
&- \frac{2\left(\left(E_{\text{F}}+2k_BT\right)^{7/2} - \left(E_{\text{F}}- 2k_BT\right)^{7/2}\right)}{4k_BT}.\end{align*}
As $k_BT \ll E_{\text{F}}$
$$\int_0^{\infty} E^{3/2}f(E)dE = \frac{2E_{\text{5/2}}}{5} \left( 1 - \frac{5k_BT}{E_{\text{F}}} + \frac{15}{2} \left(\frac{k_BT}{E_{\text{F}}}\right)^2\right) + E_{\text{F}}^{5/2}\left(1+ \frac{2k_BT}{E_{\text{F}}}\right)- E_{\text{F}}^{5/2} = \frac{2E_{\text{F}}^{5/2}}{5} + 3 \left(\frac{k_BT}{E_{\text{F}}}\right)^2E_{\text{F}}^{5/2}.$$
Therefore:
$$U = CVE_{\text{F}}^{5/2} \left( \frac{2}{5} + 3 \left( \frac{k_BT}{E_{\text{F}}}\right)^2\right)$$
and
$$E_{\text{avg}} = \frac{U}{N}$$ where $ N = \int dS = CV \int_0^{E_{\text{F}}^0} E^{1/2} dE$ as electrons are below energy below $E_{\text{F}}^0$ at $T = 0$K. It is given that fermi level is not dependent on temperature. Finally, we note that
 $E_{\text{F}}^0= E_{\text{F}} \implies N = \frac{2CV E_{\text{F}}^{3/2}}{3}$
 which means
$$E_{\text{avg}} = \frac{3}{2} E_{\text{F}} \left( \frac{2}{5} + 3 \left( \frac{k_BT}{E_{\text{F}}}\right)^2\right).$$
The heat capacity is then:
$$C_v = \frac{\text{d}E_{\text{avg}}}{\text{d}T} = \frac{9k_B^2T}{E_{\text{F}}}.$$
\end{solution}
\begin{solution}{normal}
\textbf{(a)} The energy for spin up dipole is given by $\mu_BB$ while the energy for spin down dipole $-\mu_BB$. Note that by Boltzmann’s law
\[\text{Probability:}\,\,p = Ae^{-E/k_BT}\,\,\,\,\text{where A is a constant}.\]
As there are only 2 possible states for the dipoles. The total sum of both states must add up to $1$ by the rule of probability. 
$$p_{\mu_B} + p_{-\mu_B} = 1\implies A = \frac{1}{e^{-\mu_BB/k_BT} + e^{\mu_BB/k_BT}}$$
Therefore, $ p_{\mu_B} = \frac{e^{-\mu_BB/k_BT}}{e^{-\mu_BB/k_BT} + e^{\mu_BB/k_BT}}$ and $ p_{-\mu_B} = \frac{e^{\mu_BB/k_BT}}{e^{-\mu_BB/k_BT} + e^{\mu_BB/k_BT}}$. The average energy will be the weighted average of both states which is the multiplied by $N$ or in other words:
$$E_s = \left(\frac{\mu_BBp_{\mu_B}+\left(-\mu_BBp_{-\mu_B}\right)}{p_{\mu_B}+p_{-\mu_B}}\right)N$$
Therefore,
$$E_s = N\mu_BB\left(\frac{e^{-\mu_Bb/k_BT}- e^{\mu_Bb/k_BT}}{e^{-\mu_Bb/k_BT}+e^{\mu_Bb/k_BT}}\right) = -N\mu_BB \tanh \left(\frac{\mu_Bb}{k_BT}\right) $$'

\textbf{(b)} For $1\gg \frac{\mu_BB}{k_BT}$, we can approximate the energy to be 
$$E_s \approx \frac{-N\left(\mu_BB\right)^2}{k_BT}$$
We then find that $C = \frac{\text{d}E_s}{\text{d}T} = \frac{N\left(\mu_BB\right)^2}{k_BT^2}$
\tcbline
Here is a purely statistical analysis of this problem. Suppose there are $N_\uparrow$ states and $N_\downarrow$ states such that $N_\uparrow+N_\downarrow=N$ and the total energy is $U=\mu B(N-2N_\uparrow)$. Then, the multiplicity of having $N_\uparrow$ states is given by:
$$\Omega = \frac{N!}{N_\uparrow!N_\downarrow!}$$The entropy of this is thus:
$$S=k\ln\Omega=k\left[\ln(N!)-\ln(N_\uparrow!)-\ln\left((N-N_\uparrow)!\right)\right]$$Applying Stirling's approximation, we get:
$$S/k=N\ln N-N_\uparrow \ln(N_\uparrow)-(N-N_\uparrow)\ln(N-N_\uparrow)$$The definition of temperature is given by:
$$\frac{1}{T}=\frac{\partial S}{\partial U} = \frac{\partial N_{\uparrow}}{\partial U} \frac{\partial S}{\partial N_{\uparrow}}=-\frac{1}{2\mu B}\frac{\partial S}{\partial N_\uparrow}$$where
$$\frac{\partial S}{\partial N_\uparrow}=-\ln\left(\frac{N_\uparrow}{N-N_\uparrow}\right)=\ln\left(\frac{N-U/\mu B}{N+U/\mu B}\right)$$where we have substituted $N_\uparrow = \frac{N}{2}-\frac{U}{2\mu B}$. Solving for $U$, we get the desired:
$$U=-N\mu B\tanh\left(\frac{\mu B}{kT}\right)$$
\end{solution}
\begin{solution}{normal}
We consider the vessel to be spherically symmetric. Let the mass of vessel be $m$ and the radius of vessel be $r$ and thickness of steel be $d$ with $d\ll r$
$$m = 4\pi r^2 d \rho \Rightarrow 2\pi r^2 d = \frac{m}{2\rho}$$
For the hydrogen stored $$PV = \nu RT \Rightarrow P\left(\frac{4}{3}\pi r^3\right)= \nu RT\Rightarrow P = \frac{\nu RT}{\frac{4\pi r^3}{3}}$$
For the vessel to not break $$P\cdot \pi r^2 \leq \sigma\cdot 2\pi rd \Rightarrow \frac{\nu RT}{\frac{4\pi r^3}{3}}\pi r^2 \leq \sigma\cdot 2\pi rd \Rightarrow \frac{3RT\nu}{4} \leq \sigma \cdot2\pi r^2 d.$$
This implies that
$$\frac{3RT\nu}{4} \leq \sigma\cdot \frac{m}{2\rho}\implies \frac{3RT\nu \rho}{2\sigma} \leq m \implies m_{\text{min}} = \frac{3RT\nu\rho}{2\sigma}.$$
\end{solution}


\end{document}
