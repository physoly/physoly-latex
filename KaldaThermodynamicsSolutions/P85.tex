\begin{solution}{hard}
\textbf{(a)} First, we assume that the process is reversible (even though this is not very likely). Then, the work done on the liquid is:
$$dW=(P-P_0)4\pi r^2 dR = dE_k \implies (P-P_0) R^2 dR = \frac{\rho}{2}d(r^3 \dot{r}^2)$$
The initial pressure is given by:
$$P_i=\frac{P_0R_0^3}{R_i^3}$$
so using $PV^\gamma$, we get:
$$P = P_0 \left(\frac{R_0^3}{R_i^3}\right)\left(\frac{R_i}{r}\right)^{3\gamma}$$
Since $\gamma=5/3$, this simplifies to:
$$P_0\left(49\left(\frac{R_0}{r}\right)^{5}-1\right) r^2 dr = \frac{\rho}{2}d(r^3 \dot{r}^2)$$
Integrating the left hand side, we can first make the substitution $\beta=\frac{r}{R_0}$ to simplify it to:
$$P_0R_0^3 \int_7^\alpha \left(\frac{49}{\beta^3}-\beta^2\right) d\beta = P_0R_0^3 \left(\frac{1}{2}-\frac{49}{2\alpha^2}+\frac{7^3}{3}-\frac{\alpha^3}{3}\right)$$
The right hand side evaluates to zero since it starts and ends off at rest. Thus, setting this to zero, we get the equation:
$$6\alpha^{5}+147-689\alpha^{2}$$
Making the assumption that $6\alpha^5 \ll 1$, we get a quadratic:
$$\alpha=\sqrt{\frac{147}{689}} \implies R_\text{min}=0.462R_0=2.31 \,\mathrm{\mu m}$$
We also know from $TV^{\gamma-1}$ that the maximum temperature is thus:
$$T_\text{max}=6.86 \times 10^4 \text{ K}.$$
\vspace{3mm}

\noindent \textbf{(b)} We can apply the same differential equation. The LHS stays the same, but the RHS no longer becomes zero. The RHS can be evaluated to:
$$\int_{0}^{\alpha^3\dot{\beta}^2} \frac{\rho R_0^5}{2} d(\beta^3\dot{\beta}^2)=\frac{\rho R_0^5}{2}(\alpha^3\dot{\beta}^2)$$
Setting it equal, we see that:
$$P_0R_0^3 \left(\frac{1}{2}-\frac{49}{2\alpha^2}+\frac{7^3}{3}-\frac{\alpha^3}{3}\right)=\frac{\rho R_0^5}{2}(\alpha^3\dot{\beta}^2) \implies \dot{\beta}^2 \propto \frac{689}{6\alpha^3}-\frac{49}{2\alpha^5}-\frac{1}{3}$$
This is at a maximum when:
$$\alpha=\sqrt{\frac{6\cdot5\cdot49}{2\cdot3\cdot689}}=0.596 \implies R_f=2.98 R_0$$
\vspace{3mm}

\noindent \textbf{(c)} We make the assumption that between these two times, the speed is roughly the same. The average radius is:
$$\langle R\rangle = 2.645 \,\mathrm{\mu m}$$
and thus plugging in this into our expression for $\dot{r}$ gives $\dot{r}=192.77 \text{ m/s}$ such that the total time is:
$$\Delta t= 3.48 \times 10^{-9} \text{ s}$$
\vspace{3mm}

\noindent \textbf{(d)} By Stefan-Boltzmann Law, we have that 
\[\dot{Q} = a\sigma \cdot 4\pi r^2 T^4.\]
Substituting 
\[T = T_0 \left(\frac{R_i}{r}\right)^2,\]
we result in the expression of
\[\dot{Q} = 4\pi a \sigma R_i^8 T_0^4/r^6.\]
We require that 
\[Q\leq \frac{1}{5}U\implies \left|\dot{Q}\right| \leq \left|\frac{1}{5}\dot{U}\right|\]
and therefore, we attempt to find $\dot{U}$ as well. Note that 
\[\dot{U} = -P_i \dot{V} = -P_i \left(\frac{V_i}{V}\right)^{\gamma} \dot{V} =  -4\pi P_i R_i^5 \dot{r}/r^3.\]
We now can set our expression to be 
\[\frac{4\pi a\sigma R_i^8 T_0^4}{r^6} \leq \frac{1}{5}\cdot \frac{4\pi P_i R_i^5\dot{r}}{r^3}\implies a \leq \frac{P_i r^3 \dot{r}}{5R_i^3\sigma T_0^4}\approx 0.0107\]
\end{solution}