\begin{solution}{normal}\blfootnote{The same problem was presented in the book \textit{Problems in Physics} by SS Krotov.}
\begin{center}
\begin{asy}
import olympiad;
size(5cm);
draw((1.8,0)--(4.2,0), linewidth(1.5pt));
pair A,B,C,D,M;
A = (2,0.0001);
B = (4,0.0001);
C = (2.5,-1.732/2);
D = (3.5, -1.732/2);
label("A", A+(0.1,0),SE);
label("B", B-(0.1,0),SW);
label("C", C,NE);
label("D", D,NW);
dot(A);
dot(B);
dot(C);
dot(D);
draw(A--C, linewidth(1pt));
draw(C--D, linewidth(1pt));
draw(D--B, linewidth(1pt));
M = (3,-1.732);
label("M",M,S);
draw(M--C,dotted);
draw(M--D,dotted);
draw(C--(C-(0,0.3)), arrow = Arrow(4));
label("mg", C-(0,0.3), S);
pair X = D+(1.732/2 * 0.3, -0.5*0.3);
draw(D--X, arrow = Arrow(4));
label("F", X, SE);
\end{asy}
\end{center}
Due to symmetry of the system and by the fact that $AB-CD = CD = AC = BD$, we see that $$\angle CAB = \angle DBA = 60^{\circ}$$
such that $AC \cos \angle CAB = \frac{AC}{2}$. Now consider the net torque acting on rod $CD$ about the meeting point of extended $AC$ and $DB$, which we call $M$. The tension forces due to rods $AC$ and $DB$ pass through $M$ and exert no torque. The only torques are due to $mg$ and $F$. The torque due to $mg$ is
$$mg\ell \sin 30^{\circ} = \frac{mg\ell}{2}$$
where $\ell \equiv MC = MD = CD$. For the minimum value of $F$, it must point perpendicular to $MB$ and its value must be $$\frac{mg\ell}{2\ell} = \boxed{\frac{mg}{2}}$$

\tcbline

By using symmetry, we can determine angle $\angle C = \angle D = 120^\circ$ of the isosceles trapezoid. We will use fact 20 cited in the handout which in brief states that if a mass-less rod is freely hinged at both ends, the force at the hinge must point along the rod. This is the only way for the torque to be zero.

Therefore, the rod $AC$ must provide an upwards force of $mg$. By using the given angle above and breaking the force up into its components, we can see that the horizontal force is $mg\tan 30^\circ$. This must also be the horizontal force the rod exerts on hinge $D$ due to Newton's third law. Therefore, we know that the vertical forces $BD$ and $F$ exerts on $D$ has to sum up to zero and their horizontal forces have to sum up to $mg/2$. Therefore, we have:

$$F\cos\theta + T\sin 30^\circ = mg\tan 30^\circ$$and
$$F\sin\theta = T\cos 30^\circ$$
Combining them together to remove $T$ gives:

\begin{align*}
F\left(\cos\theta + \frac{\sin\theta}{\sqrt{3}}\right) = mg/2
\end{align*}
You can minimize $F$ by taking the derivative or you can recognize that the second part of the left hand side reaches a maximum of
$$\sqrt{1^2+\left(\frac{1}{\sqrt{3}}\right)^2}=\frac{2}{\sqrt{3}}$$so the minimum $F$ is achieved when this value is reached or:
\begin{align*}
F\left(\frac{2}{\sqrt{3}}\right) &= \frac{mg}{\sqrt{3}} \\
F &= \frac{mg}{2}
\end{align*}
\end{solution}
