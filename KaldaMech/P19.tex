\newpage
\begin{solution}{hard}
In a rotating reference frame, we have that 
\[\vec\omega_3 = \vec\omega_1 + \vec\omega_2\]
where $\vec\omega_1$ is the angular velocity in the reference frame, $\vec\omega_2$ is the angular velocity of the body in the rotating reference, and $\vec\omega_3$ is that in the stationary frame. If you consider the reference point to be at infinity, then you find that the rotational motion of the disk becomes negligible. Therefore, we have that \blfootnote{This problem was found in the book 'Aptitude Test Problems in Physics' by S.S. Krotov.}
\[0 = \vec\omega_1 + \vec\omega_2\]
\[\boxed{\vec\omega_1 = -\vec\omega}\]
\end{solution}
