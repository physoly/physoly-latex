\begin{solution}{normal}
Let's just tackle part B straight away. Part A follows the same reasoning and can be derived through part B. The normal force is given by:

$$N=mg\cos\alpha-F\sin\theta$$
While adding forces in the direction along the plane gives (assuming maximum friction):
$$F\cos\theta+mg\sin\alpha-\mu N=0$$
Substituting our expression for $N$ we can write $F$ in terms of all our variables:
$$F=\frac{mg(\mu \cos\alpha- \sin\alpha)}{\cos\theta+\mu \sin\theta}$$We can determine when $F$ is at a minimum when the denominator is at a maximum. Taking the derivative of $\cos\theta + \mu\sin\theta$ and setting it to zero gives the minimum $F$ when $\tan\theta = \mu$.
\begin{center}
\begin{asy}
import olympiad;
size(5cm);
draw((0,0)--(4/3, 0), blue);
draw((0,1)--(4/3, 0), blue);
draw((0,0)--(0.48,0.64), blue);
draw((0,0)--(0, 1), blue);

dot((0,0), red);
dot((0,1), red);
dot((4/3,0), red);
dot((0.48,0.64),red);

pair A,B, C,D;
B = (0,0);
A = (0,1);
C = (4/3,0);
D = (0.48,0.64);

label("$\mu$", (1/2,0), 2S);
label("$\sqrt{\mu^2 + 1}$", (5/6,3/4), 2S);
label("$1$", (0,1/2), 1W);
draw(anglemark(B,A,C,4));
label("$\theta$",A,4*dir(270+aTan((5/9)/1)/2));

draw(anglemark(C,B,D,4));
label("$\theta$",B,4*dir(15+aTan((5/9)/1)/2));
\end{asy}
\end{center}
Using the large triangle, we can see that in this particular set-up, we have $\tan\theta = \mu$. Using the two smaller triangles, we can express the hypotenuse as $\cos\theta + \mu\sin\theta$, the hypotenuse of the expression above which happens to be $\sqrt{\mu^2+1}$ as well. Therefore, for the minimum force we can re-write it as:

$$\boxed{F_\text{min}=\frac{mg(\mu \cos\alpha- \sin\alpha)}{\sqrt{\mu^2+1}}}$$
which is equivalent (though not obviously) the same as the given answer. If we replace $\alpha=0$ we get:

$$F_\text{min} = mg \cdot \frac{\mu}{\sqrt{\mu^2+1}}$$

In the solution given above, we tried to find the maximum value of $\cos\theta+\mu \sin\theta$ by taking the derivative, then substituting it back in with a clever triangle. However, we can trivialize this step if we know that given:
$$f(x)=A\cos x+B\sin x$$the maximum value for $f(x)$ is:
$$\sqrt{A^2+B^2}$$which will give the intended result of$$\sqrt{1+\mu^2}$$

\tcbline
\textbf{Solution 2:} There are four forces: The gravitational force, the normal force, the applied force, and the friction force. We can break this up into two forces. Let
$$\vec{F}_1=\vec{N}+\vec{f}$$and let
$$\vec{F_2} = \vec{F} + m\vec{g}$$
The angle $F_1$ makes with the perpendicular is $\theta = \tan^{-1}(\mu)$
\begin{center}
\begin{asy}
import graph;
size(8cm); 
real labelscalefactor = 0.5; /* changes label-to-point distance */
pen dps = linewidth(0.7) + fontsize(10); defaultpen(dps); /* default pen style */ 
pen dotstyle = black; /* point style */ 
real xmin = -9.922742378906744, xmax = 9.390042121331888, ymin = -4.3853898879763165, ymax = 5.736880935613672;  /* image dimensions */

 /* draw figures */
draw((2,-2)--(-3,3)); 
draw((2,-2)--(1,-2)); 
draw((-1,1)--(0,2), linetype("2 2"),EndArrow(6)); 
draw((-1,1)--(-1,-2.4),  linetype("2 2"),EndArrow(6)); 
draw((0,2)--(-2,4),  linetype("4 4"),EndArrow(6)); 
draw((-1,1)--(-2,4),EndArrow(6)); 
draw((-1,-2.4)--(0,-2),  linetype("2 2"),EndArrow(6)); 
draw((-1,1)--(0,-2),EndArrow(6)); 
draw(shift((2,-2))*xscale(0.6615200786464106)*yscale(0.6615200786464106)*arc((0,0),1,135,180)); 
label("$\alpha$",(1.1,-1.5),SE*labelscalefactor); 
label("$N$",(-0.5628183119051274,1.5),SE*labelscalefactor); 
label("$\mu N$",(-0.943991690199311,3.48),SE*labelscalefactor); 
label("$F$",(-0.6475235070816127,-2.25),SE*labelscalefactor); 
label("$mg$",(-1.6216332516111926,-0.3901281821520958),SE*labelscalefactor); 
 /* dots and labels */
clip((xmin,ymin)--(xmin,ymax)--(xmax,ymax)--(xmax,ymin)--cycle); 
 /* end of picture */
\end{asy}
\end{center}
Secondly, $F_2$ must also fall upon the same line as $F_1$. In order for $F$ to be drawn such that it meets this requirement \textit{and} $F$ is at a minimum, we want $F$ to be perpendicular to the ray $F_1$ makes. Doing some angle tracing gives us $\phi=\theta-\alpha$ as the angle between $mg$ and the ray formed by $F_1$.
\vspace{2mm}

Separating $\vec{F}_2$ into its components gives:
$$\boxed{F = mg\sin\phi = mg\sin\left( \tan^{-1}(\mu)-\alpha\right)}$$
This is equivalent to the answer given above.
\end{solution}
