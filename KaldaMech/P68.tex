\begin{solution}{normal}
Consider the reference frame moving with velocity $v$ to the right. This frame is easier to work with because here the end of the rod on the ground is at rest. Let's call this end $A$. Let $t=0$ correspond to the time when the rod is vertical. At time $t$, the distance between $A$ and the vertical wall is $d=vt$. Let the angle between the rod and horizontal wall be
$$\theta = \frac{\pi}{2}-\alpha$$
such that
$$\cos \theta = \frac{vt}{2l}$$
Differentiating with respect to time gives
$$\dot \theta \sin \theta= \frac{v}{2l} \implies \dot \theta = \frac{v}{2l \sin \theta}$$

The velocity of the sphere is
$$\dot \theta (2l-x) = \frac{v(2l-x)}{2l \sin \theta}$$
perpendicular to the rod. Therefore, the $x-$ component of the velocity
$$v_x = \frac{v(2l-x)\sin \theta}{2l \sin \theta} = \frac{v(2l-x)}{2l }$$
is constant, implying that $a_x = 0$. Now the centripetal acceleration of the sphere
$$a_c= \dot \theta^2 (2l-x) = \frac{v^2(2l-x)}{4l^2 \sin^2 \theta}$$
which is pointed towards $A$. We know that the acceleration is entirely in $y$, as $a_x = 0$. As a result:
\begin{align*}
a \sin \theta &= \frac{v^2(2l-x)}{4l^2 \sin^2 \theta} \\
a &= \frac{v^2(2l-x)}{4l^2 \sin^3 \theta} \\
  &=\frac{v^2(2l-x)}{4l^2 \cos^3 \alpha}  \\
 &= \frac{v^2}{2l \cos^3 \alpha}(\frac{x}{2l}-1)
\end{align*}
which is pointed in negative $y$ direction. By Newton's second law, we have:
$$N = mg - ma \implies \boxed{N=m(g-\frac{v^2(2l-x)}{\sqrt2 l^2})}$$
\tcbline
\textbf{Solution 2:} We solve it for the general case. Let the place where both walls meet be the origin then we can write the coordinates of sphere as
$$X= x_1-x\sin\alpha$$
$$Y= y_1-x\cos\alpha$$
Now differentiating $Y$ with respect to time we get
$$ v_y=v\tan\alpha-v\tan\alpha\left(\frac{x}{2\ell}\right)$$
Now again differentiating it with respect to time we get
$$a_y=\omega v\sec^2\alpha\left(\frac{x}{2\ell}-1\right)$$
Also we have
$$\omega=\frac{v}{2\ell\cos\alpha}$$
Substituting it we get $$a_y=\frac{v^2}{2\ell}{\sec^3\alpha}\left(\frac{x}{2\ell}-1\right)$$
Now using Newton’s law in $y$ direction we get
$$4mg-N=ma_y$$
Solving we get
$$N=m\left(g-v^2\frac{(2\ell-x)}{l^2\sqrt2}\right)$$
Note that $a_x=0$. Since $\displaystyle v_x=v\left(1-\frac{x}{2l}\right)$ is constant, therefore the rod/sphere will apply no force on each other in the horizontal direction.
\end{solution}
