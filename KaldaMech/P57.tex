\begin{solution}{hard}
Let us consider the situation when the bug has travelled a distance $x$ from the upper end and is moving with a speed $v_b$. The stick always remains at rest, that is, making an angle $\alpha$ with the floor. When at a distance $\ell-x$ from the lower end of the rod, the torque on the bug about the bottom-most point (call this point O) is due to gravitational force on the bug and is equal to $$\vec{\tau}_O = \vec{r} \times \vec{F} = mg (\ell - x) \cos{\alpha} \ \hat{k}$$ At this moment, the angular momentum of the bug can be written as $$\vec{L}_O = mv\ell \sin{\alpha} \cos{\alpha} \ \hat{k}$$ So, we have $$\vec{\tau}_O= \frac{d \vec{L}_O}{dt} \implies  m\ell \sin{\alpha} \cos{\alpha} \frac{dv}{dt} = mg (\ell-x) \cos{\alpha}$$ $$\implies \frac{dv}{dt} = \frac{d^2x}{dt^2} = \frac{g}{\sin{\alpha}} (1-x/\ell) = -\frac{g}{\sin{\alpha}} (x-\ell) $$ Notice that the second derivative of position of the bug (with respect to the point O) is proportional to the negative of its distance from point O. But this resembles the equation of a simple harmonic motion being executed about the mean position O! So, as found above, we have $$a_{\text{bug}} = \boxed{ \frac{g}{\sin{\alpha}}(1-x/\ell)}$$ where $\sqrt{\frac{g}{l \sin{\alpha}}}$ can be considered to be the angular velocity of this hypothetical simple harmonic motion. The time taken by the bug to reach the bottom-most point is simply one-fourth the time period of the simple harmonic motion ($T_0$), so $$T_{\text{bug}} = \frac{T_0}{4} = \boxed{ \frac{\pi}{2} \sqrt{\frac{\ell \sin{\alpha}}{g}}}$$
\tcbline
\textbf{Solution 2:} The stick provides an acceleration $a$ to the bug, so the bug exerts a force $ma$ to the rod, pointing along the rod. In order to be stationary, we must have the normal force from the ground be $N_1=mg$. To balance torques about the bug, we have:
$$N_1\cos\alpha (\ell-x) = N_2\sin\alpha\ell \implies N_2 = mg\cot\alpha \left(1-x/\ell\right)$$The net force has to be zero, so we can add the vectors $N_1$, $N_2$, and $ma$ (which forms a right angle triangle). The horizontal component of the force the bug exerts on the rod $ma\cos\alpha$ has to balance out $N_2$ or:
$$ma\cos\alpha = mg\cdot\frac{\cos\alpha}{\sin\alpha}(1-x/\ell) \implies \boxed{a = \frac{g(1-x/\ell)}{\sin\alpha}}$$This can also be written as:
$$\ddot{x}=-\frac{g}{\ell\sin\alpha}x + \frac{g}{\sin\alpha}$$This gives the equation for simple harmonic motion with a period of:
$$T=2\pi\sqrt{\frac{\ell\sin\alpha}{g}}$$Travelling from the top to the bottom corresponds with one quarter of the period (maximum potential energy to maximum kinetic energy), so:
$$\boxed{t=\frac{\pi}{2}\sqrt{\frac{\ell\sin\alpha}{g}}}$$
\end{solution}
