\begin{solution}{normal}
\blfootnote{This problem was found in the book 'Aptitude Test Problems in Physics by S.S. Krotov.}
Let’s take the displacement $\xi$ of the wedge as coordinate describing the system’s position.\footnote{This is a solution that is based off the one given in hints, and is mainly expanding on some of the points that the hint did not give.} If the wedge moves by $\xi$, then the block moves the same amount with respect to the wedge because the rope is unstretchable. The kinetic energy changes by\[\Pi = mg\xi\sin\alpha.\]To find the velocity of the wedge and that of the block, let us add the two vectors $\dot\xi$ separated by an angle $\alpha$. The vertical components of the vector ($\dot\xi\sin\frac{\alpha}{2}$) cancel out due to symmetry. The horizontal components add up together to get $2\dot{\xi}\sin \frac{\alpha}{2}$. Thus, the velocity of the wedge is $\dot{\xi}$ and that of the block is $2\dot{\xi}\sin \frac{\alpha}{2}$, therefore the net kinetic energy is\[K=\frac{1}{2}\dot{\xi}^2\left(M+4M\sin^2 \frac{\alpha}{2}\right).\]Then we find $\Pi'(\xi) = mg \sin\alpha$ and $\mathcal{M} = M + 4m\sin^2\frac{\alpha}{2}$; their ratio gives the answer of
\[\boxed{a=\frac{mg \sin\alpha}{M + 4m\sin^2\frac{\alpha}{2}}}\]

\end{solution}
