\begin{solution}{normal}
The problem, in essence, is asking for how much the water level of the ocean changes $\Delta h$ due to the iron deposit. Let us choose the origin of the vertical x-axis to be a point on the surface of the ocean very far from the iron deposit. At $x = 0$, we take as the reference point to all other things in the system, and to the iron deposit, we take our reference point from far away such that the gravitational potential is zero. Let us consider when the iron deposit is initially not present. The potential due to this can be given as 
\[\varphi_1 = - \frac{G\rho \frac{4}{3}\pi r^3}{r + h}.\]
Now, consider when the iron deposit is put in the system. The water level increases by a height $\Delta h \ll h$ such that 
\[\varphi_2 = g\Delta h - \frac{G (\rho + \Delta \rho)\frac{4}{3}\pi r^3}{h + r}.\]
Equating both potentials (as it remains constant over time) gives us the expression 
\[gh - \frac{G\rho \frac{4}{3}\pi r^3}{r + h} = g(h + \Delta h) - \frac{G (\rho + \Delta \rho)\frac{4}{3}\pi r^3}{h + r}\implies \Delta h = \frac{\frac{4}{3}\pi G r^3\Delta \rho}{g(r + h)}.\]
\end{solution}
