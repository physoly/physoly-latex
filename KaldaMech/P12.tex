\begin{solution}{normal}
First, we'll look at the behavior of the tension at the bottom. The vertical component of the tension has to support the weight of the block so we have:
$$2T_\text{bottom,y} = 2T_\text{bottom}\cos(\beta/2) = Mg$$The horizontal component is thus:
$$T_x=T_\text{bottom}\sin(\beta/2) = \frac{Mg}{2}\tan(\beta/2)$$Notice that this horizontal tension force will be constant in a massive rope. If we look at a differential area of the string, the only other force other than tension is the gravitational force downwards. To balance horizontal forces, the horizontal components of tension have to be constant. At the top of the rope, the vertical component of the tension has to support the weight of the block and the string. We have:
$$2T_\text{top,y} = 2T_\text{top}\sin\alpha = (M+m)g$$The horizontal component will thus be:
$$T_x = T_\text{top}\cos\alpha = \frac{(M+m)g}{2}\cot\alpha$$
Setting these two expressions for the horizontal tension equal gives:
$$M\tan(\beta/2)=(M+m)\cot\alpha \implies \boxed{\beta = 2 \tan^{-1}\left(\left(1+\frac{m}{M}\right)\cot \alpha\right)}$$
\end{solution}
