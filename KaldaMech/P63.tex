\begin{solution}{normal}
First, let us move into an accelerated reference frame such that $M$ is stationary. The acceleration of $M$ is:
$$Ma=T-T\sin\alpha \implies a_M = T\left(\frac{1-\sin\alpha}{M}\right)$$
Thus, $m$ will have a fictitious force acting towards the right. The actual gravitational force and the fictitious force combine together to give us the effective gravity.

Now keep in mind that even in this accelerated reference frame, $m$ is not stationary. It is actually moving in the direction parallel to the rope holding it and due to conservation of rope, the acceleration of $m$ in this frame is $a_m=a_M$. Balancing forces, we get:
$$mg_\text{eff}-T=ma_M$$
Substituting in $g_\text{eff}=\frac{mg}{\cos\alpha}$ and $a_M$, we get:
$$\frac{mg}{\cos\alpha}-T=T\left(\frac{m}{M}\right)(1-\sin\alpha)$$
We can solve for $T$ to be:
$$T = \frac{mg}{\cos\alpha} \cdot \frac{1}{1+(m/M)(1-\sin\alpha)}$$
The ratio of the fictitious force and the gravitational force form a right angle, such that:
$$\tan\alpha = \frac{a_M}{g} = \frac{T}{mg}(1-\sin\alpha)$$
Substituting in $T$, cancelling out $\cos\alpha$ on both sides, and solving for $m/M$ (the algebra takes some time), we get:
$$\boxed{\frac{m}{M}}=\frac{\sin\alpha}{(1-\sin\alpha)^2}$$
\end{solution}
