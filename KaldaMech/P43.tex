\begin{solution}{normal}
Let us observe what happens to the work done at small changes of $da$ and $dh$.
\begin{center}
\begin{asy}
size(5cm);
draw((0,0)--(-4/3, 0), blue);
draw((0,1)--(-4/3, 0), blue);
draw((0,0)--(0, 1), blue);

dot((0,0), red);
dot((0,1), red);
dot((-4/3,0), red);
pair A,B, C;
B = (0,0);
A = (0,1);
C = (-4/3,0);

label("$da$", (-1/2,0), S);
label("$dl$", (-5/6,3/4), 2S);
label("$dh$", (0,1/2), 1E);
draw(anglemark(B, C, A ,4));
label("$\phi$",C+0.2, NE);
\end{asy}
\end{center}

The work due to friction will be
\[W_f=\mu mg\cos\phi dl\]Since $dl\cos\phi=da$ then
\[W_f=\mu mg da\]Integrating all of these small work variables from 0 to $a$ gives us the work produced by friction as
\[W_f=\mu mga\]The work produced by gravity is $mgh$ thus the total work $W_{\text{tot}}$ is
\[W_{\text{tot}}=mgh+\mu mga=\boxed{mg(h+\mu a)}\]
\end{solution}
