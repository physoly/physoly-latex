\begin{solution}{normal}
\begin{center}
\begin{asy}
import olympiad;
size(5cm);
draw(arc((0,0),(1,0), (-1,0)));
draw((1,0)--(-1,0));
pair A,B,C,D;
A = (3.1610980056834	,	-0.086367809191);
B = (-2.4770577190321	,	1.965753050762);
C = (3.5031181490091	,	0.8533248115948);
D = (-2.1350375757064	,	2.9054456715488);

pair P,L, CM;
P = (0.34202014332567, 0.93969262078591);
L=(0,1.06417777248);
CM = L+P/2;
draw((0,0)--P, dotted);
draw((0,0)--(0,1),dotted);
label("P", P, NE);
dot(P);

markscalefactor=0.1;
draw(anglemark(P, (0,0), L, 4));

label("$\theta$",(0,0.4), NE);
label("O", (0,0), S);
\end{asy}
\end{center}

Let the point at where the object slips of the hemisphere be $P$. We then have by energy conservation that 
\[
\frac{1}{2}mv^2 = mgR(1 - \cos\theta)
\] This implies that $v^2$ is 
\[
v^2 = 2gR(1- \cos\theta)
\] At any point on the circle we have the Newton’s third law pair of 
\[
F_g = F_c +F_N
\] However at the point where the object loses contact, the normal force becomes zero. This implies that
\[
mg\cos\theta = \frac{mv^2}{R}
\]
Taking out $m$ from both sides and substituting $v^2$ gives us
\begin{align*}
g\cos\theta &= \frac{2gR(1- \cos\theta)}{R}\\
2 - 2\cos\theta &= \cos\theta\\
\cos\theta &= \frac{2}{3}
\end{align*}
We know that $\cos\theta = \frac{h}{R}$, thus the height at which the object loses contact is $\boxed{h = \frac{2}{3}R}$
\end{solution}
