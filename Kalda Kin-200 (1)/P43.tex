\begin{solution}{normal}
Consider the following setup:
\begin{center}
    \begin{asy}
        unitsize(1.2cm);
        pair A = (0,0);
        pair B = (8,0);
        real a = 25*pi/180;
        real b = 60*pi/180;
        real v1 = 4;
        real v2 = 3;
        dot(A);
        dot(B);
        label("$A$",(-0.3,0));
        label("$B$",(8.3,0));
        draw("$l$",(0,-0.2)--(8,-0.2),Arrows,Bars);
        label("$\vec{v_1}$",(v1+0.3)*(cos(a),sin(a)));
        label("$\vec{v_2}$",B+(v2+0.3)*(-cos(b),sin(b)));
        draw(A--B);
        draw(A--v1*(cos(a),sin(a)),arrow=Arrow());
        draw(B--B+v2*(-cos(b),sin(b)),arrow=Arrow());
        draw(arc(A,0.7,0,a*180/pi));
        draw(arc(B,0.7,180,180-b*180/pi));
        label("$\alpha$",(cos(a/2),sin(a/2)));
        label("$\beta$",B+(-cos(b/2),sin(b/2)));
    \end{asy}
\end{center}
Now, we move into the reference frame of the boat that departed from harbour $A$.
\begin{center}
    \begin{asy}
        import olympiad;
        unitsize(1.2cm);
        pair A = (0,0);
        pair B = (8,0);
        real a = 25*pi/180;
        real b = 60*pi/180;
        real v1 = 4;
        real v2 = 3;
        dot(A);
        dot(B);
        label("$A$",(-0.3,0));
        label("$B$",(8.3,0));
        // draw("$l$",(0,-0.2)--(8,-0.2),Arrows,Bars);
        label("$\vec{v_1}$",B+(v1+0.3)*(-cos(a),-sin(a)));
        label("$\vec{v_2}$",B+(v2+0.3)*(-cos(b),sin(b)));
        draw(A--B);
        draw(B--B+v1*(-cos(a),-sin(a)),arrow=Arrow());
        draw(B--B+v2*(-cos(b),sin(b)),arrow=Arrow());
        draw(B--B+(v2*(-cos(b),sin(b))+v1*(-cos(a),-sin(a)))*1.7,dashed);
        draw(A--B+(v2*(-cos(b),sin(b))+v1*(-cos(a),-sin(a)))*1.508,dashed);
        draw(B--B+v2*(-cos(b),sin(b))+v1*(-cos(a),-sin(a)),arrow=Arrow());
        draw(arc(B,0.7,180,180+a*180/pi));
        draw(arc(B,0.7,180,180-b*180/pi));
        draw(arc(B,3.2,180,170));
        label("$\alpha$",B+(-cos(a/2),-sin(a/2)));
        label("$\beta$",B+(-cos(b/2),sin(b/2)));
        label("$\phi$",B+3.5*(-cos(5*pi/180),sin(5*pi/180)));
        label("$v_{rel}$",(B+v2*(-cos(b),sin(b))+v1*(-cos(a),-sin(a)))*1.2);
        draw(rightanglemark(A,B+(v2*(-cos(b),sin(b))+v1*(-cos(a),-sin(a)))*1.508,B));
        label("$d$",(-0.2,0.75));
    \end{asy}
\end{center}
Since $\vec{v_{rel}}=\vec{v_2}-\vec{v_1}$, we can separate them into components to find that
$$\tan\phi=\dfrac{\left| v_1\sin\alpha-v_2\sin\beta\right|}{\left| v_1\cos\alpha+v_2\cos\beta\right|}$$

From this, we can find that
$$\sin\phi=\dfrac{\left| v_1\sin\alpha-v_2\sin\beta\right|}{\sqrt{v_1^2+v_2^2+2v_1v_2\cos(\alpha+\beta)}}$$

We then have
\begin{align*}
d&=l\sin\phi\\
&=\boxed{\dfrac{l\cdot\left| v_1\sin\alpha-v_2\sin\beta\right|}{\sqrt{v_1^2+v_2^2+2v_1v_2\cos(\alpha+\beta)}}}
\end{align*}
\end{solution}